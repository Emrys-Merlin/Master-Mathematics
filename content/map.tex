\section{The boundary map}
\label{sec:map}

\subsection{The construction of the boundary map}
\label{sec:treminal-elements}

The construction of the actual map will be acomplished in this section. The main ingredient is that we need \(H_\mu\) to be empty for almost all \(\mu \in \mathcal{P}(\bar X)\). 

\begin{lemma}
  \label{lem:H=0}
  Let \(X\) be a finite-dimensional, irreducible, locally finite CAT(0) cube complex and \(\Gamma\) a group with an action \(\Gamma \to \Aut(X)\) that is essential and non-elementary. Furthermore, let \((B, \vartheta)\) be a strong \(\Gamma\)-boundary. If \(H_\mu = \varnothing\) for almost all \(\mu \in \mathcal{P}(\bar X)\) with regard to the pushforward measure from \(B\), then there exists a \(\Gamma\)-equivariant measurable map \(\phi\colon B \to \bar X\).
\end{lemma}

\begin{proof}
  By Corollary~\ref{cor:p(x)} we have a \(\Gamma\)-equivariant measurable map \(\psi\colon B \to \mathcal{P}(\bar X)\). Hence, we only need to find a map from \(\mathcal{P}(\bar X)\) to \(\bar X\). The central observation is, that if \(H_\mu= \varnothing\), then \(H_\mu^+\) is an ultrafilter. Indeed, since \(\mathcal{H} = H_\mu^+ \sqcup H_\mu \sqcup H_\mu^-\) and \((H_\mu^+)^\ast = H_\mu^-\) (c.\,f.~Lemma~\ref{lem:4.6}), we have the choice condition. For the consistency condiction we only need to know, that \(\mathfrak{h} \subset \mathfrak{k}\) implies \(\mathcal{C}(\mathfrak{h}) \subset \mathcal{C}(\mathfrak{k})\) and hence \(\mu(\mathcal{C}(\mathfrak{h}))\leq \mu(\mathcal{C}(\mathfrak{k}))\).

  By assumption we know that the image of \(\psi\) lies up to measure 0 in the set \(\mathcal{E} \coloneqq \{\mu \in \mathcal{P}(\bar X) \mid H_\mu = \varnothing\}\). We can hence concatenate it with the map
  \begin{align*}
    \xi\colon \mathcal{E} &\to \bar X,\\
    \mu &\mapsto H_\mu^+.
  \end{align*}
  This map is measurable thanks to Lemma~\ref{lem:measurable-mu} applied to the case \({(\text{\nfrac 1/2}, 1]}\) and \(\Gamma\)-equivariant.

  All in all we have that \(\xi \circ \psi\) is our desired map \(\phi\).
\end{proof}

As a next step to our main theorem we would like to convince ourselves that any map of the above type not only takes values in \(\bar X\), but in \(\partial X\). This will be accomplished with the next lemma.

\begin{lemma}[{\cite[Lemma~4.11]{MR3509968}}]
  \label{lem:4.11}
  Let \(X\) be a finite dimensional, locally countable CAT(0) cube complex, \(\Gamma \to \Aut(X)\) an essential action on \(X\), \((B, \nu)\) a doubly ergodic \(\Gamma\)-space with quasi-invariant measure \(\nu\) and \(\phi\colon B \to \bar X\) a measurable \(\Gamma\)-equivariant map. Then \(\phi\) takes values in the non-terminating ultrafilters of \(X\).
\end{lemma}

\begin{proof}
  We may first assume that \(X\) is irreducible and consider the map
  \[
    B \to \N \cup \{\infty\},\ x \mapsto |\tau(\phi(x))|,
  \]
  which is measurable and \(\Gamma\)-invariant. Hence by ergodicity it is essentially constant with essential value \(M\). If we can show, that \(M = 0\), we are done as then the image of \(\phi\) essentially contains only non-terminating ultrafilters.

  For this purpose let us consider the following map
  \[
    B \times B \to \N \cup \{\infty\},\ (x,y) \mapsto |\tau(\mathcal{H}(\phi(x), \phi(y)))|,
  \]
  which is again measurable as a composition of measurable maps (consider the previous lemma) and \(\Gamma\)-invariant and again we yield an essential value \(N\). By Remark~\ref{rem:interval} we have that \(N < \infty\) and hence \(\tau(\mathcal{H}(\phi(x), \phi(y)))\) takes values in \(\operatorname{Pot}_f(\mathcal{H})\). By Corollary~\ref{cor:4.5} this would mean that the action of \(\Gamma\) is inessential, unless \(N\) is essentially 0.

  Lastly, we will show that this is incompatible with the case that \(M > 0\). Assume that \(M > 0\) were the case, then we could find a \(x_0 \in B\) such that \(|\tau(\phi(x_0)) > 0\) and a set \(B_0 \subset B\) of full measure, such that \(\tau(\mathcal{H}(\phi(x_0), \phi(y))) = \varnothing\) for all \(y \in B_0\). By Lemma~\ref{lem:4.12} we have that for all \(\mathfrak{h} \in \tau(\phi(x_0))\), we have \(\mathfrak{h} \in \phi(y)\).

  However, \(B_0\) contains a \(\Gamma\)-orbit and hence this contradicts the fact that the action is essential.\todo{Understand and prove this last sentence.}
  
  \todo{make reducible case nicer}
  In the case that \(X\) is reducible we can decompose it into its irreducible factors and we find a finite index subgroup of \(\Gamma\), which respects this decomposition. This subgroup will still act ergodically on each of the factor. Thus using each projection the factors we get that the image of each separate \(\phi\) lands in the non-terminating ultrafilters. Since \(\bar X\) consists of the Cartesian product of the ultrafilters of the factors, we see that also in that case the image lands in the non-terminating ultrafilters.
\end{proof}

So far we have seen, that if \(H_\mu\) is empty for almost all \(\mu\), we find our desired map with all the necessary properties. So we are left with the task to prove, that this is the only possible case. As first step is provided in the following two lemmas:

\begin{lemma}
  \label{lem:h-const}
  The map
  \begin{align*}
    \mathcal{P}(\bar X) &\to \N \cup \{\infty\},\\
    \mu &\mapsto |H_\mu|
  \end{align*}
  is essentially constant.
\end{lemma}

\begin{proof}
  The map is \(\Gamma\)-equivariant and measurable as a concatenation of measurable maps (c.f.~Section~\ref{sec:meas-maps}). Since the group action is ergodic with regard to the pushforward measure on \(\mathcal{P}(\bar X)\) from \(B\), we see that the map is essentially constant.
\end{proof}

\begin{lemma}
  \label{lem:finite-zero}
  If \(|H_\mu|\) is essentially constant and not infinite then \(H_\mu\) is empty for almost all \(\mu \in \mathcal{P}(\bar X)\).
\end{lemma}

\begin{proof}
  We consider the map
  \begin{align*}
    \mathcal{P}(\bar X) & \to \operatorname{Pot}_f(\mathcal{H}),\\
    \mu & \mapsto H_\mu.
  \end{align*}
  This map is measurable and \(\Gamma\)-equivariant. Hence, by Corollary~\ref{cor:4.5}, we know that its image has to be essentially \(\varnothing\) in order for our \(\Gamma\)-action to be essential.
\end{proof}

All in all we see that, if we can show that \(|H_\mu|\) cannot be infinite for almost all \(\mu \in \mathcal{P}(\bar X)\), we are done. However, this is the most involved case and will need the next two sections, but first we need one last lemma:

\begin{lemma}
  \label{lem:hh-const}
  The map
  \begin{align*}
    \mathcal{P}(\bar X) \times \mathcal{P}(\bar X) &\to \N \cup \{\infty\},\\
    (\mu, \nu) &\mapsto |H_\mu \cap H_\nu|
  \end{align*}
  is essentially constant.
\end{lemma}

\begin{proof}
  This map is again measurabel and \(\Gamma\)-equivariant (c.\,f.~Section~\ref{sec:meas-maps}) and hence essentially constant by the doubly ergodic action of \(\Gamma\) on \(\mathcal{P}(\bar X)\).
\end{proof}

The following two sections will contain all the technical details in order to exculde the cases \(|H_\mu| = \infty\) and \(|H_\mu \cap \H_\nu| = 0\) or \(|H_\mu \cap H_\nu| = \infty\) for almost all \(\mu, \nu \in \mathcal{P}(\bar X)\). In the last section we will plug everything together and prove our main theorem.

\subsection{Exclusion of the case \(|H_\mu \cap H_\nu| = 0\)}
\label{sec:M=0}

Here we will see that \(|H_\mu| = \infty\) and \(|H_\mu \cap H_\nu| = 0\) for almost all \(\mu, \nu \in \mathcal{P}(\bar X)\) would imply that our cube complex is not finite-dimensional.

\begin{lemma}[{\cite[Lemma~4.13]{MR3509968}}]
  \label{lem:4.13}
  Let \((\mu_i)_{i \in \N_0}\) be a sequence of pairwise generic probability measures on \(\mathcal{P}(\bar X)\) (i.\,e.\ \(H_{\mu_i} \cap H_{\mu_j} = \varnothing\) whenever \(i \neq j\)), such that for each \(i > 0\) there exists an infinite descending chain \(\mathfrak{h}_n^i \in H_{\mu_0}^+ \cap H_{\mu_i}\). Then up to switching \(i\) and \(j\), any pair of measures \(\mu_i\) and \(\mu_j\), satisfies the following condition:

  There exists \(C(i,j) \in \N\) such that for every \(n \geq C(i,j)\) there is an \(M_n \geq C(i,j)\) such that if \(m \geq M_n\) then \(\mathfrak{\hat h}^j_m \pitchfork \mathfrak{\hat h}^i_n\). 
\end{lemma}

\begin{proof}
  We fix two measures and call them \(\mu\) and \(\nu\). Let \(\mathfrak{h}_n \in H_{\mu_0}^+ \cap H_\mu\) and \(\mathfrak{k}_m \in H_{\mu_0}^+ \cap H_\nu\) be the corresponding infinite descending sequences.

  We can then construct a partition of \(\N \times \N = N_1 \sqcup N_2 \sqcup N_3 \sqcup N_4\) via
  \begin{align*}
    N_1 & \coloneqq \{(n,m) \in \N^2 \mid \mathfrak{h}_n \pitchfork \mathfrak{k}_m\},\\
    N_2 & \coloneqq \{(n,m) \in \N^2 \mid \mathfrak{h}_n^\ast \subset \mathfrak{k}_m\},\\
    N_3 & \coloneqq \{(n,m) \in \N^2 \mid \mathfrak{h}_n \subset \mathfrak{k}_m\}\quad \text{and}\\
    N_4 & \coloneqq \{(n,m) \in \N^2 \mid \mathfrak{h}_n \supset \mathfrak{k}_m\}.
  \end{align*}
  However, we would like to show that up to leaving out finitely many elements of \(\N^2\), we can find an easier decomposition, namely there exists \(C \in \N\) such that
  \[
    \N_C \coloneqq \N \times \N \cap ([C, \infty) \times [C, \infty)) = N_1 \sqcup N_j,
  \]
  where \(j \in \{2,3,4\}\). We will consider different cases, depending on which \(N_j\) are not empty. If at most one of the \(N_j\ (j = 2,3,4)\)  is not empty, we have nothing to prove. So we turn towards the cases, where at most two of the sets are non-empty.
  \begin{description}
  \item[\(N_2 \neq \varnothing \neq N_3\):] We can take \((n_3, m_3) \in N_3\) and \((n,m) \in N_2\) and define \(m' \coloneqq \min\{m, m_3\}\). Then we have the following two inclusions
    \begin{align*}
      \mathfrak{h}_{n_3} & \subset \mathfrak{k}_{m_3} \subset \mathfrak{k}_{m'} \quad \text{and}\\
      \mathfrak{h}_{n}^\ast & \subset \mathfrak{k}_m \subset \mathfrak{k}_{m'}.
    \end{align*}
    If we have \(n \geq n_3\), we have \(\mathfrak{h}_n \subset \mathfrak{h}_{n_3} \subset \mathfrak{k}_{m'}\), which is impossible by the second inclusion above. Hence, fixing \(m_3\) we define
    \[
      A \coloneqq \min \{n_3 \in \N \mid (n_3, m_3) \in N_3\}
    \]
    and see that for all \(n \geq A\) and any \(m \in \N\) we have \((n,m) \notin \N_2\) or in other words we have
    \[
      \{(n,m) \in N_2 \mid n \geq A_3\} = \varnothing.
    \]
    If \(N_4\) is empty, this is already sufficient to show that \(\N_A = N_1 \sqcup N_3\). The case, when all three are not empty, will be handled below.
  \item[\(N_3 \neq \varnothing \neq N_4\):] Again we take \((n_3, m_3) \in N_3\) and \((n,m) \in N_4\). If \(n \geq n_3\), we have
    \[
      \mathfrak{k}_n \subset \mathfrak{h}_n \subset \mathfrak{h}_{n_3} \subset \mathfrak{k}_{m_3}.
    \]
    This would imply that \(\mathfrak{h}_n \in H_\mu \cap H_\nu\), since it is enclosed in two halfspaces which lie inf \(H_\nu\), however, we have \(H_\mu \cap H_\nu = \varnothing\). Hence, we define analogously to before
    \[
      B \coloneqq \min \{n_3 \in \N \mid (n_3, m_3) \in N_3\}
    \]
    and see that if \(n \geq B\), then for any \(m \in \N\) we have \((n,m) \notin N_4\). If \(N_2\) is empty, this implies \(\N_B = N_1 \sqcup N_3\).
  \item[\(N_2 \neq \varnothing \neq N_4\):] This case is completely analogous to the first with the sole difference being that we interchange the roles of \(n\) and \(m\). Hence we can define a constant
    \[
      D \coloneqq \min \{m_4 \mid (n_4, m_4) \in N_4\},
    \]
    where \(n_4 \in \N\) was fixed such that the above set is not empty. We yield that if \(m \geq B\) and \(n \in \N\) is arbitrary, we have \((n, m) \notin N_2\). As above, if \(N_3 \) is empty, this is sufficient to show that \(\N_D = N_1 \sqcup N_4\).
  \item[\(N_j \neq \varnothing\ \forall j \in\{2,3,4\}\):] In this case we can use the constants defined above and set \(C \coloneqq \max \{A,B,D\}\). Then if \((n,m) \in \N_C \setminus N_1\), we have \(m \geq B\) and hence \((n,m) \notin N_2\) and \(n \geq D\) meaning \((n,m) \notin N_4\). All in all this leads to \(\N_C = N_1 \sqcup N_3\).
  \end{description}
  Now that we have the above decomposition, we will prove the desired claim.
  \begin{description}
  \item[\(\N_C = N_1 \sqcup N_3\):] If \(N_3 \neq \varnothing\), we take \((n_0, m_0) \in N_3\) and define
    \begin{align*}
      M = M(n_0)  \coloneqq & \max \{m \in \N \mid (n_0, m) \in N_3\}\\
                      = & \max \{m \in \N \mid \mathfrak{h}_{n_0} \subset \mathfrak{k}_m \subset \mathfrak{k}_{m_0}\}.
    \end{align*}
    \(M\) is well-defined since the maximum is taken over a non-empty set (by choice of \((n_0, m_0)\)) and the set is finite since two nested halfspaces contain only finitely many halfspaces in between (c.\,f.\ Lemma~\ref{lem:finite-interval}).

    We see that if \(m > M\), we have \((n_0, m) \in N_1\), which is what we wanted.
  \item[\(\N_C = N_1 \sqcup N_2\):] This case works completely analogous, with \(\mathfrak{h}_{n_0}\) replaced by \(\mathfrak{h}_{n_0}^\ast\).
  \item[\(\N_C = N_1 \sqcup N_4\):] This is the only case where the switching of the indices comes into play. Indeed if we interchange the roles of \(\mathfrak{h}_n\) and \(\mathfrak{k}_m\), we find ourselves again in the above handled cases and are done.
  \end{description}
\end{proof}

\begin{defin}
  A \emph{directed graph \(\mathcal{G}(V,E)\)} consists of two sets \(V\) and \(E\) which are denoted its \emph{vertex set} and \emph{edge set} respectively and of two maps \(s, t \colon E \to V\) associating to each edge its source and target respectively. In our case there are no parallel edges allowed (antiparallel edges may occur) and we will not allow loops. The first condition allows us to think of \(E \subset V \times V\) and we will prefer writing \(\overline{vw} \in E\) for two vertices \(v,w \in V\). This has the further advantage of making the maps \(s\) and \(t\) obsolete. We will call \(\mathcal{G}\) \emph{complete}, if it is complete as an undirected graph, i.\,e.\ each vertex is connected via an (undirected) edge to each other vertex. For each \(v \in V\) we will by \(o(v) \subset E\) denote the number of \emph{outgoing edges} and by \(i(v) \subset E\) the number of \emph{incoming edges}. We will call a complete directed finite graph \emph{upper triangular}, if there exists an enumeration \(V = \{v_1, \dots, v_D\}\), such that for all \(j = 1, \dots, D\) we have
  \begin{align*}
    o(v_j) & = D - j\quad \text{and}\\
    i(v_j) & = j - 1.
  \end{align*}
\end{defin}

\begin{lemma}[{\cite[Lemma~A.6]{MR3509968}}]
  If \(\mathcal{G} \coloneqq \mathcal{G}(V,E)\) is a complete directed finite graph with \(|V| = D\) then there exists a \(v \in V\) such that \(o(v) \geq \frac{D-1}{2}\).
\end{lemma}

\begin{proof}
  Since \(\mathcal{G}\) is complete we have \(o(v) + i(v) \geq D - 1\) and summing over all vertices we yield
  \[
    \sum_{v \in V} o(v) + i(v) \geq D(D-1).
  \]
  Since all edges that start somewhere have to end somewhere, we have:
  \[
    \sum_{v \in V} o(v) = \sum_{v \in V} i(v),
  \]
  leading to
  \[
    \sum_{v \in V} o(v) \geq \frac{D(D-1)}{2}.
  \]
  If \(o(v)\) were smaller than \(\frac{D-1}{2}\) for each \(v \in V\), we would have that
  \[
    \sum_{v \in V} o(v) < \frac{D(D-1)}{2},
  \]
  which is a contradiction. Hence there exists at least one \(v \in V\) such that \(o(v) \geq \frac{D-1}{2}\)
\end{proof}

\begin{lemma}[{\cite[Lemma~A.8]{MR3509968}}]
  \label{lem:A.8}
  Let \(\mathcal{G} = \mathcal{G}(V,E)\) be a (not necessarily finite) complete directed graph and \(D \in \N\). If \(|V| \geq 5^D\), there exist \(D\) vertices \(v_1, \dots, v_D\) and a subset \(E_D \subset E\) such that \(\mathcal{G}(\{v_1, \dots, v_D\}, E_D)\) is complete, directed and strictly upper triangular.
\end{lemma}

\begin{proof}
  We will prove this by induction, but need a slightly stronger statement. We will prove:

  Let \(N \in \N\) and \(|V| \geq 5^N\). Then for each \(D \leq N\) there exist \(v_1,\dots, v_D \in V\) and a subset \(E_D \subset E\) such that \(\mathcal{G}(\{v_1, \dots, v_d\}, E_D)\) is complete directed and strictly upper triangular. Furthermore, for the set
  \[
    V_D \coloneqq \{v \in V \setminus \{v_1, \dots, v_d\} \mid \overline{v_iv} \in E\ \forall i =1,\dots, d\}
  \]
  we have \(|V_D| \geq 5^{N-D}\)

  Observe that it is sufficient to prove this statement for finite graphs, since for infinite graphs, we can always consider a finite sub graph with sufficiently many vertices. Thus we will proceed with the finite case.
  \begin{description}
  \item[\(D = 1\):] By the previous lemma we can find a \(v_1 \in V\) such that
    \[
      o(v_1) \geq \frac{|V| - 1}{2} \geq \frac{V}{5} \geq 5^{n-1}.
    \]
    Then \(\mathcal{G}(\{v_1\}, \varnothing)\) is clearly complete and upper triangular. Furthermore, we have that
    \[
      V_1 = \{v \in V \mid \overline{v_1v} \in E\},
    \]
    since \(\mathcal{G}\) does not contain loops. However, then we have \(|V_1| = |o(v_1)| \geq 5^{N-1}\) and we are done.
  \item[\(D \to D+1\):] By the induction hypothesis we find \(\{v_1, \dots, v_D\}\) and a subset \(E_D \subset E\) such that \(\mathcal{G}(\{v_1, \dots, v_D\}, E_D)\) is complete and strictly upper triangular and \(|V_D| \geq 5^{N-D}\). We consider the complete graph induced by \(\mathcal{G}\) on the set \(V_D \neq \varnothing\). Again by the previous Lemma, we find a vertex \(v_{D+1} \in V_D\) such that
    \[
      o(v_{D+1}) \geq \frac{|V_D| -1}{2} \geq \frac{|V_D|}{5} \geq 5^{N - (D+1)}.
    \]
    By construction this vertex is connected via an incoming edge to each of the \(v_i\). If we set
    \[
      E_{D+1} = E_D \cup \{\overline{v_iv_{D+1}} \mid \forall i = 1, \dots, D\},
    \]
    then \(\mathcal{G}(\{v_1, \dots, v_{D+1}\}, E_{D+1})\) is still complete and strictly upper triangular by construction. Additionally, we have
    \[
      V_{D+1} = \{v \in V_D \mid \overline{v_{D+1}v} \in E\}
    \]
    and thus \(|V_{D+1}| = |o(v_{D+1})| \geq 5^{N- (D+1)}\), which completes the induction.
  \end{description}
\end{proof}

\begin{prop}[{\cite[Proposition~4.10]{MR3509968}}]
  \label{prop:4.10}
  If for almost all \(\mu, \nu \in \mathcal{P}(\bar X)\) we have:
  \begin{itemize}
  \item \(|H_\mu| = |H_\nu| = \infty\),
  \item \(H_\mu \cap H_\nu = \varnothing\) and
  \item \(\tau(H_\mu \cap H_\nu^+) = \varnothing\)
  \end{itemize}
  then \(X\) contains cubes of arbitrarily large dimension.
\end{prop}

\begin{proof}
  Since \(H_\mu \cap H_\nu^+\) has no minimal elements for almost all \(\mu, \nu\), we can find a sequence \((\mu_i)_{i \in I}\) such that \(H_{\mu_0}^+ \cap H_{mu_i}\) contains an infinite descending chain, which we denote by \((\mathfrak{h}^i_n)_{n \in \N}\). Thus we can apply Lemma~\ref{lem:4.13} and find \(C(i,j) \in \N\) such that for all \(n \geq C(i,j)\) there is an \(M \geq c(i,j)\) such that if \(m > M_n\) we have \(\mathfrak{\hat h}_n^i \pitchfork \mathfrak{\hat h}_m^j\) (after possibly switching \(i\) and \(j\)).

  Using this we can construct the following graph \(\mathcal{G} \coloneqq \mathcal{G}(V,E)\) with \(V \coloneqq \{\mu_i \mid i \in I\}\) and an edge from \(\mu_i\) to \(\mu_j\) if and only if the above mentioned \(C(i,j)\) exists. With this \(\mathcal{G}\) becomes a directed graph and since the searched \(C(i,j)\) exists either for \((i,j)\) or \((j,i)\) it is complete. Hence, we can apply Lemma~\ref{lem:A.8} using any \(D \in \N\) and find (after relabeling) \(\mu_1, \dots, \mu_D \in V\) such that we find a subset of edges \(E_D\) such that \(\mathcal{G}(V,E)\) becomes strictly upper triangular. In this case this implies, that for each \(1\leq i < j \leq D\) there exists an above \(C(i,j)\). We set
  \begin{align*}
    C &\coloneqq \max\{C(i,j) \mid 1 \leq i < j \leq D\} \quad \text{and}\\
    M &\coloneqq \max\{M_C(i,j) \mid 1 \leq i < j \leq D\}.
  \end{align*}
  Fixing \(n = C\) and for each \(m \geq M\) we have that \(\mathfrak{\hat h}_n^i \pitchfork \mathfrak{\hat h}_m^j\) for each \(1 \leq i < j \leq D\). Fixing \(m\) this leads to a set of \(D\) transverse halfspaces. Since these intersect pairwise, we know (Theorem~\ref{thm:common-intersection}) that the common intersection is not empty and an element in this intersection is in a cube, which has all these hyperplanes as midcubes. Hence this cube has at least dimension \(D\). Since \(D\) was arbitrary this shows, that \(X\) contains cubes of arbitrary dimension.
\end{proof}

\subsection{Exclusion of the case \(|H_\mu \cap H_\nu| = \infty\)}
\label{sec:M=infty}

Here we will see that \(|H_\mu| = \infty\) and \(|H_\mu \cap H_\nu| = \infty\) would imply that the \(\Gamma\)-action is not essential.

\begin{defin}
  Let \(Y \subset X\) be a subcomplex. Then \(Y\) is called \emph{strongly convex}, if for every two vertices in \(Y\) we have that every shortest edge path joining the two also lies completely in \(Y\).
\end{defin}

\begin{lemma}[{\cite[Lemma~4.16]{MR3509968}}]
  \label{lem:4.16}
  Let \(X\) be a CAT(0) cube complex and \(A \subset X\) consists of (not necessarily connected) cubes. If \(Y\) is the smallest strongly convex subcomplex of \(X\) containing \(A\), then
  \[
    \mathcal{\hat H}(Y) = \mathcal{\hat H}(A) \sqcup \{\mathfrak{\hat h} \in \mathcal{\hat H}(Y) \mid \mathfrak{\hat h} \text{ separates } A \text{ in at least two non-trivial subsets}\}.
  \]
\end{lemma}

\begin{proof}
  Let \(\mathfrak{\hat h} = (\mathfrak{h}, \mathfrak{h}^\ast) \in \mathcal{\hat H}(Y) \setminus \mathcal{\hat H}(A)\). If \(A\) were in \(\mathfrak{h}\), then any geodesic between two points of \(A\) is also contained in \(h\), otherwise the geodesic would cross \(\mathfrak{\hat h}\) twice. Hence, \(Y \subset \mathfrak{h}\), as \(Y\) is the smallest strongly convex subcomplex of \(X\) containing \(A\). However, this is a contradiction to the fact that \(\mathfrak{\hat h} \in \mathcal{\hat H}(Y)\).
\end{proof}

\begin{prop}[{\cite[Proposition~4.17]{MR3509968}}]
  \label{prop:4.17}
  Let \(X\) be a CAT(0) cube complex and \(\Gamma \to \Aut(X)\) an essential action. Let \(\mathcal{H}' \subset \mathcal{H}(X)\) be a \(\Gamma\)-invariant subset of halfspaces and \(X_\alpha \subset X\) a \(\Gamma\)-invariant family (with more than one member) of subcomplexes such that \(\mathcal{\hat H}(X_\alpha) = \mathcal{\hat H}'\). Let \(Y\) be the smallest strongly convex subcomplex containing \(A \coloneqq \cup_\alpha X_\alpha\). Then \(Y = X\) and \(\bar X = \overline{X(\mathcal{\hat H}')} \times Z\), where \(Z\) is the Roller compactification of some other CAT(0) cube complex.
\end{prop}

\begin{proof}
  Let \(v \in Y\). Then either \(v \in A\), which is \(\Gamma\)-invariant, or \(v\) lies on a geodesic joining two vertices of \(A\). Since this geodesic is mapped to another geodesic joining two vertices of \(A\) under \(\Gamma\), we again have that \(v\) stays in \(Y\) under the \(\Gamma\)-action. All in all \(Y\) is a \(\Gamma\)-invariant set. The same is true for \(X \setminus Y\). Furthermore, \(X\) is connected and we can find \(v \in Y\) and \(w \in X \setminus Y\), such that the two are connected by an edge. Let \(\mathfrak{\hat h} = \mathfrak{\hat h}(vw)\) with \(v \in \mathfrak{h}\) and \(w \in \mathfrak{h}^\ast\). Since \(Y\) is strongly convex, we have that \(Y \subset \mathfrak{h}\). Now, for each \(g \in \Gamma\), we have \(gv \in Y\) and \(gw \in X \setminus Y\). Since \(g\) acts by combinatorial maps, the two vertices are still joined by an edge. However, this means that \(d(\Gamma w, \mathfrak{\hat h}) = 1\), which is a contradiction to the fact that \(\Gamma\) acts essential. Thus \(Y = X\).

  By the previous lemma, we know that the hyperplanes of \(X\) decompose as the ones contained in \(\mathcal{\hat H'} = \mathcal{\hat H}(A)\) and the ones separating one \(X_\alpha\) from another \(X_{\alpha'}\). Consider any \(\mathfrak{\hat h} \in \mathcal{\hat H'}\) and any \(\mathfrak{\hat k}\) separating two above mentioned copies. We need to show that these hyperplanes are transverse. Since \(\mathfrak{\hat h}\) separates \(X_\alpha\) we can find two \(x, x' \in X_\alpha\) such that \(x \in \mathfrak{h}\) and \(x' \in \mathfrak{h}^\ast\). Analogously, we find two vertices \(y,y' \in X_{\alpha'}\). However, then \(x,x' \in \mathfrak{k}\) and \(y,y' \in \mathfrak{k}^\ast\) (or vice versa) and we have that the four intersections \(\mathfrak{h} \cap \mathfrak{k},\ \mathfrak{h} \cap \mathfrak{k}^\ast,\ \mathfrak{h}^\ast \cap \mathfrak{k}\) and \(\mathfrak{h}^\ast \cap \mathfrak{k}^\ast\) are non-empty. Thus \(\mathcal{H}\) splits in two transverse subsets and hence \(X\) splits as the required product.
\end{proof}

\begin{lemma}[{\cite[Lemma~4.18]{MR3509968}}]
  \label{lem:4.18}
  If \(|H_\mu| = \infty\)\todo{check if I need this first condition} and \(H_\mu\) contains strongly separated halfspaces (c.\,f.~Definition~\ref{defin:strong-sep}), then \(H_\mu^+\) satisfies the descending chain condition.
\end{lemma}

\begin{proof}
  Let \(\mathfrak{h, k} \in H_\mu\) with \(\mathfrak{h} \subset \mathfrak{k}\) strongly separated and define
  \[
    P(\mathfrak{h}) \coloneqq \{\mathfrak{l} \in H_\mu^+ \mid \mathfrak{h} \parallel \mathfrak{l}\}.
  \]
  Because of the strong separation each \(\mathfrak{l} \in \mathcal{H}\) is parallel to \(\mathfrak{h}\) or \(\mathfrak{k}\). Hence, we have
  \[
    H_\mu^+ = P(\mathfrak{h}) \cup P(\mathfrak{k}).
  \]
  Next, we consider a descending chain \(\mathfrak{h_1} \supset \mathfrak{h_2} \supset \dots\) in \(H_\mu^+\). By going over to a subsequence, we can assume that all halfspaces lie in either \(P(\mathfrak{h})\) or \(P(\mathfrak{k})\). Without loss of generality, we choose \(P(\mathfrak{h})\). \(\mathfrak{h}_n \subset \mathfrak{h}\) and \(\mathfrak{h}_n \subset \mathfrak{h}^\ast\) cannot happen, as \(\mu(\mathfrak{h}) = \mu(\mathfrak{h}^\ast) < \mu(\mathfrak{h}_n)\). In the case that \(\mathfrak{h} \subset \mathfrak{h}_n\), we know by Lemma~\ref{lem:finite-interval} that there are only finitely many halfspaces between the two. Hence, the chain must terminate. The same argument holds in the case \(\mathfrak{h}^\ast \subset \mathfrak{h}_n\).
\end{proof}


\begin{lemma}
  \label{lem:strongly-sep}
  Let \(\mathfrak{h} \subset \mathfrak{k} \subset \mathfrak{l}\) be three halfspaces in \(\mathcal{H}\) and \(g \in \Gamma\). Then
  \begin{enumerate}
  \item if \((\mathfrak{h}, \mathfrak{k})\) or \((\mathfrak{k}, \mathfrak{l})\) are strongly separated, then the same is true for \((\mathfrak{h}, \mathfrak{l})\) and
  \item \((\mathfrak{h}, \mathfrak{k})\) is strongly separated if and only if the same is true for \((g\mathfrak{h}, g\mathfrak{k})\).
  \end{enumerate}
\end{lemma}

\begin{proof}
  \begin{enumerate}
  \item Assume that \(\mathfrak{m}\) were transverse to both \(\mathfrak{h}\) and \(\mathfrak{l}\). Then we would have
    \begin{align*}
      \mathfrak{m}^{\phantom{\ast}} \cap \mathfrak{k}^{\phantom{\ast}} & \supset \mathfrak{m}^{\phantom{\ast}} \cap \mathfrak{h}^{\phantom{\ast}} \neq \varnothing,\\
      \mathfrak{m}^\ast \cap \mathfrak{k}^{\phantom{\ast}}  & \supset \mathfrak{m}^\ast \cap \mathfrak{h}^{\phantom{\ast}} \neq \varnothing,\\
      \mathfrak{m}^{\phantom{\ast}} \cap \mathfrak{k}^\ast & \supset \mathfrak{m}^{\phantom{\ast}} \cap \mathfrak{l}^\ast \neq \varnothing\quad \text{and}\\
      \mathfrak{m}^\ast \cap \mathfrak{k}^\ast & \supset \mathfrak{m}^\ast \cap \mathfrak{l}^\ast \neq \varnothing.
    \end{align*}
    Hence, \(\mathfrak{m}\) were transverse to \(\mathfrak{k}\), which is a contradiction to the strong separateness of either \((\mathfrak{h}, \mathfrak{k})\) or \((\mathfrak{k}, \mathfrak{l})\).
  \item If \((\mathfrak{h}, \mathfrak{k})\) are strongly separated and we have any \(\mathfrak{m} \in \mathcal{H}\). Then \(g^{-1}\mathfrak{m} \in \mathcal{H}\) and it is parallel to either \(\mathfrak{h}\) or \(\mathfrak{k}\). Hence, \(\mathfrak{m}\) is parallel to either \(g\mathfrak{h}\) or \(g\mathfrak{k}\). The opposite direction is analogous.
  \end{enumerate}
\end{proof}

\begin{lemma}[{\cite[Lemma~4.19]{MR3509968}}]
  \label{lem:4.19}
  Let \(X\) be a finite-dimensional, irreducible, CAT(0) cube complex with an essential and non-elementary group action \(\Gamma \to \Aut(X)\). For every measure \(\mu\) either \(\hat H_\mu\) contains a pair of strongly separated hyperplanes or there exists a pair \(\mathfrak{h} \in H_\mu^-\) and \(\mathfrak{k} \in H_\mu^+\) of halfspaces, such that the hyperplanes \(\mathfrak{\hat h}\) and \(\mathfrak{\hat k}\) are strongly separated and for every \(\mathfrak{l} \in H_\mu\) we have  \(\mathfrak{h} \subset \mathfrak{l} \subset  \mathfrak{k}\) or \(\mathfrak{h} \subset \mathfrak{l}^\ast \subset \mathfrak{k}\)
\end{lemma}

\begin{proof}
  \todo{streamline proof using previous lemma}
  Suppose that \(H_\mu\) does not contain strongly separated halfspaces. By Proposition~\ref{prop:cs-5.1} we find two strongly separated halfspaces \(\mathfrak{k}_i\) such that \(\mathfrak{k}_1 \subset \mathfrak{l} \subset \mathfrak{k}_2\). By Lemma~\ref{lem:cs-dsl} we find a \(g \in \Gamma\) such that
  \[
    g\mathfrak{k}_1 \subset g\mathfrak{k}_2 \subset \mathfrak{k}_1 \subset \mathfrak{l} \subset \mathfrak{k}_2 \subset g^{-1}\mathfrak{k}_1 \subset g^{-1} \mathfrak{k}_2.
  \]
  We claim that \(g\mathfrak{k}_2\) and \(\mathfrak{k}_2\) are strongly separated. Suppose that \(\mathfrak{h}\) is transverse to \(\mathfrak{k}_2\). If it were also transverse to \(g \mathfrak{k}_2\), we would have
  \begin{align*}
    \mathfrak{k}_1 \cap \mathfrak{h} &\supset g\mathfrak{k}_2 \cap \mathfrak{h} \neq \varnothing,\\
    \mathfrak{k}_1 \cap \mathfrak{h}^\ast &\supset g\mathfrak{k}_2 \cap \mathfrak{h}^\ast \neq \varnothing,\\
    \mathfrak{k}_1^\ast \cap \mathfrak{h} &\supset \mathfrak{k}_2^\ast \cap \mathfrak{h} \neq \varnothing \quad \text{and}\\
    \mathfrak{k}_1^\ast \cap \mathfrak{h}^\ast &\supset \mathfrak{k}_2^\ast \cap \mathfrak{h}^\ast \neq \varnothing.
  \end{align*}
  Hence \(\mathfrak{h}\) would also be transverse to \(\mathfrak{k}_1\). A contradiction. Analogously, we can proceed to show that the following pairs of are strongly separated:  \((\mathfrak{k}_1,g^{-1}\mathfrak{k}_1)\), \((g\mathfrak{k}_1, g\mathfrak{k}_2)\), \((g\mathfrak{k}_1, \mathfrak{k}_1)\), \((g^{-1}\mathfrak{k}_1, g^{-1}\mathfrak{k}_2)\), \((\mathfrak{k}_2,g^{-1}\mathfrak{k}_2)\) and \((g\mathfrak{k}_1, g^{-1}\mathfrak{k}_2)\).

  We would like to show that \(g\mathfrak{k}_1 \in H_\mu^-\) and \(g^{-1}\mathfrak{k}_2 \in H_\mu^+\). If neither \(\mathfrak{k}_i\) is in \(H_\mu\), we are done (since \(\mathfrak{l}\) is in \(H_\mu\)). Suppose that \(\mathfrak{k}_2\) is in \(H_\mu\), then \(\mathfrak{k}_1 \in H_\mu^-\), because \(H_\mu\) contains no strongly separated halfspaces. Thus by the additivity of the measure we also have \(g\mathfrak{k}_1 \in H_\mu^-\). Additionally, \(g^{-1}\mathfrak{k}_2 \in H_\mu^+\) again since it is strongly separated from \(\mathfrak{k}_2\). The case if \(\mathfrak{k}_1 \in H_\mu\) can be proven similarly.

  There is one additional step necessary. We define \(\mathfrak{h} \coloneqq g^2\mathfrak{k}_1 \in H_\mu^-\) and \(\mathfrak{k} \coloneqq g^{-2}\mathfrak{k}_2 \in H_\mu^+\), which are strongly separated by the same argument as above. Furthermore, we set \(\mathfrak{k_0} \coloneqq g\mathfrak{k}_1 \in H_\mu^-\) and \(\mathfrak{k}_3 \coloneqq g^{-1}\mathfrak{k}_2\). Then we have the following sequence:
  \[
    \mathfrak{h} \subset \mathfrak{k}_0 \subset \mathfrak{k}_3 \subset \mathfrak{k},
  \]
  where the pairs \((\mathfrak{h}, \mathfrak{k}_0)\), \((\mathfrak{k_0}, \mathfrak{k}_3)\) and \((\mathfrak{h}, \mathfrak{k})\) are strongly separated. 

  If we take any other \(\mathfrak{l}' \in H_\mu\), then \(\mathfrak{k} \not\subset \mathfrak{l}'\) and \(\mathfrak{k} \not \subset \mathfrak{l}'^\ast\), because of the measure. Additionally, \(\mathfrak{l}' \not\pitchfork \mathfrak{k}\), since then it would be parallel to \(\mathfrak{k}_3\) and as before we would have \(\mathfrak{k}_3 \not\subset \mathfrak{l}'\) and \(\mathfrak{k}_3 \not\subset \mathfrak{l}'^\ast\). Hence, \(\mathfrak{l}'\) or \(\mathfrak{l}'^\ast\) would have to contain \(\mathfrak{k}_3\) and thus \(\mathfrak{k}\). All in all this shows that either \(\mathfrak{l}'\) or \(\mathfrak{l}'^\ast\) would have to contain \(\mathfrak{k}\). After a possible renaming, we can assume that \(\mathfrak{l}' \subset \mathfrak{k}\). A similar argument applied to \(\mathfrak{h}\) and \(\mathfrak{l}'\), will then show that \(\mathfrak{h} \subset \mathfrak{l}'\) and hence \(\mathfrak{\hat l}' \in \mathfrak{h}^\ast \cap \mathfrak{k}\).
\end{proof}


\begin{lemma}[{\cite[Lemma~4.20]{MR3509968}}]
  \label{lem:4.20}
  Let \(\mu_i \in \mathcal{P}(\bar X)\) be measures such that \(\hat H_{\mu_i}\) does not contain strongly separated hyperplanes for all \(i\) and \(H_{\mu_i} \cap H_{\mu_j} \neq \varnothing\) for all \(i,j\). Then there exists a pair of halfspaces \(\mathfrak{h} \subset \mathfrak{k}\) such that \(\mathfrak{\hat h}\) and \(\mathfrak{\hat k}\) are strongly separated and for every \(\mathfrak{l} \in H_{\mu_j}\), \(\mathfrak{h} \subset \mathfrak{l} \subset \mathfrak{k}\) or \(\mathfrak{h} \subset \mathfrak{l}^\ast \subset \mathfrak{k}\).
\end{lemma}

\begin{proof}
  We fix \(\mu_0\) and apply the previous lemma to find strongly separated halfspaces \(\mathfrak{h_2} \subset \mathfrak{h_3}\). By Lemma~\ref{lem:cs-dsl}, we find a \(g \in \Gamma\) such that \(g\mathfrak{h}_3 \subset \mathfrak{h}_2 \subset \mathfrak{h}_3\). We set \(\mathfrak{h}_0 \coloneqq g^2 \mathfrak{h}_2\), \(\mathfrak{h}_1 \coloneqq g\mathfrak{k}_2\), \(\mathfrak{h}_4 \coloneqq g^{-1}\mathfrak{h}_3\) and \(\mathfrak{h}_5 \coloneqq g^{-2} \mathfrak{h}_3\). Then we have the sequence
  \[
    \mathfrak{h}_0 \subset \mathfrak{h}_1 \subset \mathfrak{h}_2 \subset \mathfrak{h}_3 \subset \mathfrak{h}_4 \subset \mathfrak{h}_5
  \]
  and since \(\mathfrak{h}_2\) and \(\mathfrak{h}_3\) are strongly separated, by the previous lemma we have that the above halfspaces are pairwise strongly separated.

  We would like to show that \(\mathfrak{h}_0 \subset \mathfrak{l} \subset \mathfrak{h}_5\) for each \(\mathfrak{l} \in H_{\mu_i}\) and every \(i\). This is in fact already true for every \(l \in H_{\mu_i} \cap H_{\mu_0}\) and since this intersection is not empty, we fix for each \(i\) one \(\mathfrak{l}_i\).

  Now, for every \(\mathfrak{l}\) as above, we see that \(\mathfrak{l}\) can be parallel to at most one \(\mathfrak{h}_i\), since they are pairwise strongly separated. If it is transverse to any \(\mathfrak{h}_i\), where \(1 \leq i \leq 4\), we are done, because in this case \(\mathfrak{l}\) is parallel to \(\mathfrak{h}_0\) and \(\mathfrak{h}_5\). Assume that \(\mathfrak{l}\) were transverse to \(\mathfrak{h}_0\). Then it would have to be parallel to \(\mathfrak{h}_1\) and we could consider the following cases. We could have the chain \(\mathfrak{l}' \subset \mathfrak{h}_1 \subset \mathfrak{h}_2 \subset \mathfrak{l}_i\), where \(\mathfrak{l}'\) were either \(\mathfrak{l}\) or its complement. In either case, we would have \(\mathfrak{l}', \mathfrak{l}_i \in H_{mu_i}\) and thus these same were also true for the two enclosed halfspaces. However, \(\mathfrak{h}_1\) and \(\mathfrak{h}_2\) are strongly separated and \(H_{\mu_i}\) does not contain pairs of strongly separated halfspace, so this cannot happen. The only remaining case were, where \(\mathfrak{h_0} \subset \mathfrak{h}_1 \subset \mathfrak{l}'\), but then \(\mathfrak{h}_0\) could not be transverse to \(\mathfrak{l}\). Thus \(\mathfrak{l}\) can never be transverse to \(\mathfrak{h}_0\). Analogously, one can show that \(\mathfrak{l}\) is never transverse to \(\mathfrak{h}_5\).

  If \(\mathfrak{l}\) is parallel to each \(\mathfrak{h}_i\), then \(\mathfrak{l}' \not\subset \mathfrak{h_0}\), since otherwise \(H_{\mu_i}\) would again contain pairs of strongly separated halfspaces. By the same argument \(\mathfrak{h}_5 \not\subset \mathfrak{l}'\), where \(\mathfrak{l}'\) is defined as above. As before we yield \(\mathfrak{h}_0 \subset \mathfrak{l} \subset \mathfrak{h}_5\) or \(\mathfrak{h}_0 \subset \mathfrak{l}^\ast \subset \mathfrak{h}_5\).
\end{proof}

\begin{thm}[{\cite[Theorem~4.1]{Caprace2010}}]
  \label{thm:cs-flipping}
  Assume that \(X\) is a finite-dimensional CAT(0) cube complex and let \(\Gamma \leq \Aut(X)\) be any subgroup. Let \(\mathfrak{h} \in \mathcal{H}(X)\) such that \(\mathfrak{h}^\ast \not\subset g\mathfrak{h}\) for each \(g \in \Gamma\). Then \(\Gamma\) has a fixed point in the visual boundary or \(\mathfrak{h}\) is not essential with regard to \(\Gamma\).
\end{thm}

\begin{cor}[{\cite[Corollary~4.21]{MR3509968}}]
  \label{cor:4.21}
  Assume that for almost every \(\mu \in \mathcal{P}(\bar X)\), there are no strongly separated pairs in \(H_\mu\). If \(H_\mu \cap H_\nu \neq \varnothing\) for almost every pair \((\mu, \nu)\) then the \(\Gamma\)-action is non-essential.
\end{cor}

\begin{proof}
  \todo{prove this}
  
\end{proof}

\subsection{The main theorem}
\label{sec:main-proof}

\begin{thm}[{\cite[Theorem~4.1]{MR3509968}}]
  \label{thm:4.1}
  Let \(\Gamma \to \Aut(Y)\) be a group action on an irreducible finite dimensional CAT(0) cube complex \(Y\). Assume the action is essential and non-elementary. If \(B\) is a strong \(\Gamma\)-boundary, there exists a \(\Gamma\)-equivariant measurable map \(\phi\colon B \to \partial Y\) taking values in the non-terminating ultrafilters in \(\partial Y\).
\end{thm}

\begin{proof}
  By Lemma~\ref{lem:h-const} \(|H_\mu|\) is essentially constant.
  If \(H_\mu\) is finite we can invoke Lemma~\ref{lem:finite-zero} in order to see that \(H_\mu\) needs to be empty. Hence Lemma~\ref{lem:H=0} and Lemma~\ref{lem:4.11} lead to our desired map.

  The only thing left to prove is, that \(H_\mu\) cannot be infinite. Contrarily, assume that it were and consider the sets \(H_\mu \cap H_\nu\). By Lemma~\ref{lem:hh-const} their cardinality must be essentially constant and we consider the case that the sets are infinite for almost all \(\mu, \nu \in \mathcal{P}(\bar X)\).
  We consider the subset
  \[
    \mathcal{E} \coloneqq \{(\mu, \nu) \in \mathcal{P}(\bar X) \times \mathcal{P}(\bar X)\mid H_\mu = H_\nu\}.
  \]
  This set is \(\Gamma\)-invariant and measurable\todo{think about measurability} and by the doubly ergodic action of \(\Gamma\) on \(\mathcal{P}(\bar X)\), we see that it has either measure 0 or measure 1. Let us consider the two cases
  \begin{description}
  \item[\(\vartheta(\mathcal{E})=1\):] Since \(H_\mu = H_\nu\) for almost all \(\mu\) and \(\nu\) and by the definition of the product measure we can find a \(\mu_0 \in \mathcal{P}(\bar X)\) and a full measure subset, \(\Gamma\)-invariant subset \(M \subset \mathcal{P}(X)\)\todo{understand why you find this \(\Gamma\)-invariant full measure subset}, such that \(H_\nu = H_{\mu_0}\) for all \(\nu \in M\). This leads to \(H_{g\nu} = gH_\nu = H_\nu\) for all \(\nu \in M\) because of the \(\Gamma\)-invariance. The set \(H_\nu\) is a pocset and the decomposition \(\mathcal{H} = H_\mu \sqcup H_{g\mu}^+ \sqcup H_{g\mu}^-\) is a lifting decomposition for every \(g \in \Gamma\). Then by Lemma~\ref{lem:lifting} we find for each \(g\nu\) a subcomplex \(\bar X_{g\nu} \subset \bar X\) with \(H_{g\nu} = H_\nu\) as its corresponding pocset of halfspaces. We would now like to show that that not all of these subcomplexes are identical. In order to achieve this, we will need to show that the lifting decomposition is not \(\Gamma\)-invariant.
    The \(\Gamma\)-action on \(X\) is essential and without fixed point at infinity, hence we can apply Theorem~\ref{thm:cs-flipping} and see that if for any \(\mathfrak{h} \in H_{\nu}^+\) there exists a \(g \in \Gamma\) such that \(\mathfrak{h} \subset g^{-1}\mathfrak{h}^\ast \). Then we have that \(g^{-1}\mathfrak{h} \in H_\nu^+\) and by definition \(g^{-1} \mathfrak{h} \in H_{g^{-1}\nu}^-\). Hence \(H_\nu^+ \cap H_{g^{-1}\nu}^- \neq \varnothing\) and the same is true for \(H_{g\nu}^+ \cap H_{\nu}^-\). If we take any \(\mathfrak{h} \in \mathcal{H}\) such that \(g\mathfrak{h} \in H_{g\nu}^+ \cap H_\nu^-\), then \(\mathfrak{h} \in H_{\nu}^+\), but \(g\mathfrak{h} \not \in \H_\mu^+\), as it lies in \(H_\nu^-\). Hence, not all lifting decompositions are the same and some of the subcomplexes have to differ. This allows us to apply Proposition~\ref{prop:4.17}, which shows that in this case our cube complex would split as a product of two cube complexes, which is a contradiction to the fact that it is irreducible.
  \item[\(\vartheta(\mathcal{E})=0\):] In this case we have \(H_\mu \neq H_\nu\) for almost all \(\mu\) and \(\nu\). We define the set
    \[
      \mathcal{S} \coloneqq \{(\mathfrak{h}, \mathfrak{k}) \in \mathcal{H} \times \mathcal{H} \mid \mathfrak{h} \text{ and } \mathfrak{k} \text{ are strongly separated}\}.
    \]
    By Lemma~\ref{lem:strongly-sep} we see that this set is \(\Gamma\)-invariant and we can consider the following map
    \begin{align*}
      \mathcal{P}(\bar X) &\to \N \cup \{infty\},
      \mu &\mapsto |H_\mu \times H_\mu \cap \mathcal{S}|,
    \end{align*}
    which is by the above observation \(\Gamma\)-invariant and also measurable\todo{measurability}. Hence, it is essentially constant. If the essential value is greater than 0, we can use Lemma~\ref{lem:4.18} and see that \(H_\mu^+\) satisfies the descending chain condition. This implies that the set \(H_\mu^+ \cap H_\nu\) contain terminal elements, whenerver the intersection is not empty. Furthermore, \(\tau([H_\mu^+ \cap H_\nu] \cup [H_\nu^+ \cap H_\mu])\) is finite by Lemma~\ref{lem:finite-terminal}. However, then
    \begin{align*}
      \mathcal{P}(\bar X) \times \mathcal{P}(\bar X) &\to \operatorname{Pot}_f(\mathcal{H}),\\
      (\mu,\nu) &\mapsto \tau([H_\mu^+ \cap H_\nu] \cup [H_\nu^+ \cap H_\mu])
    \end{align*}
    is \(\Gamma\)-equivariant and measurable\todo{measurability} and Corollary~\ref{cor:4.5} assures that \(H_\mu^+ \cap H_\nu = \varnothing\) for almost all \(\mu\) and \(\nu\). However, \(H_\nu \neq H_\mu\) for almost all \(\mu\) and \(\nu\). Hence there exists a \(\mathfrak{h} \in H_\nu \setminus H_\mu\) and thus \(\mathfrak{h} \in H_\mu^\pm\). Without loss of generality, we can assume that it lies in \(H_\mu^+\) and we see that \(H_\mu^+ \cap H_\nu \neq \varnothing\) for almost all \(\mu\) and \(nu\), which is a contradiction.
  \end{description}
  Hence we see that \(H_\mu \cap H_\nu\) cannot be infinite, but then we have the map
  \begin{align*}
    \mathcal{P}(\bar X) \times \mathcal{P}(\bar X) & \to \operatorname{Pot}_f(\mathcal{H}),\\
    (\mu,\nu) &\mapsto H_\mu \cap H_\nu,
  \end{align*}
  which is \(\Gamma\)-invariant and measurable (c.\,f.~Section~\ref{sec:meas-maps}) and takes values in the finite subsets of \(\mathcal{H}\). With the help of Corollary~\ref{cor:4.5}, we see that this implies that \(H_\mu \cap H_\nu\) need to be empty for almost all \(\mu\) and \(\nu\).

  In this case we consider the following map
  \begin{align*}
    \mathcal{P}(\bar X) \times \mathcal{P}(\bar X) &\to \N \cup \{infty\},\\
    (\mu, \nu) \mapsto |\tau([H_\mu \cap H_\nu^+] \cup [H_\nu \cap H_\mu^+])|,
  \end{align*}
  where \(\tau\) is the mapp assigning the set of terminal elements of any subset of \(\mathcal{H}\) (c.\,f.~Definition~\ref{defin:tau}). This map is again measurable and \(\Gamma\)-invariant\todo{show measurability} and hence essentially constant. Furthermore, by Lemme~\ref{lem:finite-terminal} we see that the map takes only finite values. However, invoking again Corollary~\ref{cor:4.5}, we have that \([H_\mu \cap H_\nu^+] \cup [H_\nu \cap H_\mu^+]\) contains no terminal elements for almost all \(\mu\) and \(\nu\). This allows us to apply Proposition~\ref{prop:4.10}, which leads to the impossibility of the case \(H_\mu \cap H_\nu = \varnothing\). 
  
  This implies that the case \(|H_\mu| = \infty\) for almost all \(\mu \in \mathcal{P}(\bar X)\) is impossible. Then Lemma~\ref{lem:finite-zero} implies that \(H_\mu\) is empty for almost all \(\mu \in \mathcal{P}(\bar X)\). This allows us to apply Lemma~\ref{lem:H=0} and Lemma~\ref{lem:4.11}, which yield our map.
  % By Corollary~\ref{cor:p(x)}, we yield a \(\Gamma\)-equivariant measurable map \(\psi\colon B \to \mathcal{P}(\bar X)\). Via this map we can pushforward the probability measure \(\vartheta\) on \(B\) to \(\mathcal{P}(\bar X)\). With the help of Proposition~\ref{prop:coeff-ergodic} and Lemma~\ref{lemma:ergodicity-pushforward} \(\mathcal{P}(\bar X)\) becomes (doubly) ergodic with regard to the \(\Gamma\) action.

  % Consider the map \(\mathcal{P}(\bar X) \to \N \cup \{\infty\},\ \mu \mapsto |H_\mu|\). This map is measurable because of Lemma~\ref{lem:measurable-mu} and~\ref{lem:measurable-countable}. Additionally consider \(g \in \Gamma\), then we have
  % \begin{align*}
  %   |H_{g\mu}|
  %   & = |\{\mathfrak{h} \in \mathcal{H} \mid (g\mu)(\mathfrak{h}) = \text{\nfrac{} 1/2}\}|\\
  %   & = |\{\mathfrak{h} \in \mathcal{H} \mid \mu(g^{-1} \mathfrak{h}) = \text{\nfrac{} 1/2}\}|\\
  %   & = |\{\mathfrak{gh} \in \mathcal{H} \mid \mu(\mathfrak{h}) = \text{\nfrac{} 1/2}\}|\\
  %   & = |g H_\mu|
  % \end{align*}
  % and the map is \(\Gamma\)-invariant. By the ergodicity of \(\mathcal{P}(\bar X)\) we have that it is essentially constant. Let \(N \in \N \cup \{\infty\}\) be this essential value. If \(N = 0\) we will find our desired map as follows:

  % Up to measure 0 the image of \(\psi\) lies in \(\mathcal{E} \coloneqq \{ \mu \in \mathcal{P}(\bar X) \mid H_\mu = \varnothing\}\). This set is measurable by Lemma~\ref{lem:measurable-mu} applied to the case \(I = \{\text{\nfrac{} 1/2}\}\) and using the preimage of \(\varnothing\).

  % If \(H_\mu = \varnothing\) then \(H_\mu^+\) is an ultrafilter and we have a well-defined mapped
  % \[
  %   \xi\colon \mathcal{E} \to \bar X,\ \mu \mapsto H_\mu^+.
  % \]
  % This map is measurable (again thanks to Lemma~\ref{lem:measurable-mu} this time applied to \(I = (\text{\nfrac{} 1/2}, 1]\)) and as above \(\Gamma\)-equivariant (up to measure 0). All in all we yield the desired map \(\phi = \xi \circ \psi\) which is both measurable and \(\Gamma\)-equivariant as a concatenation of measurable and \(\Gamma\)-equivariant functions.

  % With the help of Lemma~\ref{lem:4.11} we see that the image of \(\phi\) will lie in the non-terminating ultrafilters of \(X\).

  % All that is left to prove now, is that the essential value \(N\) can take only the value \(0\). For this we will consider multiple cases
  % \begin{description}
  % \item[\(0 < N < \infty\):] In this case we would have a measurable and \(\Gamma\)-equivariant map \(\mathcal{P}(\bar X) \to Pot_f(\mathcal{H})\) which by Corollary~\ref{cor:4.5} would imply that the \(Gamma\)-action on \(Y\) is not essential.
  % \item[\(N = \infty\):] This case is a bit more involved and we have to consider another measurable and \(\Gamma\)-equivariant map
  %   \[
  %     \mathcal{P}(\bar X) \times \mathcal{P}(\bar X) \to \N \cup \{\infty\},\ (\mu, \nu) \mapsto |H_\mu \cap H_\nu|.
  %   \]
  %   Again by ergodicity on \(\mathcal{P}(\bar X) \times \mathcal{P}(\bar X)\) it is essentially constant and we have another essential value \(M \in \N \cup \{\infty\}\).
  %   \begin{description}
  %   \item[\(M = 0\):] For this case we have to consider yet another map, namely
  %     \[
  %       T\colon \mathcal{P}(\bar X) \times \mathcal{P}(\bar X) \to \N \cup \{\infty\},\ T(\mu, \nu) \coloneqq |\tau((H_\mu \cap H_\nu^+) \cup (H_\nu \cap H_\mu^+)).
  %     \]
  %     By Lemma~\ref{lem:tau} this is a measurable map as a composition of measurable maps. Furthermore, it is \(\Gamma\)-equivariant and thus essentially constant by ergodicity. By Lemma~\ref{lem:finite-terminal}, we know that any subset of \(H_\mu\) and \(H_\nu\) contains only finitely many terminal elements. Hence the essential value must be finite. Hence \(T\) (ignoring taking the cardinality) we have an \(\Gamma\)-equivariant map to \(\operatorname{Pot}_f(\mathcal{H})\) using Corollary~\ref{cor:4.5}, we see that the essential values must be 0. In this case \(H_\mu \cap H_\nu^+\) contains no terminal elements for almost all \((\mu, \nu)\). However, Proposition~\ref{prop:4.10} shows that this is a contradiction to the finite dimensionality of \(X\).
  %   \item[\(0 < M < \infty\):] This case is completely analogous to the case \(0 < N < \infty\) and we use again Corollary~\ref{cor:4.5} to infer that the \(\Gamma\)-action were not essential in this case. 
  %   \item[\(M = \infty\):] 
  %   \end{description}
  % \end{description}
\end{proof}

\begin{cor}[\cite{MR3509968}]
  Let \(\Gamma \to \Aut(X)\) be a group acting on a finite dimensional CAT(0) cube complex \(X\). Assume that there is no finite orbit in the visual boundary \(\partial_\sphericalangle X\) and denote by \(Y\) the essential core of \(X\). Then there exists a \(\Gamma\)-equivariant measurable map \(\phi \colon B \to \partial Y \subseteq \partial X\).
\end{cor}

\begin{proof}
  Since there is no finite orbit of \(\Gamma\) in the visual boundary, we also have no fixed point there. Hence, we can apply Proposition~\ref{prop:cs-3.5}, which yields that the essential core \(Y\) is not empty. As a convex subcomplex, we have \(\partial_\sphericalangle Y \subset \partial_\sphericalangle X\) and we also have no finite orbits of \(\Gamma\) in \(\partial_\sphericalangle Y\). We decompose \(Y\) as \(Y = Y_1 \times \dots \times Y_m\), where each \(Y_i\) is irreducible (which is possible by Proposition~\ref{prop:cs-2.6}). By Lemma~\ref{lem:2.28} we know that \(\Gamma_0 \leq \Gamma\) has still no finite orbit in \(\partial_\sphericalangle Y_i\) for each \(i\) and the action on each \(Y_i\) is essential.\todo{Problem: How can \(\Gamma\) operate on these factors?}

  Assume that \(\Gamma\) acts on each \(Y_i\), then, by Theorem~\ref{thm:4.1}, we find \(\phi_i \colon B \to \partial Y_i\) which are measurable and \(\Gamma\)-equivariant. Each halfspaces in \(Y_i\) corresponds to a unique halfspace in \(Y\) and thus for family of ultrafilters \(\alpha_i \in \bar Y_i\) can be mapped to an ultrafilter \(\alpha \in \bar Y\). This can be most easily seen in the product space definition of \(\bar Y\), since then we just join the tuples of \(\alpha_i\) (after choosing an appropriate enumeration of \(\bar Y\)). The choice condition is then automatically satisfied and for the consistency condition, we only have to note that if two halfspaces come from two different \(\alpha_i\), then they must be transverse. Using basic open sets it is easy to see that this map is continuous and hence measurable. Lastly, \(\bar Y \subset \bar X\), since \(Y\) is convex subcomplex. All in all, we see that the \(\phi_i\) lead to a measurable map \(\phi\colon B \to \bar X\). Since each the image of each \(\phi_i\) lies in the non-terminating ultrafilters of \(Y_i\), we see that each ultrafilter in the image of \(\phi\) has at least one non-terminating descending chain and thus we have that indeed \(\phi\) takes values in \(\partial X\).
\end{proof}

%%% Local Variables:
%%% mode: latex
%%% TeX-master: "../Master"
%%% End:
