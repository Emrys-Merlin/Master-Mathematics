\section{Group actions on CAT(0) cube complexes and strong group boundaries}
\label{sec:group}

\subsection{Extending the group action to the Roller boundary}
\label{sec:ga-roller}

In this section I would like to remark on how one can extend a group action on a CAT(0) cube complex \(X\) to its Roller boundary \(\bar X\). For that matter, let in the following \(\Gamma\) be a discrete countable group with an action \(\Gamma \to \Aut(X)\), where \(X\) is a finite dimensional CAT(0) cube complex. \(Aut(X)\) consists of the combinatorial isomorphisms (c.\,.f.\ Def.~\ref{def:morphism-ccc}).

The following proposition collects some facts about how cubical isomorphisms act on \(X\).

\begin{prop}
  Let \(g \in \Aut(X)\). Then there holds
  \begin{enumerate}
  \item \(\mathfrak{\hat h} \in \mathcal{\hat H}(X) \Rightarrow  g\mathfrak{\hat h} \in \mathcal{\hat H}(X)\)
  \item \(\mathfrak{h} \in \mathcal{H}(X) \Rightarrow  g\mathfrak{h} \in \mathcal{H}(X)\)
  \item \(\forall \mathfrak{h} \in \mathcal{H}(X)\colon g(\mathfrak{h}^\ast) = (g\mathfrak{h})^\ast\)
  \item \(\mathfrak{h,h'} \in \mathcal{H}(X)\colon \mathfrak{h} \subset \mathfrak{h'} \Rightarrow g\mathfrak{h} \subset g\mathfrak{h'}\)
  \item \(\alpha \in \bar X \Rightarrow g\alpha \in \bar X\)
  \item If \(\alpha\) satisfies the descending chaing condition then so does \(g\alpha\).
  \end{enumerate}
\end{prop}

\begin{rem}
  Since I defined the ultrafilters on \(X\) as a special subset of a \emph{product space}, it is not immediatly clear how \(g\) operates on this space. However, what is meant is that \(g\) operates on each halfspace contained in \(\alpha\) separately and these new halfspaces determine a unique new ultrafilter which will be denoted by \(g\alpha\).
\end{rem}

\begin{proof}
  1.\ is an immedate consequence of the fact, that \(g\) is an isometry. This leads directly to 2 and 3. For 4.\ we only need that \(g\) is a bijection and 5 and 6 are then simple applications of 4.
\end{proof}

With the above proposition in place we already see that each group action \(\Gamma \to \Aut(X)\) immediately leads to an action \(\Gamma \to \operatorname{Perm}(\bar X)\). However, this is not yet what we want. It would be preferable, if the image lay in the homeomorphism group of \(X\). This will be acomplished with the next lemma.

\begin{lemma}
  Let \(g \in \Aut(X)\) and \(\mathcal{U} \coloneqq \mathcal{U}(\mathfrak{h}_1, \dots, \mathfrak{h}_n) \subset \bar X\) a basic open set. Then we have
  \[
    g^{-1} \mathcal{U} = \mathcal{U}(g^{-1}\mathfrak{h}_1, \dots, g^{-1}\mathfrak{h}_n).
  \]
  Hence, \(g \in \operatorname{Homeo}(\bar X)\).
\end{lemma}

Together we arrive at the following result.

\begin{thm}
  \label{thm:roller-action}
  Let \(\Gamma\) be a countable discrete group and \(\Gamma \to \Aut(X)\) a group action on a CAT(0) cube complex \(X\). Then this action extends to an action \(\Gamma \to \operatorname{Homeo}(\bar X)\) on the Roller compactification.
\end{thm}

\begin{defin}[(Non-)elementary action]
  A group action \(\Gamma \to \Aut(X)\) is called \emph{elementary}, if there exists a finite orbit of the action on \(X \sqcup \partial_{\sphericalangle}X\). Otherwise the action is called \emph{non-elementary}.
\end{defin}
\todo{important for essential core}

\begin{defin}[Essential action]
  \todo{Depends on the essential core, reference}
  A group action \(\Gamma \to \Aut(X)\) is called \emph{essential}, if the \(\Gamma\)-essential core is the whole space \(X\).
\end{defin}


\begin{defin}[strong \(\Gamma\)-boundary]
  \todo{Look up strong \(Gamma\)-boundary}
\end{defin}


\subsection{Strong \(\Gamma\)-boundaries}
\label{sec:grp-boundary}

\begin{defin}[Amenable group action]
  Let \(E\) be a separable Banach space. Then I denote by \(E^\ast_1\) the unit ball in its dual space. Let \(S\) be a standard Borel space and for each \(s \in S\) consider \(A_s \subset E^\ast_1\) a non-empty convex weak\(\ast\)-compact subspace. Then \((A_s)_{s \in S}\) will be called a \emph{Borel field of compact convex sets} if \(\{(s, \lambda) \mid \lambda \in A_s\}\) is a Borel subset of \(S \times E^\ast_1\).

  Let \(\Gamma\) be a locally compact group with a measure class preserving group action on \(S\). Furthermore, let \(M\) be a Borel group. Then \(\alpha \colon \Gamma \times s \to M\) is called a \emph{(left) cocycle} if \(\alpha(gh, s) = \alpha(g, hs) \alpha(h, s)\) for all \(g, h \in \Gamma\) and almost all \(s \in S\).

  Each element \(T\) of \(\Isom(E)\) gives rise to a homeomorphism \(T^\ast\) of \(E^\ast_1\) via \((T^\ast\Phi)(x) \coloneqq \Phi(Tx)\) for every \(x \in E\). Thus every cocycle \(\alpha \colon S \times \Gamma \to \Isom(E)\) gives rise to a cocycle \(\alpha^\ast \colon S \times \Gamma \to \operatorname{Homeo}(E^\ast)\) via \(\alpha^\ast (g, s) = (\alpha(g, s)^{-1})^\ast\). A Borel field \((A_s)_{s \in S}\) is called \emph{\(\alpha\)-invariant} if \(\alpha^\ast(g, s) A_{s} = A_{gs}\) for each \(g \in \Gamma\) and almost all \(s \in S\).
  
  Let \(\Gamma\) be a locally compact group and \(S\) a standard Borel space with measure class preserving Borelian \(\Gamma\)-action. The \(\Gamma\)-action is said to be \emph{amenable} if for every separable Banach space \(E\), every Borelian (left) cocycle \(\alpha \colon S \times G \to \Isom(E)\) and every \(\alpha\)-invariant Borel field \((A_s)_{s \in S}\), there exists a Borel map \(\phi \colon S \to E^\ast_1\) such that \(\phi(s) \in A_s\) for almost all \(s\) and for each \(g \in \Gamma\)\(\alpha^\ast(g, s) \phi(s) = \phi(gs)\) almost everywhere.
\end{defin}

\begin{defin}[Strong \(\Gamma\)-boundary]
  Let \(\Gamma\) be a discrete, countable group. A standard Borel space \(B\) is called a \emph{strong \(\Gamma\)-boundary}, if
  \begin{enumerate}
  \item there is a measurable group action of \(\Gamma\) on \(B\) and this action is amenable and
  \item 
  \end{enumerate}
\end{defin}

\begin{lemma}[{\cite[Thm.\ V.6.6]{Conway}}]
  If \(X\) is a compact metric space. Then \(C(X)\) is separable.
\end{lemma}

\begin{thm}
  Let \(B\) be a strong \(\Gamma\)-boundary and \(X\) a compact metric space with a continuous \(\Gamma\) action. Then there exists a \(\Gamma\)-equivariant measurable map \(\phi \colon B \to \mathcal{P}(X)\), where \(\mathcal{P}(X)\) is the set of all probability measures on \(X\).
\end{thm}

\begin{proof}
  Consider \(C(X)\) the space of continuous functions from \(X\) to \(\R\). Together with the supremum norm this is a Banach space, which (by the previous lemma) is also separable. Furthermore, there exists a group action of \(\Gamma\) onf \(C(X)\) via \((gf)(x) = f(g^{-1}x)\) for each \(g \in \Gamma\), \(f \in C(X)\) and \(x \in X\). This action is clearly via isometries. Also for \(\mu \in \mathcal{P}(X)\) we define the action to be \((g\mu)(A) = \mu(g^{-1} A)\) for everye \(g \in \Gamma\) and \(A \in \Sigma\). Then the dual pairing established in the Riesz-Markow representation theorem yields \(\langle gf, \mu\ \rangle = \langle f, g^{-1} \mu \rangle\) or in other words \(g^\ast = g^{-1}\).
  Next, consider
  \[
    \alpha: \Gamma \times B \to \Isom(C(X)), (g, b) \mapsto g.
  \]
  This is a left cocycle.
  Since \(X\) is compact, we have \(C(X) = C_0(X)\) and thus, using the Riesz-Markow representation theorem, we yield \(C(X)^\ast \cong M_{s}(X)\). By Corollary~\ref{cor:banach-alaoglu} we know that \(\mathcal{P}(X)\) is weak\(\ast\)-compact and contained in the unit ball of \(M_s(X)\). Furthermore, \(\mathcal{P}(X)\) is convex and non-empty (take any normalized dirac measure), thus we can choose \(A_b = \mathcal{P}(X)\) for all \(b \in B\). This is in fact an \(\alpha\)-invariant Borel field. Since \(B\) is a strong \(\Gamma\)-boundary, the \(\Gamma\) action is amenable and we yield a measurable map \(\phi \colon B \to C(X)^\ast_1\) such that \(\phi(b) \in A_b = \mathcal{P}(X)\), i.\,e.\ \(\phi \colon B \to \mathcal{P}(X)\) (which is still measuralbe). Lastly, we have
  \begin{align*}
    \phi(gb) & = \alpha^\ast(g, b) \phi(b)\\
               & = \left(\alpha(g,b)^{-1}\right)^\ast \phi(b)\\
               & = \left ( g^{-1}\right)^\ast \phi(b)\\
               & = g\phi(b).
  \end{align*}
\end{proof}

\begin{cor}
  \label{cor:p(x)}
  Let \(X\) be a finite dimensional CAT(0) cube complex and \(\bar X\) its Rolle compactification. Let \(\Gamma \to \Aut(X)\) be a discrete countable group acting on \(X\). Furthermore, let \(B\) be a strong \(\Gamma\)-boundary. Then there exists \(\Gamma\)-equivariant measurable map \(\phi\colon B \to \mathcal{P}(\bar X)\), where \(\mathcal{P}(\bar X)\) is the set of probability measures on \(\bar X\).
\end{cor}

\begin{proof}
  By Corollary~\ref{cor:comp-met} it was established that \(\bar X\) is a compact metrizable space. Furthermore, the \(\Gamma\)-action on \(X\) extends to a \(\Gamma\)-action on \(\bar X\)(c.\,f.\ Theorem~\ref{thm:roller-action}). Thus all conditions for the previous theorem are satisfied and we get the desired map \(\phi\colon B \to \mathcal{P}(\bar X)\).
\end{proof}

%%% Local Variables:
%%% mode: latex
%%% TeX-master: "../Master"
%%% End:
