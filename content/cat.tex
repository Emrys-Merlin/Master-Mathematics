\subsection{CAT(0) spaces}
\label{sec:cat(0)}

% \begin{prop}[Projections~{\cite[Prop II.2.7]{MR1744486}}]
%   Let \(X\) be a complete CAT(0) space and \(A \subset X\) be convex and closed. Then
%   \begin{enumerate}
%   \item for any point \(x \in X\) there exists a unique point \(\pi_A(x)\) realizing the diestance from \(x\) to \(A\), i.\,e.
%     \begin{align*}
%       d(x, \pi_A(x)) = d(x, A) = \inf_{a \in A} d(x, a).
%     \end{align*}
%     In this way \(\pi_A\) defines a map \(X \to A\).
%   \item \(\pi_A\) is distance non-increasing.
%   \item if \(x' \in X\) lies on the geodesic segment \(\left[x, \pi_A(x)\right]\), we have \(\pi_A(x') = \pi_A(x)\).
%   \end{enumerate}
% \end{prop}

% \begin{prop}[Fix points]
%   Let \(\phi \colon X \to X\) be an isometry and \(X\) a CAT(0) space. Then
%   \begin{align*}
%     X^\phi \coloneqq \{x \in X \mid \phi(x) = x\} \subset X
%   \end{align*}
%   is closed and convex.
% \end{prop}

% \begin{proof}
%   The fix point set of any homeomorphism of a Hausdorff space to itself is closed. Furthermore, let \(\gamma\) be a geodesic joining two fix points \(x\) and \(y\). Then \(\phi \circ \gamma\) is again a geodesic joining the same two points. As \(X\) is uniquely geodesic, we yield \(\gamma = \phi \circ \gamma\) and hence the whole geodesic segment is fixed by \(\phi\), which is what we needed to show for convexity.
% \end{proof}


%%% Local Variables:
%%% mode: latex
%%% TeX-master: "../Master"
%%% End:
