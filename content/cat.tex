\subsection{CAT(0) spaces}
\label{sec:cat(0)}

This section will contain some important facts about general CAT(0) spaces. Where no proof is given, it can be found in \cite{MR1744486}. Recall that the general definition of a CAT(\(\kappa\)) space was given in Definition~\ref{def:cat}. \todo{decide which proofs to give}

\begin{prop}[{\cite[Prop II.1.4]{MR1744486}}]
  Let \(X\) be a CAT(0) space. Then
  \begin{enumerate}
  \item \(X\) is uniquely geodesic and
  \item \(X\) is contractible.
  \end{enumerate}
  \todo{proof by picture}
\end{prop}

\begin{defin}
  Let \(X,Y\) be metric spaces. \(\phi \colon X \to Y\) is called
  \begin{itemize}
  \item an \emph{isometric embedding}, if \(d_X(x,y) = d_Y(\phi(x), \phi(y))\) for any two \(x,y \in X\),
  \item an \emph{isometry}, if it is an isometric embedding and surjective (and hence bijective) and
  \item a \emph{local isometry}, if for each \(x \in X\) ther exists an open neighborhood \(U \subset X\) containing \(x\), such that \(\phi|_U \colon U \to \phi(U)\) is an isometry.
  \end{itemize}
\end{defin}

\begin{prop}[{\cite[Propositions 1 \& 2]{Rolli2012}}]
  Let \(X,Y\) be geodesic spaces and let \(Y\) be CAT(0). Then every local isometry \(\phi \colon X \to Y\) is an isometric embedding. In particular, every local geodesic is a geodesic.
\end{prop}

\todo{enter definition of convexity (do it above before convex polyhedral cells)}

% \begin{prop}[Projections~{\cite[Prop II.2.7]{MR1744486}}]
%   Let \(X\) be a complete CAT(0) space and \(A \subset X\) be convex and closed. Then
%   \begin{enumerate}
%   \item for any point \(x \in X\) there exists a unique point \(\pi_A(x)\) realizing the diestance from \(x\) to \(A\), i.\,e.
%     \begin{align*}
%       d(x, \pi_A(x)) = d(x, A) = \inf_{a \in A} d(x, a).
%     \end{align*}
%     In this way \(\pi_A\) defines a map \(X \to A\).
%   \item \(\pi_A\) is distance non-increasing.
%   \item if \(x' \in X\) lies on the geodesic segment \(\left[x, \pi_A(x)\right]\), we have \(\pi_A(x') = \pi_A(x)\).
%   \end{enumerate}
% \end{prop}

% \begin{prop}[Fix points]
%   Let \(\phi \colon X \to X\) be an isometry and \(X\) a CAT(0) space. Then
%   \begin{align*}
%     X^\phi \coloneqq \{x \in X \mid \phi(x) = x\} \subset X
%   \end{align*}
%   is closed and convex.
% \end{prop}

% \begin{proof}
%   The fix point set of any homeomorphism of a Hausdorff space to itself is closed. Furthermore, let \(\gamma\) be a geodesic joining two fix points \(x\) and \(y\). Then \(\phi \circ \gamma\) is again a geodesic joining the same two points. As \(X\) is uniquely geodesic, we yield \(\gamma = \phi \circ \gamma\) and hence the whole geodesic segment is fixed by \(\phi\), which is what we needed to show for convexity.
% \end{proof}

In every CAT(0) space each geodesic can be extended indefinitely\todo{cite something}. This allows the definition of the following boundary.

\begin{defin}[Visual boundary]
  Let \(\gamma_i \colon [0, \infty) \to X\) be two geodesic rays. \(\gamma_1 \sim \gamma_2\) if and only if there exists a constant \(K > 0 \) such that \(d(\gamma_1(t), \gamma_2(t)) < K\) for all \(t \geq 0\). The set of equivalence classes of this relation \(\partial_\sphericalangle X\) is called the \emph{visual boundary of \(X\)}.

  Clearly, each group action on \(X\) by isometries extends to an action on \(\partial_\sphericalangle X\).
\end{defin}

\begin{rem}
  \(X \sqcup \partial_{\sphericalangle}X\) can be topologized in a way that it becomes a compact space. The topology can be chosen to coincide with the one induced by the metric on \(X\). \todo{cite something for this. Consider how important it is}
\end{rem}

%%% Local Variables:
%%% mode: latex
%%% TeX-master: "../Master"
%%% End:
