\section{Introduction}
\label{sec:introduction}

\todo{Write introduction}

In 2013 \textcite{MR3095714} showed that if a countable group \(\Gamma\) acts cocompactly and properly discontinuously on a CAT(0) cube complex then a subspace of the Roller compactification is a geometric model of the Poisson boundary of this group. This opened the door to some geometric proofs regarding properties of the group and the complex. In particular, \textcite{MR3095714} were able to prove Property A for certain CAT(0) cube complexes by geometrical means.

A Poisson boundary always needs a random walk on the group to be defined. It would be desirable to find a notion independent of such a walk. This has been done by introducing \emph{strong group boundaries} as seen in \textcite{MR2006560}. In this paper he also showed that under certain conditions (in particular, the transition probability needs to be spread out, c.\,f.\ Example~\ref{bsp:poisson}) the Poisson boundary of a group in fact is its strong group boundary.

With this notion in place the remaining question, was whether it is possible to relate strong group boundaries to the Roller compactification of a CAT(0) cube complex \(X\), whenever the group \(\Gamma\) was acting on \(X\) by automorphisms. A partial answer to this question was given by \textcite{MR3509968}, which showed that if \(\Gamma\) acts essentially and non-elementary and if \(X\) is locally countable, finite-dimensional and irreducible\todo{check irreducible}, then there exists a \(\Gamma\)-equivariant, measurable map
\[
  \phi\colon B \to \partial X,
\]
where \(B\) is the strong \(\Gamma\)-boundary and \(\partial X\) is the Roller boundary of \(X\). They even proved a stronger result, showing that the image of \(\phi\) lies in the non-terminating ultrafilters of \(X\). The name given to this map was \emph{boundary map}. A first simple observation, yields that this is an elegant proof showing that under the above assumptions \(X\) dos have non-terminating ultrafilters. However, the applications only start there. In their paper, they used the map to show the non-vanishing cohomology groups of \(\Gamma\), leading to further superrigidity results.

In this thesis we will be concerned with the construction of the above mentioned map \(\phi\). The statement of the main theorem can be found in Theorem~\ref{thm:4.1}. The previous section filling in all the necessary technical details for the proof. In Chapter~\ref{sec:complexes}, we will introduce CAT(0) cube complexes and build up all the necessary background about their geometry and their combinatorial properties. In Chapter~\ref{sec:roller}, we will then introduce pocsets and ultrafilters leading to the Roller compactification \(\bar X\) and the Roller boundary \(\partial X\) of our complex. Clearly,these last two objects are some of the most central in this thesis as \(\phi\) will take values in \(\partial X\). After these two chapters, we will leave the realm of CAT(0) cube complexes and (in Chapter~\ref{sec:measure}) turn towards measure theory and functional analysis. We will introduce all the properties of measures, we will need later on, introduce certain measurable sets of halfspaces and prove the measurability of certain maps. Up to this point, we did not talk about our group \(\Gamma\) at all and in Chapter~\ref{sec:group} this will be remedied. First, we will talk about groups acting on CAT(0) cube complexes. We will see, how the group action can be extended to the Roller compactification \(\bar X\) and we will introduce the notions of \emph{essential} and \emph{non-elementary} group actions citing some important results by \textcite{Caprace2010}. The second half of the chapter is concerned with the introduction of \emph{strong group boundaries}. One essential ingredient is \emph{ergodicity}, which will be introduced in its own section (Section~\ref{sec:ergodic}). Lastly, we will generalize the notion of ergodicity to \emph{ergodicity with coefficients} and introduce the the notion of a \emph{amenable group action}. With these two concepts in place, we can finally define the strong boundaries. With that we finish the preliminaries and turn towards the proof of our main theorem. Chapter~\ref{sec:map} is divided in four sections, where the first three contain all the necessary technical details and the last contains the actual proof. 

With regards to content, this thesis follows closely the exposition of \textcite{MR3509968}.
%%% Local Variables:
%%% mode: latex
%%% TeX-master: "../Master"
%%% ispell-local-dictionary: "en_US"
%%% End:
