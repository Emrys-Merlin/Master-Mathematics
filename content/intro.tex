\section{Introduction}
\label{sec:introduction}

Geometric group theory is a relatively new field in mathematics. One starting point might be seen in Felix Klein's \emph{Erlangen program}~\cite{Klein}. There he pointed out the deep connection between (abstract) groups and their realization as automorphism groups of topological and geometric spaces. In the following years this stand point was refined and graphs and trees (the Cayley graph) were identified as important geometrical objects in order to understand groups from a geometrical point of view. Until the 1980s there was slow but steady progress in the field. This changed dramatically with Gromov's work\ \cite{MR919829} and the progress in Thurstons's geometrization program\ \cite{MR648524}. In his treatise Gromov introduced (word-)hyperbolic groups and in this brought group theory closer to the geometry of hyperbolic spacer or more generally to the geometry of non-positively curved space, captured in the notion of CAT(0) spaces.

In the 1990, \textcite{MR1347406} brought certain special CAT(0) spaces to the attention of a wider audience, the so called \emph{CAT(0) cube complexes}. It turned out that general CAT(0) spaces are often too complicated to facilitate the understanding of a group. Cube complexes introduce a combinatorial structure which makes these spaces more \enquote{rigid} and hence, easier to handle. At the same time, they are still flexible enough to admit enough interesting group actions. In particular, every fundamental group of a surface of genus at least one admits an action on a CAT(0) cube complex (see Example~\ref{bsp:ccc}). Another reason why CAT(0) cube complexes were quickly adopted as a natural object of study is that every tree is a CAT(0) cube complex. Hence, this new object generalized the old \enquote{workhorse} of the field. Interestingly, Sageev was not the first to introduce the complexes. This had already been done by Gromov in his 1987 treatise\ \cite{MR919829}, but only as a particular example of a CAT(0) space.

Another old and very important object class in geometric group theory are the Lie groups. Again in the 1990s \textcite{MR1090825} was able to prove an astounding result which earned the name \emph{Margulis superrigidity}. It states that under certain (rather weak) conditions a linear representation of a lattice in a Lie group can be extended to the whole group. This superrigidity result had a deep impact and it became an objective to find analogous results in slightly different settings. In particular, the case of groups acting on CAT(0) cube complexes came to mind.

In 2016, \textcite{MR3509968} succeeded in this endeavor proving a superrigidity theorem for group acting \emph{essential} and \emph{non-elementary} on CAT(0) cube complexes (see Section~\ref{sec:special}). The main ingredient in their result was the construction of a so called \emph{boundary map}. This maps connects the group \(\Gamma\) via a \emph{strong \(\Gamma\)-boundary} \(B\) to the \emph{Roller boundary} \(\partial X\) (a subset of the \emph{Roller compactification} \(\bar X\)) of a CAT(0) cube complex \(X\). More precisely, they proved the existence of a measurable map
\[
  \phi\colon B \to \partial X
\]
which is \(\Gamma\)-equivariant almost everywhere. The space \(B\) is in fact a probability space. The set \(\bar X\) is deeply intertwined with the combinatorial structure of \(X\). Each CAT(0) cube complex has an associated set of hyperplanes \(\mathcal{\hat H}\) dividing the complex in two convex components. The set of all these components \(\mathcal{H}\) is called the set of all \emph{halfspaces} of \(X\). Certain subsets of these sets are denoted \emph{ultrafilters} (see Section~\ref{sec:pocset}). The space \(\bar X\) is then simply given as the set of all ultrafilters (the in detail construction can be found in Chapter~\ref{sec:roller}).

The aim of this thesis is to understand and present the construction of this map (Theorem~\ref{thm:4.1}). The idea of the proof goes along the following lines: Strong \(\Gamma\)-boundaries \(B\) are defined via two key properties. First, the \(\Gamma\)-action on \(B\) must be amenable. Secondly, the \(\Gamma\)-action must be doubly ergodic with coefficients, which is a strengthening of the standard notion of (double) ergodicity. \(\Gamma\) also acts on \(X\) and it can be seen that this action can be extended to the Roller compactification \(\bar X\), where \(\Gamma\) acts via homeomorphisms. This is already sufficient for the amenability to guarantee a measurable map
\begin{align}
  \psi\colon B \to \mathcal{P}(\bar X)\label{eq:psi}
\end{align}
which is \(\Gamma\)-equivariant almost everywhere and where \(\mathcal{P}(\bar X)\) denotes the set of all regular probability measures on \(\bar X\). The hard part of the proof is then that every probability measure in the image of \(\psi\) identifies a point in \(\bar X\), i.\,e.\ in some sense we would like the probability measures in the image to have a point mass. In order to make this desire precise, we need to introduce \emph{weighted halfspaces} (see Section~\ref{sec:weight}). Let \(\mu\) be a probability measure then the associated weighted halfspaces give the following decomposition of the set of halfspaces
\[
\mathcal{H} = H_\mu^- \sqcup H_\mu \sqcup H_\mu^+.
\]
It turns out that if \(H_\mu = \varnothing\) then \(H_\mu^+\) is an ultrafilter. This would give the desired map from \(\mathcal{P}(\bar X)\) to \(\bar X\). The main work consists is then to show that \(H_\mu\) indeed vanishes for every \(\mu\) in the image of \(\psi\). For this part to work, we need to introduce further restrictions. We need our complex \(X\) to be \enquote{indecomposable} (i.\,e.\ \emph{irreducible}, see Section~\ref{sec:complex}) and finite-dimensional. Furthermore, the group action on the complex needs to be well-behaved. This is encoded in two properties. First, the group needs to act \emph{essential} which means that \(\Gamma\) needs to be well-behaved with regard to the combinatorial structure of \(X\). Secondly, the group needs to act \emph{non-elementary} which means that \(\Gamma\) needs to be well-behaved with regard to the CAT(0) structure of \(X\). The details of both notions can be found in Section~\ref{sec:special}. With all these definitions in place, we will first be able to show that \(H_\mu\) is always finite. Then, in a second step, we will see that finiteness always implies emptiness. This closes the main proof. As a last step, we will see that the image of \(\phi\) indeed is in the Roller \emph{boundary} not only in the Roller \emph{compactification}. In all the steps of the proof ergodicity or ergodicity with coefficients will play a crucial role.

\subsubsection*{Chapter outline}

In Chapter~\ref{sec:complexes}, we will introduce CAT(0) cube complexes. We will start with some metric preliminaries before introducing general CAT(0) spaces. Most important in this early part is the definition of the \emph{visual boundary} (see Definition~\ref{defin:visual}). Afterwards, we will introduce cube complexes and combinatorial maps, which are the structure-preserving mappings for our objects. We will give a combinatorial property (Gromov's link condition, see Theorem~\ref{thm:link}) to check the CAT(0) property for cube complexes. Afterwards, we will talk about hyperplanes and halfspaces and some of their important properties (convexity, non-empty intersections, countability).

In Chapter~\ref{sec:roller}, we will first introduce \emph{pocsets}. Pocsets are partially ordered sets admitting a fixed point free, order reversing involution. It turns out that the set of halfspaces of CAT(0) cube complex is always a (discrete) pocset. \textcite{Roller1999} proved the reverse, namely that every (discrete) pocset gives rise to a unique CAT(0) cube complex with this pocset as pocset of halfspaces. The main ingredient of this construction is the notion of an \emph{ultrafilter} which we introduce next. The important observation was that there is a one-to-one correspondence between principal ultrafilters and the vertex set of the CAT(0) cube complex. However, Roller went further and noted that the set of all ultrafilters equipped with a natural topology is a compactification of the vertex set of every CAT(0) cube complex. This lead to the definition of one of our main objects of study: the \emph{Roller compactification} of a CAT(0) cube complex. We will introduce some topological and metric results regarding this space. Afterwards, we will revisit ultrafilters and introduce a second (equivalent) viewpoint, which is more natural for the later arguments. Lastly, we turn towards applications and introduce \emph{intervals} of ultrafilters. These are special (sub-)complexes which have strong restrictions when it comes to sets of halfspaces. In particular, we will see that sets of halfspaces can at most contain finitely many \emph{terminal elements} (i.\,e.\ minimal or maximal elements with regard to the partial order and allowing for switching by the involution). This property is the main reason we are interested in the intervals. We will close the chapter by studying sub-pocsets of halfspaces. We will answer the question under which conditions the associated CAT(0) cube complex can be embedded into CAT(0) cube complex associated to the actual pocset (Lemma~\ref{lem:lifting}). 

After these two chapters, we will shortly leave the realm of CAT(0) cube complexes and (in Chapter~\ref{sec:measure}) turn towards measure theory and functional analysis. We will start with generalities concerning \emph{measurable spaces, measurable maps} and \emph{(probability) measures}. We will recall the connection between the vector space of continuous functions and the vector space of (signed) measures. However, with these generalities in place, we return to our special case and introduce \emph{weighted halfspaces} (see Definition~\ref{defin:weight}). Lastly, we will prove measurability for certain key maps. 

Up to this point, we did not talk about group actions. This will be remedied in Chapter~\ref{sec:group}. First, we will talk about groups acting on CAT(0) cube complexes. We will see, how the group action can be extended to the Roller compactification and we will introduce the notions of \emph{essential} and \emph{non-elementary} group actions citing some important results by \textcite{Caprace2010}. The second half of the chapter is concerned with the introduction of \emph{strong \(\Gamma\)-boundaries} (where \(\Gamma\) is a second countable, locally compact group). One essential ingredient for this boundary as well as for the proof in general is \emph{ergodicity}. Hence, we have a whole section reserved for this topic. The most important results are that
\begin{itemize}
\item if we have a finite group acting ergodically on a space \(B\) then \(B\) is purely atomic (Lemma~\ref{lem:ergodic-atomic});
\item ergodicity is inherited by finite index subgroups (Lemma~\ref{lem:4.3}).
\end{itemize}
Afterwards, we will strengthen the notion of ergodicity to \emph{ergodicity with coefficients}. Both notions can be defined via requiring certain \(\Gamma\)-equivariant, measurable maps to be constant. Whereas in the case of ergodicity these maps always have \(\R\) as codomain (with the trivial action by \(\Gamma\)), in the case of ergodicity with coefficients we allow any separable Banach space that admits a unitary \(\Gamma\)-action. This stronger version of ergodicity leads to a condition for the essentiality of the \(\Gamma\)-action (see Corollary~\ref{cor:4.5}), which we will later use. Next, we turn towards \emph{amenable group actions}, which guarantee the existence of certain measurable maps, which are \(\Gamma\)-equivariant almost everywhere. With this notion in place we can define the strong \(\Gamma\)-boundaries, which are special probability spaces on which \(\Gamma\) acts amenably and ergodically with coefficients. We close the chapter with the result that (thanks to the amenability) we find the map \(\psi\) in Equation~\eqref{eq:psi} (Corollary~\ref{cor:p(x)}).

Chapter~\ref{sec:map} contains the statement and the proof of our main result (Theorem~\ref{thm:4.1} and Corollary~\ref{cor:4.2}). We will first construct the boundary map using the further assumption that \(H_\mu\) vanishes (as described above). Furthermore, we will prove that if the map exists then it will take values in the Roller boundary. We will then see that \(H_\mu\) to be finite already implies it to be empty. The remainder of the chapter build up the necessary tools to exclude \(H_\mu\) to be infinite. The most important results being Proposition~\ref{prop:4.10}, Proposition~\ref{prop:4.17} and Corollary~\ref{cor:4.21}. Finally, we descend into the main proof plugging all the previous results together.

As a closing remark, we would like to point out that the material in this thesis by nature very close to the exposition in \textcite{MR3509968}.
%%% Local Variables:
%%% mode: latex
%%% TeX-master: "../Master"
%%% ispell-local-dictionary: "en_US"
%%% End:
