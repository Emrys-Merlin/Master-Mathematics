\section{The Roller boundary}
\label{sec:roller}

\begin{prop}
  \(\mathcal{U}(A) \subset P(A)\) is a compact space.
\end{prop}

\begin{proof}
  Since \(P(A)\) is already compact it suffices to show, that \(\mathcal{U}(A)\) is closed in \(P(A)\). However, the complement
  \begin{align*}
    P(A) \setminus \mathcal{U}(A)
    & = \{\alpha \in P(A) \mid \exists a, b \in A\colon a \in \alpha,\ b \notin \alpha,\ a \subset b\}\\
    & = \bigcup_{a \in A} \{\alpha \in P(A) \mid a \in \alpha,\ \exists b \in A\colon a \subset b,\ b \notin \alpha\}\\
    & = \bigcup_{a \in A} \{\alpha \in P(A) \mid a \in \alpha,\ \exists b \in A\colon a \subset b,\ b^\ast \in \alpha\}\\
    & = \bigcup_{a \in A} \bigcup_{a \subset b}\{\alpha \in P(A) \mid a \in \alpha,\ b^\ast \in \alpha\}\\
    & = \bigcup_{a \in A} \bigcup_{a \subset b}\mathcal{U}(a, b^\ast)
  \end{align*}
  is open, which proves the claim.
\end{proof}

\begin{defin}
  A ultrafilter \(\alpha\) satisfies the \emph{descending chain condition}, if all descending chains in \(\alpha\) become stationary.
\end{defin}

\begin{thm}
  Let \(X\) be a CAT(0) cube complex and \(V(X)\) its vertex set with associated pocset \((\mathfrak{H}, \subset, ^\ast)\).
  \begin{enumerate}
  \item For every vertex \(v \in V(X)\) we have that
    \[
      \alpha_v \coloneqq \{\mathfrak{h} \in \mathfrak{H}(X) \mid v \in \mathfrak{h}\}
    \]
    is an ultrafilter. Furthermore, it satisfies the descending chain condition and ultrafilter satisfying the descending chain condition arises in this way.
  \item The map \(V(X) \to \mathcal{U}(X) \coloneqq \mathcal{U}(\mathfrak{H}),\ v \mapsto \alpha_v\) is injective, continuous and the image lies dense in \(\mathcal{U}(X)\). 
  \item If \(X\) is locally finite, then the image of \(V(X)\) is open in \(\mathcal{U}(X)\).
  \end{enumerate}
\end{thm}

%%% Local Variables:
%%% mode: latex
%%% TeX-master: "../Master"
%%% End:
