\subsection{Halfspaces}
\label{sec:halfspaces}

This section follows closely the lecture notes by \textcite{Rolli2012}.

\begin{defin}
  Let \(X\) be a cube complex
  \begin{itemize}
  \item 0-, 1- and 2-dimensional cubes are respectively called \emph{vertices, edges and squares}.
  \item A \emph{square relation} on the set of edges is given by \(e \sim e'\) if and only if there is a sequence of edges \(e_1, \dots, e_n\), such that \(e_1 = e\) and \(e_n = e'\) and such that any two edges \(e_{i-1}, e_i\) are opposite edges in a common square in \(X\). If we allow for an edge to be opposite to itself, this yields an equivalence relation of the set of all edges.
  \item A midcube \(M \subset X\) is \emph{transversal} to a square relation class \(E = [e]_\sim\), write \(M \pitchfork E\), if \(M \cap X^{(1)}\) contains only midpoints of edges in \(E\).
    \todo{define midcubes}
  \item The \emph{halfplane} defined by \(E\) is given by
    \begin{align*}
      \mathfrak{\hat h}(E) \coloneqq \bigcup_{M \pitchfork E} M \subset X.
    \end{align*}
    If the equivalence class \(E\) is not important, we will often write \(\mathfrak{\hat h}\) instead of \(\mathfrak{\hat h}(E)\).
  \end{itemize}
\end{defin}

\begin{prop}[Convexity of halfspaces,{\cite[Propositions 18 \& 19]{Rolli2012}}]
  Let \(X\) be a CAT(0) cube complex and \(\halfplane{h} \subset X\) be a hyperplane. Then \(\halfplane{h}\) is closed and convex. Furthermore, if \(\halfplane{h}\) contains at least two points of the image of any geodesix \(\gamma\), then the whole image of \(\gamma\) is contained in \(\halfspace{h}\).
\end{prop}

\begin{cor}
  Let \(X\) be a CAT(0) cube complex. Every \(\halfspace{h} \subset X\) is itself a CAT(0) cube complex.
\end{cor}

\begin{thm}[Separation, {\cite[Proposition 21]{Rolli2012}}]
  Any halfplane \(\halfplane{h}\) separates \(X\) in exactly two connected components.
\end{thm}

\begin{defin}
  The two connected components of \(X \setminus \halfplane{h}\) are called \emph{halfspaces}. If \(\halfspace{h} \subset X \setminus \halfplane{h}\) is one of these halfspaces, then \(\halfspace{h}^\ast\) denotes the opposite halfspace leading to \(X = \halfspace{h}\, \dot \cup\, \halfplane{h}\, \dot \cup\, \halfspace{h}^\ast \).
\end{defin}

\begin{thm}[Intersection, {\cite[Proposition 22 \& 24]{Rolli2012}}]
  \begin{enumerate}
  \item Let \(\halfplane{h}_1, \dots, \halfplane{h}_n\) be halfplanes with with pairwise non-trivial intersection. Then
    \begin{align*}
      \bigcap_{i=1}^n \halfplane{h}_i \neq \varnothing.
    \end{align*}
  \item Let \(\halfspace{h}_1, \dots, \halfspace{h}_n\) be halfspaces with pairwise non-trivial intersection. Then
    \begin{align*}
      \bigcap_{i=1}^n \halfspace{h}_i \neq \varnothing.
    \end{align*}
    In particular, the intersection contains a vertex of \(X\).
  \end{enumerate}
\end{thm}

\todo{Überleitung}

\begin{cor}
  If \(X\) is a locally finite CAT(0) cube complex. Then its set of hyperplanes \(\mathcal{\hat H}\) and its set of haflspaces \(\mathcal{H}\) are countable.
\end{cor}

\begin{proof}
  We fix a vertex \(x_0 \in V(X)\) and consider the sets
  \[
    Y_n \coloneqq \{(x,y) \in X_{n-1} \times X_{n} \mid y \in N(x)\} \subset X_{n-1} \times X_n \quad \forall n \in \N.
  \]
  By Lemma~\ref{lem:lf-countable} these are countable and we have that
  \[
    \mathcal{\hat H} = \bigcup_{n=1}^\infty \bigcup_{e \in Y_n} \mathfrak{\hat h}([e])
  \]
  is countable. Since every hyperplane has exactly two halfspaces associated to it the same is true for \(\mathcal{H}\).
\end{proof}

%%% Local Variables:
%%% mode: latex
%%% TeX-master: "../Master"
%%% End:
