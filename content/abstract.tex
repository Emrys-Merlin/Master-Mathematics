\selectlanguage{german}

\subsubsection*{Zusammenfassung}
\addcontentsline{toc}{section}{Abstract}%
\label{sec:Zusammenfassung}
\todo{Zusammenfassung}
\selectlanguage{english}

\subsubsection*{Abstract}
\label{sec:abstract}

In this thesis we will try to understand the boundary map between the strong group boundary \(B\) of a group \(\Gamma\) into the Roller boundary \(\partial X\) of a CAT(0) cube complex \(X\) on which \(\Gamma\) operates by automorphisms:
\[
  \phi\colon B \to \partial X.
\]
This boundary map was first constructed by \textcite{MR3509968} in their endeavor to define a certain non-vanishing cohomology class (the median class) of the second bounded cohomology of the automorphism group \(\Aut(X)\) of a finite dimensional CAT(0) cube complex. This lead to a superrigidity result in the spirit of \textcite{MR1090825}.

We will follow the steps of Chatterji, Fernós and Iozzi and give an introduction to all the necessary objects on their way to the boundary map. We will introduce CAT(0) cube complexes \(X\), talk about their geometry and their combinatorics, which will lead to their pocset of halfspaces \(\mathcal{H}(X)\), ultrafilters and the Roller compactification \(\bar X\). Afterwards, we will lie the necessary measure theoretic ground work and talk about groups. In particular, we will define the notion of essential and non-elementary group actions on CAT(0) cube complexes. One of the most important tools is in the proof is ergodicity (with coefficients), which will also be handled. Lastly, we define strong group boundaries, which will put us in a position to plug all the scattered parts together in order to construct the boundary map.

%%% Local Variables: 
%%% mode: latex
%%% TeX-master: "../Master"
%%% ispell-local-dictionary: "en_US"
%%% End: 
