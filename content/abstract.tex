\selectlanguage{german}

\subsubsection*{Zusammenfassung}
\addcontentsline{toc}{section}{Abstract}%
\label{sec:Zusammenfassung}
Diese Abschlussarbeit hat zum Ziel eine sogenannte \emph{Randabbildung} zwischen einem starken \(\Gamma\)-Rand \(B\) einer diskreten, abzählbaren Gruppe \(\Gamma\) und dem Roller-Rand \(\partial X\) eines CAT(0) Kubenkomplexes \(X\) auf dem \(\Gamma\) via Autmorphismen operiert:
\[
  \phi\colon B \to \partial X.
\]
Wir werden sehen, dass diese Randabbildung messbar und fas überall \(\Gamma\)-equivariant ist. Die Existenz dieser Abbildung wurde als erste von \textcite{MR3509968} bewiesen; unter den zusätzlichen Annahmen, dass \(X\) zusammenhängend, lokal abzählbar und endlich dimensional und die Gruppenwirkung von \(\Gamma\) auf \(X\) nicht-elementar ist.

Diese Arbeit hat einen einführenden Charakter. Wir werden zunächst eine kurze Einführung in CAT(0) Kubenkomplexe und uns anschließend mit der Roller-Dualität auseinandersetzen, die direkt zum Roller-Rand führt. Zusätzlich werden wir Gruppenoperationen auf CAT(0) Kubenkomplexen untersuchen und dabei die Begriffe der Nicht-Elementarität und Essentialität einführen. Schlussendlich werden wir ergodische Gruppenoperationen (mit Koeffizienten) und starke \(\Gamma\)-Ränder einführen.

\selectlanguage{english}

\subsubsection*{Abstract}
\label{sec:abstract}

In this thesis we will expose the construction of a so-called \emph{boundary map} between the strong \(\Gamma\)-boundary \(B\) of a discrete, countable group \(\Gamma\) and the Roller boundary \(\partial X\) of a CAT(0) cube complex \(X\) on which \(\Gamma\) acts by automorphisms:
\[
  \phi\colon B \to \partial X.
\]
This boundary map will be seen to be measurable and \(\Gamma\)-equivariant almost everywhere. The existence was first proven by \textcite{MR3509968} under the furhter assumption that \(X\) is connected, locally finite and finite-dimensional and that \(\Gamma\) acts non-elementary on \(X\). 

This thesis has an expository nature. We will give a brief introduction to CAT(0) cube complexes and then turn towards the Roller duality, which will lead us immediately to the Roller boundary. Additionally, we will explore group actions on CAT(0) cube complexes introducing the notions of non-elementarity and essentiality. Lastly, we will define ergodic group actions (with coefficients) and strong \(\Gamma\)-boundaries.

%%% Local Variables: 
%%% mode: latex
%%% TeX-master: "../Master"
%%% ispell-local-dictionary: "en_US"
%%% End: 
