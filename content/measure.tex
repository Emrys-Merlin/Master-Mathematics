\section{Measure theoretic preliminaries}
\label{sec:measure}

\subsection{Properties of probability measures on \(\bar X\)}
\label{sec:prob}

In this section I am concerned with some general properties of (probability) measures on the Roller compactification. These properties will be needed in the construction of the boundary map, especially for the construction of a map from the strong \(\Gamma\)-boundary \(B\) to \(P(\bar X)\) via the amenable action of \(\Gamma\) on \(B\). Thus we need to see that \(P(\bar X)\) is weak-\(\ast\) compact, convex and non-empty and contained in the unit ball of the dual of some topological vector space on which we have a group action by isometries of \(\Gamma\).

I will begin by collecting some important measure theoretic and functional analytic results

\begin{defin}
  \label{def:vanishing}
  Let \(X\) be a topological space. The vector space \(C_0(X)\) of \emph{continuous functions vanishing at infinity} is defined via
  \[
    C_0(X) \coloneqq \{f \in C(X) \mid \forall \epsilon > 0 \exists K \subset X \text{ compact}\colon f|_{X\setminus K} < \epsilon\}.
  \]
\end{defin}

\begin{defin}
  Let \((X, \Sigma)\) be a measure space. A map \(\mu \colon \Sigma \to \R\) is called a \emph{signed measure} if it is \(\sigma\)-additive, i.\,e.
  \[
    \mu\left(\bigcup_{i \in \N}A_i \right) = \sum_{i \in \N} \mu(A_i)
  \]
  for arbitrary \(A_i \in \Sigma\) such that \(A_i \cap A_j = \varnothing\) whenever \(i \neq j\). This is meant in the sense that the right hand side needs to converge.

  For any \(A \in \Sigma\) the \emph{total variation} \(|\mu|(A)\) is defined as
  \[
    |\mu|(A) \coloneqq \sup \left\{\sum_{i \in \N} \mu(A_i) \relmid A_i \in \Sigma \text{ and } A = \dot \bigcup_{i \in \N} A_i \right \}.
  \]
  The \emph{positive} and \emph{negative variation} are defined as
  \[
    \mu^\pm \coloneqq \frac12 (|\mu| \pm \mu).
  \]
  We call \(\mu\) a \emph{measure}, if \(\mu(A) \geq 0\) for every \(A \in \Sigma\) and a \emph{probability measure}, if it is a measure and \(\mu(X) = 1\). If \(\Sigma\) is the Borel \(\sigma\)-algebra of a topology on \(X\), then a measure \(\mu\) is called Borel, if every \(x \in X\) has an open neighborhood \(U \subset X\) such that \(\mu(U) > 0\).
  
  A Borel measure \(\mu\) is called \emph{inner regular} if \(\mu(A) = \sup \{ \mu(K) \mid K \subset A \text{ compact}\}\) and \emph{outer regular} if \(\mu(A = \inf \{ \mu{U} \mid U \supset A \text{ open}\})\). If it is both it is called \emph{regular}. A signed Borel measure \(\mu\) is called \emph{regular} if \(|\mu|\) is regular in the before mentioned sense (This makes sense because of the next proposition).
\end{defin}

\begin{rem}
  In this thesis we are mostly interested in probability measures and hence our measures are always finite. Thus we defined signed measures and the like only with image in \(\R\), ignoring infinities. Since we have this restrictions in place, the above definition of measures in fact satisfies all the standard conditions, in particular \(\mu(\varnothing) = 0\).
\end{rem}

\begin{prop}[{\cite[Ch.\ 6.1]{RudinFunctional}}]
  \(|\mu|\) and \(\mu^\pm\) are measures on \((X, \Sigma)\) and \(|\mu|(X) < \infty\) holds.
\end{prop}

\begin{thm}[Riesz-Markow representation, {\cite[Thm 6.19]{RudinFunctional}}]
  \label{thm:riesz-markow}
  If \(X\) is a locally compact Hausdorff space, then every bounded linear functional \(\Phi\) on \(C_0(X)\) is represented by a unique regular complex Borel measure \(\mu\) in the sense that \[
    \Phi f = \int_X f \d\mu
  \]
  for every element \(f \in C_0(X)\). Moreover, there is \(\|\Phi\| = |\mu|(X)\). In other words, there exists an isometry of normed vector spaces between \(X^\ast\) the dual of \(X\) equipped with the operator norm \(\|\cdot \|\) and \(M_{s}(X)\) the space of signed measures equipped with total variation \(|\mu|(X)\) as norm.
\end{thm}

\begin{thm}[Banach-Alaoglu, {\cite{RudinAnalysis}}]
  If \(V\) is a neighborhood of 0 in a topological vector space \(X\) and if
  \[
    K \coloneqq \{ \Phi \in X^\ast \mid |\Phi x| \leq 1 \quad \forall x \in V\},
  \]
  then \(K\) is weak\(\ast\)-compact.
\end{thm}

\begin{cor}
  \label{cor:banach-alaoglu}
  If \(X\) is a compact metric space, then the space of all regular probability measures \(\mathcal{P}(X)\) is weak\(\ast\)-compact and contained in the unit ball of all signed measures \(M_{s}(X)\).
\end{cor}

\begin{proof}
  Considering \(C_0(X)\) together with the supremum norm is a Banach space and choosing the unit ball as \(V\), we yield that the unit ball \(B \subset M_s(X) \cong C_0(X)^\ast\) is weak\(\ast\)-compact. Since for each probability measure \(\mu\), we have \(|\mu|(X) = \mu(X) = 1\) it follows that \(P(X) \subset B\). Thus we only need to show that \(P(X)\) is weak\(\ast\)-closed in \(B\) and we are done. However,
  \begin{align*}
    P \coloneqq & \left\{\mu \in M_s(X) \relmid \int_X f \d\mu \geq 0 \quad \forall f \geq 0\right\},\\
    N \coloneqq & \left\{\mu \in M_s(X) \relmid \int_X \d\mu = \int_X \chi_X \d\mu = 1\right\},s\\
    \mathcal{P}(X)  = & B \cap P \cap N.
  \end{align*}
  The second set assures that the measure is positive and the last enforces the normalization. This are all the necessary restrictions to a probability measure. Additionally, these sets are clearly weak\(\ast\)-closed.
\end{proof}

\begin{thm}[Mackey's point realization, {\cite[330]{Mackey1962}}]
  \label{thm:mackey}
  Let \((M, \sigma, \vartheta)\) be a standard probability space. Let a locally compact, second countable group \(\Gamma\) act on \(M\) by measurable transformations. Let \(\Lambda\) be a subs-sigma-algebra of \(\Sigma\) which is \(\Gamma\)-invariant, and such that for any \(A \in \Lambda\) and \(g \in \Gamma\) \(\vartheta(A) = 0 \) if and only if \(\vartheta(gA) = 0\). Then there exists a standard probability \(G\)-space \((M', \Sigma', \vartheta')\) and a \(\Lambda\)-measurable, \(\Gamma\)-equivariant map \(p \colon M \to M'\) such that \(p_\ast \vartheta = \vartheta'\).
\end{thm}\todo{so far only in lecture notes on poisson boundary. Find better source.}


\subsection{Weighted halfspaces}
\label{sec:weight}

\begin{defin}
  Let \(\mathcal{P}(\bar X)\) denote the space of probability measures on \(\bar X\). If \(\mu \in \mathcal{P}(\bar X)\) define
  \begin{align*}
    H_\mu^{\phantom{+}} \coloneqq &\ \{h \in \mathcal{H}(X) \mid \mu(\mathcal{C}(h)) = \mu(\mathcal{C}(h^\ast))\},\\
    H_\mu^+ \coloneqq &\ \{h \in \mathcal{H}(X) \mid \mu(\mathcal{C}(h)) > \text{\nfrac{} 1/2} \},\\
    H_\mu^- \coloneqq &\ \{h \in \mathcal{H}(X) \mid \mu(\mathcal{C}(h)) < \text{\nfrac{} 1/2}\} \text{ and}\\
    H_\mu^\pm \coloneqq &\ \{h \in \mathcal{H}(X) \mid \mu(\mathcal{C}(h)) \neq \text{\nfrac{} 1/2}\}.
  \end{align*} 
  The above four sets are called \emph{balanced, heavy, light} and \emph{unbalanced halfspaces} respectively.
\end{defin}

\begin{lemma}[{\cite[Lem.\ 4.6]{MR3509968}}]
  \label{lem:4.6}
  Let \(\mu,\nu \in \mathcal{P}(\bar X)\)
  \begin{enumerate}
  \item \(H_\mu\) is closed under involution. Also, involution is a bijection between \(H_\mu^+\) and \(H_\mu^-\).
  \item There is the following partition \(\mathcal{H}(X) = H_\mu \sqcup H_\mu^\pm = H_\mu \sqcup H_\mu^+ \sqcup H_\mu^-\).
  \item If \(\mathfrak{h, k} \in H_\mu\) (resp.\ \(H_\mu^+\) or \(H_\mu^-\)), then either \(\mathfrak{h} \pitchfork k\) or all halfspaces between \(\mathfrak{h}\) and \(\mathfrak{k}\) lie in \(H_\mu\) (resp.\ \(H_\mu^+\) or \(H_\mu^-\)).
  % \item There are no facing triples\todo{facing triples} of halfspaces in \(H_\mu\). If \(X\) is not Euclidean\todo{Euclidean}, then \(H_\mu^+ \neq \varnothing\).\todo{check if I need this and if not remove it}
  \item If \(H_\mu\) and \(H_\nu\) are not empty and \(H_\mu \cap H_\nu = \varnothing\), then \(H_\mu \cap H_\nu^+ \neq \varnothing\) and \(H_\mu \cap H_\nu^- \neq \varnothing\).
  \item If \(\mathfrak{h, k}  \in H_\mu\) are two parallel halfspaces with \(\mathfrak{h} \subset \mathfrak{k}\), then \(\mu)(\mathfrak{h}^\ast \cap \mathfrak{k}) = 0\).
  \item The assignments \(\mu \mapsto H_\mu\), \(\mu \mapsto H_\mu^+\) and \(\mu \mapsto H_\mu^-\) are \(\Aut(X)\)-equivariant for the natural actions on \(\mathcal{P}(\bar X)\) and \(\operatorname{Pot}(\mathcal{H}(X))\).
  \end{enumerate}
\end{lemma}
\todo{check if I need \(H_\mu^+ \neq \varnothing\) if not euclidean and if I need facing triples.}

\begin{proof}
  1.\ -- 3.\ are clear from the definitions and the additivity of the measure. For 4.\ we see that \(H_\mu \subset H_\nu^+ \cap H_\nu^-\). Since \(H_\mu\) is invariant under involution, but \(H_\nu^+\) and \(H_\nu^-\) get interchanged, we see that both intersections cannot be empty. The last assertion follows again easily from the definitions.
\end{proof}


\subsection{Measurability of certain maps}
\label{sec:meas-maps}

In the main proof of our theorem ergodicity will play a central role. We will mostly use it in the form that measurable \(\Gamma\)-invariant maps have to be essentially constant. Hence, we need all our important maps to be measurable. These proofs are mainly technical and not very enlightening, which is why they were exported to this section.

\begin{lemma}
  \label{lem:tau}
   Let \(\tau\colon \operatorname{Pot}(\mathcal{H}) to \operatorname{Pot}(\mathcal{H})\) the map assigning to each subset of \(\mathcal{H}\) its set of terminal elements (c.\,f.~Definition~\ref{defin:tau}). Then \(\tau\) is measurable.
\end{lemma}

\begin{proof}
  We take an arbitrary cylinder set \(C(F_1, F_2)\) and are interested in the preimage
  \begin{align*}
    \tau^{-1}(C(F_1, F_2))
    & = \{H \subset \mathcal{H} \mid F_1 \subset \tau(H) \text{ and } \forall \mathfrak{h} \in F_2\colon \mathfrak{h} \in H\ \Rightarrow\ \mathfrak{h} \not\in \tau{H}\} \\
    & = \{H \subset \mathcal{H} \mid F_1 \subset \tau(H)\} \cap \{H \subset \mathcal{H} \mid \forall \mathfrak{h} \in F_2\colon \mathfrak{h} \in H\ \Rightarrow\ \mathfrak{h} \not\in \tau(H)\}\\
    & \eqqcolon T \cap N.
  \end{align*}
  Considering the first set \(T\) we have the following decomposition:
  \begin{align*}
    T
    & = \bigcap_{\mathfrak{h} \in F_1}\{H \subset \mathcal{H} \mid \mathfrak{h} \in \tau(H)\}\\
    & = \bigcap_{\mathfrak{h} \in F_1}\left (\{H \subset \mathcal{H} \mid \mathfrak{h} \in H \text{ minimal}\} \cup \{H \subset \mathcal{H} \mid \mathfrak{h} \in H \text{ maximal}\}\right )\\
    & = \bigcap_{\mathfrak{h} \in F_1}\left ( \bigcap_{\substack{\mathfrak{k} \in \mathcal{H}\\\mathfrak{k} \subset \mathfrak{h}}}C(\{\mathfrak{h}\}, \{\mathfrak{k}\}) \cup \bigcap_{\substack{\mathfrak{k} \in \mathcal{H}\\\mathfrak{h} \subset\mathfrak{k}}} C(\{\mathfrak{h}\}, \{\mathfrak{k}\})\right ).
  \end{align*}
  This set is measurable as it is a countable intersection and union of measurable sets. Let us turn towards \(N\):
  \begin{align*}
    N
    & = \bigcup_{F \subset F_2} \{H \subset \mathcal{H} \mid F \subset H \setminus \tau(H) \text{ and } (F_2 \setminus F) \cap H = \varnothing\}\\
    & = \bigcup_{F \subset F_2} \left ( \{H \subset \mathcal{H} \mid F \subset H \setminus \tau(H)\} \cap C(\varnothing, F_2 \setminus F)\right).\\
  \end{align*}
  If we can show that the set, where a finite subset of elements are not allowed to be terminal is measurable, we are done. This can be achieved via an induction over \(n = |F|\). The case \(n = 0\) is clear, since any \(\sigma\)-algebra needs to contain the whole set. Assume the assertion is true for every finite subset \(F \subset \mathcal{H}\) of \(n\) elements. If \(F\) were now to contain \(n+1\) elements, fixing \(\mathfrak{h} \in F\) and \(\tilde F = F \setminus \{\mathfrak{h}\}\), we could do the following decomposition:
  \begin{align*}
    \{H \subset \mathcal{H} \mid F \subset \mathcal{H} \setminus \tau(H)\}
    & = \{H \subset \mathcal{H} \mid \{\mathfrak{h}\} \cup \tilde F \subset H \setminus \tau(H)\}\\
    & = \{H \subset \mathcal{H} \mid \tilde{F} \subset H \setminus \tau(H)\} \setminus \{H \subset \mathcal{H} \mid \mathfrak{h} \in \tau(H)\}.
  \end{align*}
  The first set is measurable by induction hypothesis, the second one is measurable by our computations above. All in all we yield that \(\tau^{-1}(C(F_1, F_2))\) is measurable.
\end{proof}

\begin{lemma}
  \label{lem:measurable-countable}
  Let \(N\) be a countable set. Then
  \[
    f\colon \operatorname{Pot}(N) \to \N \cup \{\infty\},\ A \mapsto |A|
  \]
  is measurable, where \(\N \cup \{\infty\}\) is equipped with the discrete topology.
\end{lemma}
\begin{proof}
  We have to show that a basis of the topology is mapped to measurable sets via the preimage. So let us consider first \(n \in \N\). Then we have
  \begin{align*}
    f^{-1}(\{n\})
    & = \{ A \subset N \mid |A| = n\}\\
    & = \bigcup_{\substack{A \subset N\\|A| = N}} \left ( \bigcap_{\substack{F \subset N\\|F|< \infty}}C(A,F)\right),
  \end{align*}
  where \(C(A,F)\) is a cylinder set as defined in Definition~\ref{def:pot-top}. Since the set of all finite subsets of \(N\) is countable, the above preimage is measurable as a countable union and intersection of measurable sets.

  Lastly, we have to consider \(f^{-1}(\{\infty\})\). However, here we have
  \[
    f^{-1}(\{\infty\}) = N \setminus \left (\bigcup_{n=0}^\infty f^{-1}(\{n\})\right)
  \]
  which is measurable as the complement of a measurable set.
\end{proof}

\begin{lemma}[{\cites[Lem.\ A.1]{MR3509968}}]
  \label{lem:measurable-mu}
  Let \(I \subset [0,1]\) be a subinterval that is either open, closed or half open. Let \(H^I_\mu \coloneqq \{h \in \mathcal{H}(X) \mid \mu(\mathcal{C}(h)) \in I\}\). Then the map
  \[
    \mathcal{P}(\bar X) \to \operatorname{Pot}(\mathcal{H}(X)), \mu \mapsto H^I_\mu
  \]
  is measurable with respect to the weak\(\ast\)-topology on \(\mathcal{P}(X)\).
\end{lemma}

\begin{proof}
  In Section~\ref{sec:roller} we defined a topology on the power set (c.\,f.\ Definition~\ref{def:pot-top} and Proposition~\ref{prop:pot-top}). Hence, we need to show that the preimages of the basic open sets are mapped to measurable sets in \(\mathfrak{P}(\bar X)\). So let \(F_1, F_2 \subset \mathcal{H}(X)\) be finite and consider the cylinder set \(C(F_1, F_2)\). Then the preimage is given by the set
  \[
    K(F_1, F_2) = \{\mu \in \mathcal{P}(\bar X) \mid H^I_\mu \in C(F_1, F_2)\}.
  \]
  For now, let us consider the sets \(E_I(\mathfrak{h}) \coloneqq \{\mu \in \mathcal{P}(\bar X) \mid \mu(\mathcal{U}(\mathfrak{h})) \in I\}\). We would like to convince ourselves that these sets are measurable. We know that \(\bar X = \mathcal{U}(\mathfrak{h}) \sqcup \mathcal{U}(\mathfrak{h}^\ast)\). Hence, \(\mathfrak{\tilde h} \coloneqq \mathcal{U}(\mathfrak{h})\) is both open and closed in \(\bar X\). Thus the indicator function \(\chi_{\mathfrak{\tilde h}}\) is continuous. However, the weak\(\ast\)-topology is defined such that each map
  \[
    T_f\colon \mathcal{P}(\bar X) \to \R,\ \mu \mapsto \int_X f \d\mu
  \]
  is continuous (for each \(f \in C(\bar X)\)). Hence, \(T \coloneqq T_{\chi_{\mathfrak{\tilde h}}}\) is continuous and thus also measurable. Furthermore, we have that \(T^{-1}(I) = E_I(\mathfrak{h})\). \(I\) is measurable, so the same is true for \(E_I(\mathfrak{h})\).

  Together with the following observation this finishes the proof:
  \[
    K(F_1, F_2) = \left (\bigcap_{\mathfrak{h} \in F_1} E_I(\mathfrak{h}) \right ) \cap \left ( \bigcap_{\mathfrak{h} \in F_2} E_I(\mathfrak{h})^{c}\right).
  \]
\end{proof}

\begin{lemma}[{\cite[Lem. 4.7]{MR3509968}}]
  \label{lem:interval}
  The complex \(\bar X(H_\mu)\) is an interval.
\end{lemma}

\begin{proof}
  Let
  \[
    p \colon \bar X \to \bar X (H_\mu), \alpha \mapsto \alpha \cap H_\mu
  \]
  be the projection. \(p\) is continuous and we can push forward \(\mu\) along \(p\). Next, choose \(\alpha \in \supp(p_\ast \mu)\), i.\,e.\ every open neighborhood of \(\alpha\) must have non-zero measure. \(\supp(p_\ast \mu)\) is not empty as \(\mu\) is non-vanishing. We claim that \(\alpha^\ast\) is also an ultrafilter. It automatically satisfies the choice condition, so we only need to check consistency. So let \(\mathfrak{h} \in \alpha^\ast\) and \(\mathfrak{k} \in H_\mu\) such that \(\mathfrak{h} \subset \mathfrak{k}\). Assume that \(\mathfrak{k}\) were not in \(\alpha^\ast\). Then \(\mathcal{C}(\mathfrak{h^\ast}) \cap \mathcal{C}(\mathfrak{k})\) would be an open neighborhood of \(\alpha\) in \(\bar X(H_\mu)\) and we would need its measure to be positive. However, by Lemma~\ref{lem:4.6}(6) we know that the measure is indeed zero. Hence we have \(\mathfrak{k} \in \alpha^\ast\) and we can conclude the proof thanks to the previous lemma.
\end{proof}

\begin{lemma}
  The map
  \[
    f\colon \bar X \times \bar X \to \operatorname{Pot}(\mathcal{H}),\ (\alpha, \beta)\mapsto \mathcal{H}(\alpha, \beta)
  \]
  is measurable.
\end{lemma}

\begin{proof}
  We will indeed see that the map is even continuous. Take \(\mathfrak{h}_1,\dots \mathfrak{h}_n, \mathfrak{h}_1', \dots, \mathfrak{h}_l' \in \mathcal{H}\) and consider the following preimage
  \begin{align*}
    f^{-1}(\mathcal{C}(\{\mathfrak{h}_1, \dots, \mathfrak{h}_n\}, \{\mathfrak{h}_1', \dots \mathfrak{h}_l'\}))  = & \{(\alpha, \beta) \in \bar X \times \bar X \mid \forall i \leq n\colon \mathfrak{h}_i \in \alpha \wedge \mathfrak{h}_i^\ast \in \beta\\
    & \wedge \forall j \leq l \colon \mathfrak{h}_j'^\ast \in \alpha \vee \mathfrak{h}_j' \in \beta\}\\
    = & (\mathcal{C}(\{\mathfrak{h}_1, \dots \mathfrak{h}_n\}, \varnothing)  \times \bar X)\\
    & \cap (\bar X \times C(\{\mathfrak{h}_1^\ast, \dots, \mathfrak{h}_n^\ast\}))\\
                                                                                                                  & \cap \left ( \bigcap_{j=1}^l \left[(\mathcal{C}(\{h\}_j'^\ast, \varnothing) \times \bar X) \cup (\bar X \times \mathcal{C}(\{\mathfrak{h}_j'\}, \varnothing))\right]\right).
  \end{align*}
  This set is open as a finite intersection of basic open sets. Hence \(f\) is continuous and measurable.
\end{proof}


%%% Local Variables:
%%% mode: latex
%%% TeX-master: "../Master"
%%% End:
