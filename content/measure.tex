\section{Measuretheoretic preliminaries}
\label{sec:measure}

\begin{defin}
  Let \((M, \Sigma, \theta)\) be a probability space and \(T\colon M \to M\) a measurable and measure-preserving (\(\theta(A) = \theta(T^{-1}(A)) \ \forall A \in \Sigma\)) map. \(T\) is called ergodic, if for every \(A \in \Sigma\ T^{-1}(A) = A\) implies that \(\theta(A) \in \{0, 1\}\).
\end{defin}

\begin{thm}[Mackey's point realization, {\cite[330]{Mackey1962}}]
  Let \(\Gamma\) be a separable locally compact group anad let \(B\) be a Boolean \(\Gamma\)-space which is standard as a \(\sigma\)-Boolean algebra. Then there exists a Borel \(\Gamma\)-space \(S\) and a finite quasi-invariant measure \(\mu\) on \(S\) such that \(B\) is equivalent to the Boolean  \(\Gamma\)-space associated with \(S\) and \(\mu\).
\end{thm}
\todo{look up what 'standard' means in this context}

\todo{I replaced measure space by probability space. Should be enough and I am not sure if it works in the general case}
\begin{lemma}[{\cite[Lem 4.3]{MR3509968}}]
  Let \(\Gamma\) be a group acting on a measure space \((M, \vartheta)\). If \(\Gamma\) acts ergodically \((M \times M, \vartheta \times \theta)\), then every  finite index subgroup \(\Gamma_0 \leq \Gamma\) acts ergodically on \((M, \vartheta)\).
\end{lemma}

\begin{lemma}[{\cite[Lem. 4.4]{MR3509968}}]
  Let \(C\) be a countable set with a \(\Gamma\)-action and \((B, \vartheta)\) a Lebesgue space with a measure class preserving \(\Gamma\)-action that is in addition doubly ergodic with Hilbert coefficients. If \(\psi \colon B \times B \to C\) or \(\psi \colon B \to C\) is a \(\Gamma\)-equivariant measurable map, then \(\psi\) is essentially constant.
\end{lemma}

\begin{cor}[{\cite[Cor. 4.5]{MR3509968}}]
  Let \(\operatorname{Pot}_f(\mathcal{h}(X)) \subset \operatorname{Pot}(\mathcal{h}(X))\) be the set containing only finite subsets of \(\mathcal{h}(X)\). If there exists a \(\Gamma\)-equivariant measurable map \(\mathcal{P}(\bar X) \times \mathcal{P}(\bar X) \to \operatorname{Pot}_f(\mathcal{h}(X))\) or \(\mathcal{P}(\bar X) \to \operatorname{Pot}_f(\mathcal{h}(X))\), then the \(\Gamma\)-action on \(X\) is not essential.
\end{cor}

\subsection{Weighted halfspaces}
\label{sec:weight}

\begin{defin}
  Let \(\mathcal{P}(\bar X)\) denote the space of probability measures on \(\bar X\). If \(\mu \in \mathcal{P}(\bar X)\) define
  \begin{align*}
    H_\mu \coloneqq &\ \{h \in \mathcal{H}(X) \mid \mu(h) = \mu(h^\ast)\},\\
    H_\mu^+ \coloneqq &\ \{h \in \mathcal{H}(X) \mid \mu(h) > \text{\nfrac{} 1/2} \},\\
    H_\mu^- \coloneqq &\ \{h \in \mathcal{H}(X) \mid \mu(h) < \text{\nfrac{} 1/2}\} \text{ and}\\
    H_\mu^\pm \coloneqq &\ \{h \in \mathcal{H}(X) \mid \mu(h) \neq \text{\nfrac{} 1/2}\}.
  \end{align*}
  The above four sets are called \emph{balanced, heavy, light} and \emph{unbalanced halfspaces} respectively.
\end{defin}

\begin{lemma}[{\cite[Lem. 4.6]{MR3509968}}]
  Let \(\mu,\nu \in \mathcal{P}(\bar X)\)
  \begin{enumerate}
  \item \(H_\mu\) is closed under involution. Also, involution is a bijection between \(H_\mu^+\) and \(H_\mu^-\).
  \item There is the following partiton \(\mathcal{H}(X) = H_\mu \sqcap H_\mu^\pm = H_\mu \sqcap H_\mu^+ \sqcap H_\mu^-\).
  \item If \(h, k \in H_\mu\) (resp.\ \(H_\mu^+\) or \(H_\mu^-\)), then either \(h \pitchfork k\) or all halfspaces between \(h\) and \(k\) lie in \(H_\mu\) (resp.\ \(H_\mu^+\) or \(H_\mu^-\)).
  \item There are no facing triples\todo{facing triples} of halfspaces in \(H_\mu\). If \(X\) is not Euclidean\todo{Euclidean}, then \(H_\mu^+ \neq \varnothing\).
  \item If \(X\) is not Euclidean, \(H_\mu\) and \(H_\nu\) is not empty and \(H_\mu \cap H_\nu = \varnothing\), then \(H_\mu \cap H_\nu^+ \neq \varnothing\) and \(H_\mu \cap H_\nu^- \neq \varnothing\).
  \item If \(h, k  \in H_\mu\) are two parallel halfspaces with \(h \subset k\), then \(\mu)(h^\ast \cap k) = 0\).
  \item The assignments \(\mu \mapsto H_\mu\), \(\mu \mapsto H_\mu^+\) and \(\mu \mapsto H_\mu^-\) are \(\Aut(X)\)-equivariant for the natural actions on \(\mathcal{P}(\bar X)\) and \(\operatorname{Pot}(\mathcal{H}(X))\).
  \end{enumerate}
\end{lemma}

\begin{lemma}[{\cite[Lem. 4.7]{MR3509968}}]
  The complex \(\bar X(H_\mu)\) is an interval.
\end{lemma}

\subsection{Properties of probability measures on \(\bar X\)}
\label{sec:prob}

In this section I am concerned with some general properties of (probability) measures on the Roller compactification. These propberties will be needed in the construction of the boundary map, especially for the construction of a map from the strong \(\Gamma\)-boundary \(B\) ot \(P(\bar X)\) via the amenable action of \(\Gamma\) on \(B\). Thus we need to see that \(P(\bar X)\) is weak-\(\ast\) compact, convex and non-empty and contained in the unit ball of the dual of some topological vector space on which we have a group action by isometries of \(\Gamma\).

I will begin by collecting some important measure theoretic and functional analytic results

\begin{defin}
  Let \(X\) be a topological space. The vector space \(C_0(X)\) of \emph{continuous functions vanishing at infinity} is defined via
  \[
    C_0(X) \coloneqq \{f \in C(X) \mid \forall \epsilon > 0 \exists K \subset X \text{ compact}\colon f|_{X\setminusK } < epsilon\}.
  \]
\end{defin}

\begin{defin}
  Let \(X, \Sigma\) be a measure space. A map \(\mu \colon \Sigma \to \R\) is called a \emph{signed measure} if it is \(\sigma\)-additive, i.\,e.
  \[
    \mu\left(\bigcup_{i \in \N}A_i \right) = \sum_{i \in \N} \mu(A_i)
  \]
  for arbitrary \(A_i \in \Sigma\) such that \(A_i \cap A_j = \varnothing\) whenever \(i \neq j\). This is meant in the sence that the right hand side needs to converge.

  For any \(A \in \Sigma\) the \emph{total variation} \(|\mu|(A)\) is defined as
  \[
    |\mu|(A) \coloneqq \sup \left\{\sum_{i \in \N} \mu(A_i) \relmid A_i \in \Sigma \text{ and } A = \dot \bigcup_{i \in \N} A_i \right \}.
  \]
  The \emph{positive} and \emph{negative variation} are defined as
  \[
    \mu^\pm \coloneqq \frac12 (|\mu| \pm \mu).
  \]
  A Borel measure \(\mu\) is called \emph{inner regular} if \(\mu(A) = \sup \{ \mu(K) \mid K \subset A \text{ compact}\}\) and \emph{outer regular} if \(\mu(A = \inf \{ \mu{U} \mid U \supset A \text{ open}\})\). If it is both it is called \emph{regular}. A signed Borel measure \(\mu\) is called \emph{regular} if \(|\mu|\) is regular in the before mentioned sense (This makes sense because of the next proposition).
\end{defin}
\todo{perhaps define Borel measure}

\begin{prop}[{\cite{Rudin}}]
  \(|\mu|\) and \(\mu^\pm\) are measures on \(X, \Sigma\) and \(|\mu|(X) < \infty\) holds.
\end{prop}

\begin{thm}[Riesz-Markow representation, {\cite[Thm 6.19]{Rudin}}]
  If \(X\) is a locally compact Hausdorff space, then every bounded linear functional \(\Phi\) on \(C_0(X)\) is represented by a unique regular complex Borel measure \(\mu\) in the sense that \[
    \Phi f = \int_X f \d\mu
  \]
  for every element \(f \in C_0(X)\). Moreover, there is \(\|\Phi\| = |\mu|(X)\). In other words, there exists an isometry of normed vector spaces between \(X^\ast\) the dual of \(X\) equippend with the operator norm \(\|\dot \|\) and \(M_{s}(X)\) the space of signed measures equipped with total variation \(|\mu|(X)\) as norm.
\end{thm}

\begin{thm}[Banach-Alaoglu, {\cite{Rudin2}}]
  If \(V\) is a neighborhood of 0 in a topological vector space \(X\) and if
  \[
    K \coloneqq \{ \Phi \in X^\ast \mid \|\Phi x\| \leq 1 \quad \forall x \in V\},
  \]
  then \(K\) is weak\(\ast\)-compact.
\end{thm}

\begin{cor}
  \label{cor:banach-alaoglu}
  If \(X\) is a compact metric space, then \(P(X)\) the space of all regular probability measures is weak\(\ast\)-compact and contained in unit ball of all signed measures \(M_{s}(X)\).
\end{cor}

\begin{proof}
  Considering \(C_0(X)\) together with the supremum norm is a Banach space and choosing the unit ball as \(V\), we yield that the unit ball \(B \subset M_s(X) \cong C_0(X)^\ast\) is weak\(\ast\)-compact. Since for each probability measure \(\mu\), we have \(|\mu|(X) = \mu(X) = 1\) it follows that \(P(X) \subset B\). Thus we only need to show that \(P(X)\) is weak\(\ast\)-closed in \(B\) and we are done. However, \(P(X) = B \cap \{\mu \in M_r(X) \mid \int_X f \d\mu \geq 0 \quad \forall f \geq 0\} \cap \{\mu \in M_s(X) \mid \int_X \d\mu = \int_X \xi_X \d\mu = 1\}\). The second set assures that the measure is positive and the last enforces the normalization. This are all the necessary restrictions fo a probability measure. Additionally, these sets are clearly weak\(\ast\)-closed.
\end{proof}

%%% Local Variables:
%%% mode: latex
%%% TeX-master: "../Master"
%%% End:
