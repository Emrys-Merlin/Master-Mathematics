\section{Measure theoretic preliminaries}
\label{sec:measure}

Measure theory will play an important role in the main proof. Hence, we have this dedicated chapter introducing the necessary notation.

This chapter is divided into three sections. Section~\ref{sec:prob} collects general facts about measures, measure theory and functional analysis.
Section~\ref{sec:weight} introduces the notion of \emph{weighted halfspaces} which are central to our problem. We will highlight some of their elementary properties. Section~\ref{sec:weight} is the most important of the three.
Section~\ref{sec:meas-maps} contains a collection of unrelated lemmas proving measurability for some maps which will be used later on. All of these maps are technical in nature and the same is true for the proofs.
\subsection{Properties of probability measures and functional analysis}
\label{sec:prob}

This section is a conglomeration of facts about measures and their connection to certain spaces of continuous functions. We will need deep measure theoretic and functional analytic results, which we will mostly state without proof. All of the results will culminate in Section~\ref{sec:grp-boundary} in Theorem~\ref{thm:p(x)} which is the first big step towards our boundary map.

\subsubsection*{Probability measures}
\label{sec:prob-measures}

\begin{defin}[Signed and probability measures]
  Let \((B, \Sigma)\) be a measure space. A map \(\mu \colon \Sigma \to \R\) is called a \emph{signed measure} if it is \emph{\(\sigma\)-additive}, i.\,e.
  \[
    \mu\left(\bigcup_{i \in \N}A_i \right) = \sum_{i \in \N} \mu(A_i)
  \]
  for arbitrary \(A_i \in \Sigma\) such that \(A_i \cap A_j = \varnothing\) whenever \(i \neq j\). Here we mean that the right hand side needs to converge.

  We call \(\mu\) a \emph{measure}, if \(\mu(A) \geq 0\) for every \(A \in \Sigma\) and a \emph{probability measure}, if it is a measure and \(\mu(X) = 1\).

  \(A \in \Sigma\) is called \emph{null}, if \(\mu(A) = 0\) and \emph{conull} or \emph{full measure} if its complement is null.
\end{defin}

\begin{rem}
  In this thesis we are mostly interested in probability measures and hence our measures are always finite. Thus we defined signed measures only with image in \(\R\), ignoring infinities. Since we have this restrictions in place, the above definition of (signed) measures in fact satisfies all the standard conditions, in particular \(\mu(\varnothing) = 0\).
\end{rem}

\begin{defin}[Positive, negative and total variation]
    For any \(A \in \Sigma\) and any signed measure \(\mu\) the \emph{total variation} \(|\mu|(A)\) is defined as
  \[
    |\mu|(A) \coloneqq \sup \left\{\sum_{i \in \N} \mu(A_i) \relmid A_i \in \Sigma \text{ and } A = \bigsqcup_{i \in \N} A_i \right \}.
  \]
  The \emph{positive} and \emph{negative variation} are defined as
  \[
    \mu^\pm \coloneqq \frac12 (|\mu| \pm \mu).
  \]
\end{defin}

\begin{prop}[{\cite[Ch.\ 6.1]{RudinFunctional}}]
  The maps \(|\mu|\) and \(\mu^\pm\) are measures on \((X, \Sigma)\) and \(|\mu|(X) < \infty\) holds.
\end{prop}

Often the \(\sigma\)-algebras we are considering stem from a topology on our space, i.\,e.\ are the Borel \(\sigma\)-algebra of its topology. In this case we would also like our measures to be related to the topology of our space. Hence, we need the following definitions:

\begin{defin}[Borel measures and support]
  If \(\Sigma\) is the Borel \(\sigma\)-algebra of a topology on \(X\), then a measure \(\mu\) is called \emph{Borel}, if every \(x \in X\) has an open neighborhood \(U \subset X\) such that \(\mu(U) > 0\).

  The support of a Borel measure \(\mu\) is defined to be the set
  \[
    \supp(\mu) \coloneqq \{x \in X \mid \forall\ U \subset X \text{ open with } x \in U\colon \mu(U) > 0\}.
  \]
\end{defin}

\begin{defin}[Regular measures]
  A Borel measure \(\mu\) is called \emph{inner regular} if
  \[
    \mu(A) = \sup \{ \mu(K) \mid K \subset A \text{ compact}\}
  \]
  and \emph{outer regular} if
  \[\
    mu(A) = \inf \{ \mu(U) \mid U \supset A \text{ open}\}.
  \]
  If it is both it is called \emph{regular}. A signed Borel measure \(\mu\) is called \emph{regular} if \(|\mu|\) is regular.
\end{defin}

Lastly, let us recall the definition of measurable maps and how to pushforward measures along these maps:

\begin{defin}[Measurable maps]
  \label{defin:pushforward}
  Let \((A, \Sigma_A)\) and \((B, \Sigma_B)\) be two measure spaces and \(f\colon A \to B\) a map. \(f\) is called \emph{measurable} if \(f^{-1}(S) \in \Sigma_A\) for every \(S \in \Sigma_B\).

  Let \(\mu\) be a (signed) measure on \(A\) and \(f\) measurable. Then \(f\) is called \emph{essentially constant (with regard to \(\mu\))} if there exists a conull set \(S \in \Sigma_A\) such that \(f|_S\) is constant. The \emph{pushforward of \(\mu\) along \(f\)} is defined via
  \[
    (f_\ast \mu)(S) \coloneqq \mu(f^{-1}(S))
  \]
  for every \(S \in \Sigma_B\).
\end{defin}

\begin{rem}
  Clearly, \(f_\ast\mu\) is again a (signed) measure.
\end{rem}

\begin{lemma}
  \label{lem:mu-supp}
  Let \(X\) be a topological space and \(\mu\) an inner regular measure on \(X\). Then for every measurable set \(A \subset X \setminus \supp(\mu)\) we have that \(\mu(A) = 0\). In particular, if \(\mu\) is non-zero the set \(\supp(\mu)\) is not empty.
\end{lemma}

\begin{proof}
  We fix a measurable \(A \subset X \setminus \supp(\mu)\). Since \(\mu\) is inner regular, we find a sequence of compact sets \(K_n \subset A\) such that \(\mu(K_n) \xrightarrow{n \to \infty} \mu(A)\). Let \(x \in K_n\), then \(x \not \in \supp(\mu)\) and there exists \(U_x \subset X\) open such that \(x \in U_x\) and \(\mu(U_x) = 0\). Since \(K_n\) is compact we find an \(l = l(n) \in \N\) and finitely many \(x_1, \dots, x_l \in K_n\) such that \(K_n \subset \bigcup_{i=1}^l U_{x_i}\). Hence, we have
  \[
    \mu(K_n) \leq \sum_{i=1}^l \mu(U_{x_i}) = 0
  \]
  for every \(n \in \N\). By the convergence, we obtain \(\mu(A) = 0\).

  If \(\supp(\mu)\) were empty, then the set \(X \setminus \supp(\mu)\) would be measurable of full (non-zero) measure, which is a contradiction.
\end{proof}

\subsubsection*{Functional analytic preliminaries}
\label{sec:func-ana}

\begin{defin}
  \label{def:vanishing}
  Let \(X\) be a topological space. The vector space \(C_0(X)\) of \emph{continuous functions vanishing at infinity} is defined via
  \[
    C_0(X) \coloneqq \{f \in C(X) \mid \forall \epsilon > 0\ \exists K \subset X \text{ compact}\colon f|_{X\setminus K} < \epsilon\}.
  \]
\end{defin}

\begin{thm}[Riesz-Markow representation, {\cite[Theorem 6.19]{RudinFunctional}}]
  \label{thm:riesz-markow}
  Let \(X\) be a locally compact Hausdorff space. Every bounded linear functional \(\Phi\) on \(C_0(X)\) is represented by a unique regular signed Borel measure \(\mu\) in the sense that \[
    \Phi f = \int_{\hspace{-5pt}X}\hspace{-5pt}f \d\mu
  \]
  for every element \(f \in C_0(X)\). Moreover, we have \(\|\Phi\| = |\mu|(X)\). In other words, there exists an isometry of normed vector spaces between \(X^\ast\) the (topological) dual of \(X\) equipped with the operator norm \(\|\cdot \|\) and \(M_{s}(X)\) the space of signed measures equipped with total variation \(|\cdot|(X)\) as norm.
\end{thm}

\begin{thm}[Banach-Alaoglu, {\cite{RudinAnalysis}}]
  \label{thm:banach-alaoglu}
  Let \(X\) be a topological vector space and \(V \subset X\) a neighborhood of \(0\). Then
  \[
    K \coloneqq \{ \Phi \in X^\ast \mid |\Phi x| \leq 1 \quad \forall x \in V\},
  \]
  is weak\(\ast\)-compact.
\end{thm}

\begin{cor}
  \label{cor:banach-alaoglu}
  If \(X\) is a compact metric space, then the space of all regular probability measures \(\mathcal{P}(X)\) is weak\(\ast\)-compact and contained in the unit ball of all regular signed measures \(M_{s}(X)\).
\end{cor}

\begin{proof}
  Note that \(C_0(X)\) together with the supremum norm is a Banach space. Let \(V\) be the unit ball in \(C_0(X)\). Theorem~\ref{thm:banach-alaoglu} yields that the unit ball \(B \subset M_s(X) \cong C_0(X)^\ast\) is weak\(\ast\)-compact. For each probability measure \(\mu\), we have \(|\mu|(X) = \mu(X) = 1\). Thus it follows that \(P(X) \subset B\). Now, we only need to show that \(P(X)\) is weak\(\ast\)-closed in \(B\). However,
  \begin{align*}
    P \coloneqq & \left\{\mu \in M_s(X) \relmid \int_X f \d\mu \geq 0 \quad \forall f \geq 0\right\},\\
    N \coloneqq & \left\{\mu \in M_s(X) \relmid \int_X \d\mu = \int_X \chi_X \d\mu = 1\right\},\\
    \mathcal{P}(X)  = & B \cap P \cap N.
  \end{align*}
  The second set ensures that the measure is positive and the last enforces the normalization. These are all the necessary restrictions for a probability measure. Additionally, these sets are clearly weak\(\ast\)-closed.
\end{proof}

We will close the section with a result concerning the separability of the vector space of continuous functions:

\begin{lemma}[{\cite[Theorem~V.6.6]{Conway}}]
  \label{lem:continuous-separable}
  If \(X\) is a compact metric space then the vector space of continuous functions \(C(X)\) equipped with the supremum norm is separable.
\end{lemma}

\subsection{Weighted halfspaces}
\label{sec:weight}

This section defines the main technical tool for our main proof, namely \emph{weighted halfspaces}. These are special subsets of the pocset of halfspaces \(\mathcal{H}\) of a CAT(0) cube complex, which are defined using probability measures on the Roller compactification. In order to understand the following section, it is important to keep the notation introduced in Sections~\ref{sec:halfspaces} and~\ref{sec:rb} in mind.

Unless noted otherwise, \(X\) is a connected, locally countable, finite-dimensional CAT(0) cube complex with pocset of halfspaces \(\mathcal{H}\) and Roller compactification \(\bar X\). Recall the topology on \(\bar X\) introduced in Definition~\ref{def:pot-top} with open sets \(\mathcal{C}(\mathfrak{h})\) for every \(\mathfrak{h} \in \bar X\).

\begin{defin}
  \label{defin:weight}
  Let \(\mu\) be a regular probability measure on \(\bar X\). We define
  \begin{align*}
    H_\mu^{\phantom{+}} \coloneqq &\ \{\mathfrak{h} \in \mathcal{H}(X) \mid \mu(\mathcal{C}(\mathfrak{h})) = \mu(\mathcal{C}(\mathfrak{h}^\ast))\},\\
    H_\mu^+ \coloneqq &\ \{\mathfrak{h} \in \mathcal{H}(X) \mid \mu(\mathcal{C}(\mathfrak{h})) > \text{\nfrac{} 1/2} \},\\
    H_\mu^- \coloneqq &\ \{\mathfrak{h} \in \mathcal{H}(X) \mid \mu(\mathcal{C}(\mathfrak{h})) < \text{\nfrac{} 1/2}\} \text{ and}\\
    H_\mu^\pm \coloneqq &\ \{\mathfrak{h} \in \mathcal{H}(X) \mid \mu(\mathcal{C}(\mathfrak{h})) \neq \text{\nfrac{} 1/2}\}.
  \end{align*} 
  The above four sets are called \emph{balanced, heavy, light} and \emph{unbalanced halfspaces} respectively.
\end{defin}

\begin{lemma}[{\cite[Lemma\ 4.6]{MR3509968}}]
  \label{lem:4.6}
  Let \(\mu,\nu \in \mathcal{P}(\bar X)\).
  \begin{enumerate}
  \item \(H_\mu\) is closed under involution. Also, involution is a bijection between \(H_\mu^+\) and \(H_\mu^-\).
  \item There is the following partition \(\mathcal{H}(X) = H_\mu \sqcup H_\mu^\pm = H_\mu \sqcup H_\mu^+ \sqcup H_\mu^-\).
  \item If \(\mathfrak{h, k} \in H_\mu\) (resp.\ \(H_\mu^+\) or \(H_\mu^-\)), then either \(\mathfrak{h} \pitchfork \mathfrak{k}\) or (after possibly switching switching \(\mathfrak{h}\) and \(\mathfrak{k}\)) the interval \([\mathfrak{h}, \mathfrak{k}]\) lies in \(H_\mu\) (resp.\ \(H_\mu^+\) or \(H_\mu^-\)).
  % \item There are no facing triples\todo{facing triples} of halfspaces in \(H_\mu\). If \(X\) is not Euclidean\todo{Euclidean}, then \(H_\mu^+ \neq \varnothing\).\todo{check if I need this and if not remove it}
  \item If \(H_\mu\) and \(H_\nu\) are both non-empty and \(H_\mu \cap H_\nu = \varnothing\), then \(H_\mu \cap H_\nu^+ \neq \varnothing\) and \(H_\mu \cap H_\nu^- \neq \varnothing\).
  \item If \(\mathfrak{h, k}  \in H_\mu\) are two parallel halfspaces with \(\mathfrak{h} \subset \mathfrak{k}\), then \(\mu(\mathcal{C}(\mathfrak{h}^\ast) \cap \mathcal{C}(\mathfrak{k})) = 0\).
  \item The assignments \(\mu \mapsto H_\mu\), \(\mu \mapsto H_\mu^+\) and \(\mu \mapsto H_\mu^-\) are \(\Aut(X)\)-equivariant for the natural actions on \(\mathcal{P}(\bar X)\) and \(\operatorname{Pot}(\mathcal{H}(X))\).
  \end{enumerate}
\end{lemma}

\begin{proof}
  1.\ -- 3.\ are clear from the definitions and the additivity of the measure.

  For 4.\ we see that \(H_\mu \subset H_\nu^+ \cap H_\nu^-\). Since \(H_\mu\) is invariant under involution, but \(H_\nu^+\) and \(H_\nu^-\) get interchanged, we see that both intersections must be non-empty.

  For 5.\ we have
  \begin{align*}
    \frac{1}{2} = \mu(\mathcal{C}(\mathfrak{k}))
    & = \mu(\mathcal{C}(\mathfrak{k}) \cap \mathcal{C}(\mathfrak{h^\ast})) + \mu(\mathcal{C}(\mathfrak{k}) \cap \mathcal{C}(\mathfrak{h}))\\
    & = \mu(\mathcal{C}(\mathfrak{k}) \cap \mathcal{C}(\mathfrak{h^\ast})) + \mu(\mathcal{C}(\mathfrak{h}))\\
    & = \mu(\mathcal{C}(\mathfrak{k}) \cap \mathcal{C}(\mathfrak{h^\ast})) + \frac{1}{2},
  \end{align*}
  where we have \(\mathcal{C}(\mathfrak{h}) \subset \mathcal{C}(\mathfrak{k})\) because we have \(\mathfrak{h} \subset \mathfrak{k}\) and ultrafilters satisfy the consistency condition. Hence, \(\mu(\mathcal{C}(\mathfrak{h^\ast}) \cap \mathcal{C}(\mathfrak{k})) = 0\). Assertion 6 follows again easily from the definitions.
\end{proof}

\begin{lemma}[{\cite[Lemma 4.7]{MR3509968}}]
  \label{lem:interval}
  The complex \(\bar X(H_\mu)\) is an interval (in the sense of Definition~\ref{defin:uf-interval}).
\end{lemma}

\begin{proof}
  Let
  \begin{align*}
    p \colon \bar X &\to \bar X (H_\mu),\\
    \alpha &\mapsto \alpha \cap H_\mu
  \end{align*}
  be the projection. Since \(p\) is continuous, we can have \(p_\ast\mu\). Next, choose \(\alpha \in \supp(p_\ast \mu)\), i.\,e.\ every open neighborhood of \(\alpha\) must have non-zero measure. The set \(\supp(p_\ast \mu)\) is not empty by Lemma~\ref{lem:mu-supp}. We claim that \(\alpha^\ast\) is also an ultrafilter. It automatically satisfies the choice condition, so we only need to check consistency. Let \(\mathfrak{h} \in \alpha^\ast\) and \(\mathfrak{k} \in H_\mu\) such that \(\mathfrak{h} \subset \mathfrak{k}\). Assume that \(\mathfrak{k}\) is not in \(\alpha^\ast\). Then \(\mathcal{C}(\mathfrak{h^\ast}) \cap \mathcal{C}(\mathfrak{k})\) is an open neighborhood of \(\alpha\) in \(\bar X(H_\mu)\) and its measure positive. However, by Lemma~\ref{lem:4.6}(5) we know that the measure is zero. Hence, we have \(\mathfrak{k} \in \alpha^\ast\) and we can conclude the proof thanks to Lemma~\ref{lem:x-interval}.
\end{proof}

\begin{lemma}
  \label{lem:finite-terminal}
  For any \(H \subset H_\mu\), we have that \(|\tau(H)| < \infty\) (see Definition~\ref{defin:tau}).
\end{lemma}

\begin{proof}
  By Lemma~\ref{lem:interval} and Theorem~\ref{thm:interval}, we know that \(\bar X(H_\mu)\) is an interval and is embeddable in some \(\R^d\). The observation in Example~\ref{bsp:finite-terminal} shows that any subset \(H \subset H_\mu \subset \mathcal{H}(\R^d)\) has at most \(2d\) terminal elements.
\end{proof}

\begin{lemma}
  \label{lem:proj-weight}
  Let \(X\) be a finite-dimensional CAT(0) cube complex with pocset of halfspaces \(\mathcal{H}\) and \(Y\) one of its irreducible factors with pocset of halfspaces \(\mathcal{H}'\). Then the projection
  \begin{align*}
    p\colon \bar X & \to \bar Y,\\
    \alpha & \mapsto \alpha \cap \mathcal{H}'
  \end{align*}
  is continuous. If \(\mu\) is a regular probability measure on \(\bar X\) we have
  \[
    H_{p_\ast \mu} = H_\mu \cap \mathcal{H}'
  \]
\end{lemma}

\begin{proof}
  Let \(F, G \subset \mathcal{H}'\) be finite. Then
  \begin{align*}
    p^{-1}(\mathcal{C_{\mathcal{H}'}}(F,G)) = \mathcal{C_{\mathcal{H}}}(F,G) \subset \mathcal{H}
  \end{align*}
  and \(p\) is continuous. Hence, the pushforward of the measure is well-defined and for every \(\mathfrak{h} \in \mathcal{H}'\) we have
  \[
    (p_\ast \mu))(\mathcal{C}_{\mathcal{H}'}(\mathfrak{h})) = \mu(p^{-1}(\mathcal{C}_{\mathcal{H}'}(\mathfrak{h}))) = \mu(\mathcal{C}_{\mathcal{H}}(\mathfrak{h}))
  \]
  proving the last equality.
\end{proof}

\subsection{Measurability of certain maps}
\label{sec:meas-maps}

In the main proof of our theorem ergodicity will play a central role (see Section~\ref{sec:ergodic}). We will mostly use it in the form that measurable \(\Gamma\)-invariant maps have to be essentially constant. Hence, we need all our important maps to be measurable. These proofs are mainly technical and have been collected in this section. The results are mostly stand-alone and there is no deeper connection between them.

Unless noted otherwise, \(X\) is a locally countable, finite-dimensional CAT(0) cube complex with \(\mathcal{H} \coloneqq \mathcal{H}(X)\) its pocset of halfspaces and \(\bar X\) its Roller compactification.

\begin{lemma}
  \label{lem:measurable-countable}
  Let \(N\) be a countable set. Then
  \begin{align*}
    f\colon \operatorname{Pot}(N) &\to \N \cup \{\infty\},\\
    A &\mapsto |A|
  \end{align*}
  is measurable, where \(\N \cup \{\infty\}\) is equipped with the discrete topology.
\end{lemma}

\begin{proof}
  We will see that a basis of the topology is mapped to measurable sets via the preimage. First, let us consider \(n \in \N\). Then we have
  \begin{align*}
    f^{-1}(\{n\})
    & = \{ A \subset N \mid |A| = n\}\\
    & = \bigcup_{\substack{A \subset N\\|A| = n}} \left ( \bigcap_{\substack{F \subset N\\|F|< \infty}}\mathcal{C}(A,F)\right),
  \end{align*}
  where \(\mathcal{C}(A,F)\) is a cylinder set as defined in Definition~\ref{def:pot-top}. Since the set of all finite subsets of \(N\) is countable, the above preimage is measurable as it is a countable union and intersection of measurable sets.

  Lastly, we consider \(f^{-1}(\{\infty\})\). However, here we have
  \[
    f^{-1}(\{\infty\}) = N \setminus \left (\bigcup_{n=0}^\infty f^{-1}(\{n\})\right)
  \]
  which is measurable as the complement of a measurable set.
\end{proof}

\begin{lemma}
  \label{lem:tau}
   Let \(\tau\colon \operatorname{Pot}(\mathcal{H}) \to \operatorname{Pot}(\mathcal{H})\) be the map assigning to each subset of \(\mathcal{H}\) its set of terminal elements (c.\,f.~Definition~\ref{defin:tau}). Then \(\tau\) is measurable.
\end{lemma}

\begin{proof}
  We take an arbitrary cylinder set \(\mathcal{C}(F_1, F_2)\) and are interested in the preimage
  \begin{align*}
    \tau^{-1}(\mathcal{C}(F_1, F_2))
    & = \{H \subset \mathcal{H} \mid F_1 \subset \tau(H) \text{ and } \forall \mathfrak{h} \in F_2\colon \mathfrak{h} \in H\ \Rightarrow\ \mathfrak{h} \not\in \tau{H}\} \\
    & = \{H \subset \mathcal{H} \mid F_1 \subset \tau(H)\} \cap \{H \subset \mathcal{H} \mid \forall \mathfrak{h} \in F_2\colon \mathfrak{h} \in H\ \Rightarrow\ \mathfrak{h} \not\in \tau(H)\}\\
    & \eqqcolon T \cap N.
  \end{align*}
  We now decompose \(T\) as follows:
  \begin{align*}
    T
    & = \bigcap_{\mathfrak{h} \in F_1}\{H \subset \mathcal{H} \mid \mathfrak{h} \in \tau(H)\}\\
    & = \bigcap_{\mathfrak{h} \in F_1}\left (\{H \subset \mathcal{H} \mid \mathfrak{h} \in H \text{ minimal}\} \cup \{H \subset \mathcal{H} \mid \mathfrak{h} \in H \text{ maximal}\}\right )\\
    & = \bigcap_{\mathfrak{h} \in F_1}\left ( \bigcap_{\substack{\mathfrak{k} \in \mathcal{H}\\\mathfrak{k} \subset \mathfrak{h}}}\mathcal{C}(\{\mathfrak{h}\}, \{\mathfrak{k}\}) \cup \bigcap_{\substack{\mathfrak{k} \in \mathcal{H}\\\mathfrak{h} \subset\mathfrak{k}}} \mathcal{C}(\{\mathfrak{h}\}, \{\mathfrak{k}\})\right ).
  \end{align*}
  This set is measurable as it is a countable intersection and union of measurable sets. Next let us decompose \(N\):
  \begin{align*}
    N
    & = \bigcup_{F \subset F_2} \{H \subset \mathcal{H} \mid F \subset H \setminus \tau(H) \text{ and } (F_2 \setminus F) \cap H = \varnothing\}\\
    & = \bigcup_{F \subset F_2} \left ( \{H \subset \mathcal{H} \mid F \subset H \setminus \tau(H)\} \cap \mathcal{C}(\varnothing, F_2 \setminus F)\right).\\
  \end{align*}
  We will now show that the set where a finite subset of elements are not terminal is measurable. This will conclude the proof. This can be achieved via an induction over \(n = |F|\). The case \(n = 0\) is clear, since any \(\sigma\)-algebra needs to contain the whole set. Assume the assertion is true for every finite subset \(F \subset \mathcal{H}\) of \(n\) elements. If \(F\) contains \(n+1\) elements, fixing \(\mathfrak{h} \in F\) and \(\tilde F \coloneqq F \setminus \{\mathfrak{h}\}\), we do the following decomposition:
  \begin{align*}
    \{H \subset \mathcal{H} \mid F \subset \mathcal{H} \setminus \tau(H)\}
    & = \{H \subset \mathcal{H} \mid \{\mathfrak{h}\} \cup \tilde F \subset H \setminus \tau(H)\}\\
    & = \{H \subset \mathcal{H} \mid \tilde{F} \subset H \setminus \tau(H)\} \setminus \{H \subset \mathcal{H} \mid \mathfrak{h} \in \tau(H)\}.
  \end{align*}
  The first set is measurable by induction hypothesis, the second one is measurable by our computations above. All in all we obtain that \(\tau^{-1}(\mathcal{C}(F_1, F_2))\) is measurable.
\end{proof}


\begin{lemma}[{\cites[Lemma~ A.1]{MR3509968}}]
  \label{lem:measurable-mu}
  Let \(I \subset [0,1]\) be a subinterval that is either open, closed or half open. Let \(H^I_\mu \coloneqq \{h \in \mathcal{H}(X) \mid \mu(\mathcal{C}(h)) \in I\}\). Then the map
  \begin{align*}
    \mathcal{P}(\bar X) &\to \operatorname{Pot}(\mathcal{H}(X)),\\
    \mu &\mapsto H^I_\mu
  \end{align*}
  is measurable with respect to the weak\(\ast\)-topology on \(\mathcal{P}(X)\).
\end{lemma}

\begin{proof}
  In Section~\ref{sec:roller} we defined a topology on the power set (c.\,f.\ Definition~\ref{def:pot-top} and Proposition~\ref{prop:pot-top}). Hence, we need to show that the preimages of the basic open sets are mapped to measurable sets in \(\mathcal{P}(\bar X)\). So let \(F_1, F_2 \subset \mathcal{H}(X)\) be finite and consider the cylinder set \(\mathcal{C}(F_1, F_2)\). Then the preimage is given by the set
  \[
    K(F_1, F_2) = \{\mu \in \mathcal{P}(\bar X) \mid H^I_\mu \in \mathcal{C}(F_1, F_2)\}.
  \]
  For now, let us consider the sets \(E_I(\mathfrak{h}) \coloneqq \{\mu \in \mathcal{P}(\bar X) \mid \mu(\mathcal{U}(\mathfrak{h})) \in I\}\). We will now show that these sets are measurable. We know that \(\bar X = \mathcal{U}(\mathfrak{h}) \sqcup \mathcal{U}(\mathfrak{h}^\ast)\). Hence, \(\mathfrak{\tilde h} \coloneqq \mathcal{U}(\mathfrak{h})\) is both open and closed in \(\bar X\). Thus the indicator function \(\chi_{\mathfrak{\tilde h}}\) is continuous. However, the weak\(\ast\)-topology is defined such that each map
  \begin{align*}
    T_f\colon \mathcal{P}(\bar X) &\to \R,\\
    \mu &\mapsto \int_X f \d\mu
  \end{align*}
  is continuous (for each \(f \in C(\bar X)\)). Hence, \(T \coloneqq T_{\chi_{\mathfrak{\tilde h}}}\) is continuous and thus also measurable. Furthermore, we have that \(T^{-1}(I) = E_I(\mathfrak{h})\). The interval \(I\) is measurable, so the same is true for \(E_I(\mathfrak{h})\).

  Together with the following observation this finishes the proof:
  \[
    K(F_1, F_2) = \left (\bigcap_{\mathfrak{h} \in F_1} E_I(\mathfrak{h}) \right ) \cap \left ( \bigcap_{\mathfrak{h} \in F_2} E_I(\mathfrak{h})^{c}\right).
  \]
\end{proof}


\begin{lemma}
  \label{lem:measurable-interval}
  Let \(\alpha, \beta \in \bar X\). Let \(\mathcal{H}(\alpha, \beta)\) be the set of halfspaces separating \(\alpha\) from \(\beta\) (c.\,f.\ Definition~\ref{defin:separating}). Then the map
  \begin{align*}
    f\colon \bar X \times \bar X &\to \operatorname{Pot}(\mathcal{H}),\\
    (\alpha, \beta) &\mapsto \mathcal{H}(\alpha, \beta)
  \end{align*}
  is measurable.
\end{lemma}

\begin{proof}
  We will see that the map is even continuous. Take finite subsets \(F,G \subset \mathcal{H}\) and consider the following preimage
  \begin{align*}
    f^{-1}(\mathcal{C}(F, G))  & = \{(\alpha, \beta) \in \bar X \times \bar X \mid F \subset \mathcal{H}(\alpha, \beta) \} \cap \{(\alpha, \beta) \in \bar X \times \bar X  \mid G \cap \mathcal{H}(\alpha, \beta) = \varnothing\}\\
                                 &\eqqcolon S \cap N.
  \end{align*}
  We will consider \(S\) and \(N\) separately:
  \begin{align*}
    S & = \bigcup_{F' \subset F} \left( \mathcal{C}(F' \cup (F \setminus F')^\ast, \varnothing) \times \mathcal{C}(F'^\ast  \cup F \setminus F', \varnothing) \right)
  \end{align*}
  and
  \begin{align*}
    N & = \bigcap_{\mathfrak{h} \in G}\left[(\mathcal{C}(\mathfrak{h}) \times \mathcal{C}(\mathfrak{h})) \cup (\mathcal{C}(\mathfrak{h}^\ast) \times \mathcal{C}(\mathfrak{h}^\ast))\right].
  \end{align*}
  Then \(f^{-1}(\mathcal{C}(F,G))\) is open as a finite intersection of open sets.
\end{proof}

\begin{lemma}
  \label{lem:measurable-str-sep}
  The map
  \begin{align*}
    \mathcal{P}(\bar X) & \to \N \cup \{\infty\},\\
    \mu & \mapsto |(H_\mu\times H_\mu) \cap \mathcal{S}|,
  \end{align*}
  where
  \[
    \mathcal{S}\coloneqq \{(\mathfrak{h},\mathfrak{k}) \in \mathcal{H} \times \mathcal{H} \mid \mathfrak{h} \text{ and } \mathfrak{k} \text{ are strongly separated}\}
  \]
  is measurable.
\end{lemma}


\begin{proof}
  We decompose the map as follows
  \begin{alignat*}{4}
    \mathcal{P}(\bar X) &\xrightarrow{f} \mathcal{P}(\bar X)^2&&\xrightarrow{g} \operatorname{Pot}(\mathcal{H}^2) &&\xrightarrow{h} \operatorname{Pot}(\mathcal{H}^2) &&\xrightarrow{j} \N \cup \{\infty\},\\
    \mu &\mapsto (\mu, \mu) &&\mapsto (H_\mu, H_\mu)&&\mapsto H_\mu \times H_\mu \cap \mathcal{S} &&\mapsto |H_\mu \times H_\mu \cap \mathcal{S}|.
  \end{alignat*}
  \(f\) is continuous and hence measurable. \(g\) is measurable, because the map on each factor is measurable by Lemma~\ref{lem:measurable-mu}. \(j\) is measurable by Lemma~\ref{lem:measurable-countable}. We are left with \(h\). Consider the two finite subsets \(F, G \subset \mathcal{H}^2\). Then the preimage of the associated cylinder set is given by
  \begin{align*}
    h^{-1}(\mathcal{C}(F,G)) & \coloneqq \{(H, K) \in \operatorname{Pot}(\mathcal{H})^2 \mid F \subset H \times K \wedge G \subset (H \times K)^c\\
    & \wedge (\mathfrak{h}, \mathfrak{k}) \in H \times K \text{ is strongly separated}\}\\
      & = \left (\bigcap_{(\mathfrak{h},\mathfrak{k}) \in F} \mathcal{C}(\mathfrak{h}) \times \mathcal{C}(\mathfrak{k})\right)\\
      & \cap \left (\bigcap_{(\mathfrak{h},\mathfrak{k}) \in G} \mathcal{C}(\mathfrak{h})^c \times \operatorname{Pot}(\mathcal{H}) \cup \operatorname{Pot}(\mathcal{H}) \times \mathcal{C}(\mathfrak{k})^c \right)\\
      & \cap \left( \bigcap_{\substack{(\mathfrak{h}, \mathfrak{k}) \in \mathcal{H}^2\\\text{not str.\ sep.}}} \mathcal{C}(\mathfrak{h}) \times \mathcal{C}(\mathfrak{k})\right)^c,
  \end{align*}
  which is a countable union of measurable sets. Hence our map is measurable as a composition of measurable maps.
\end{proof}

\begin{lemma}
  Let \(A\) be a set. The maps
  \begin{align*}
    f \colon \operatorname{Pot}(A) \times \operatorname{Pot}(A) & \to \operatorname{Pot}(A)\\
    (H, K) & \mapsto H \cap K
  \end{align*}
  and
  \begin{align*}
    g \colon \operatorname{Pot}(A) \times \operatorname{Pot}(A) & \to \operatorname{Pot}(A)\\
    (H, K) & \mapsto H \cup K
  \end{align*}
  are measurable with regard to the Borel-\(\sigma\)-algebra of the cylinder topology.
\end{lemma}

\begin{proof}
  \label{lem:measurable-intersection-union}
  Let \(F, G \subset A\) be two finite subsets. Then we have
  \begin{align*}
    f^{-1}(\mathcal{C}(F,G))
    & = \{(H,K) \in \operatorname{Pot}(A)^2 \mid F \subset H \cap K \wedge G \subset H^c \cup K^c\} \\
    & = \bigcap_{f \in F} \mathcal{C}(f) \times \mathcal{C}(f)\\
    & \cap \bigcap_{g \in G} (\mathcal{C}(g)^c \times \operatorname{Pot}(A) \cup \operatorname{Pot}(A) \times \mathcal{C}(g)^c)
  \end{align*}
  and
  \begin{align*}
    g^{-1}(\mathcal{C}(F,G))
    & = \{(H,K) \in \operatorname{Pot}(A)^2 \mid F \subset H \cup K \wedge G \subset H^c \cap K^c\} \\
    & = \bigcap_{f \in F} (\mathcal{C}(f) \times \operatorname{Pot}(A) \cup \operatorname{Pot}(A) \times\mathcal{C}(f))\\
    & \cap \bigcap_{g \in G} \mathcal{C}(g)^c \times \mathcal{C}(g)^c.
  \end{align*}
  Hence both maps are measurable.
\end{proof}

\begin{lemma}
  \label{lem:measurable-tau-int}
  The map
  \begin{align*}
    \mathcal{P}(\bar X) \times \mathcal{P}(\bar X) &\to \operatorname{Pot}_f(\mathcal{H}),\\
    (\mu,\nu) &\mapsto \tau([H_\mu^+ \cap H_\nu] \cup [H_\nu^+ \cap H_\mu])
  \end{align*}
  is measurable.
\end{lemma}

\begin{proof}
  We know that the diagonal embedding and the product of measurable functions is again measurable. Hence the only interesting thing to see is that the intersection and union preserve measurability. However, this has been proven in Lemma~\ref{lem:measurable-intersection-union}. Together with Lemma~\ref{lem:tau} this proves the assertion.
\end{proof}

%%% Local Variables:
%%% mode: latex
%%% TeX-master: "../Master"
%%% ispell-local-dictionary: "en_US"
%%% End:
