\section{The essential core}
\label{sec:essential}

\begin{defin}[The essential core]
  A halfspace \(h \in \mathcal{H}\) is calle\emph{\(\Gamma\)-essential}, if for some \(x \in X\) the orbit in \(h\), \(\Gamma x \cap h\), is \emph{not} a bounded distance away from \(\hat h\). A hyperplane \(\hat h \in \mathcal{\hat H}\) is called \emph{\(\Gamma\)-essential}, if both its halfspaces are \(\Gamma\)-essential and \emph{half-essential}, if onlye one of its halfspaces is \(\Gamma\)-essential.

  \(\Ess(X,\Gamma)\subset \mathcal{\hat H}(X)\) denotes the set of all essential hyperplanes. Accordingly, \(\nEss(X,\Gamma) \subset \mathcal{\hat H}(X)\) dentos the set of all non-essential hyperplanes, leading to
  \[
    \mathcal{\hat H}(X) = \Ess(X, \Gamma) \sqcup \nEss(X, \Gamma).
  \]
\end{defin}

\begin{prop}
  \(\Ess(X, \Gamma)\) and \(\nEss(X, \Gamma)\) are \(\Gamma\)-invariant.
\end{prop}

\begin{defin}
  The \emph{essential core} is the complex corresponding to the the hyperplanes \(\Ess(X, \Gamma)\). It can be isometrically embedded into \(X\).
\end{defin}

\todo{Cite CS11, Prop3.5 that the essential core is not empty if visual boundary yadda yadda}

\begin{rem}
  Even if \(X\) is irreducible the same might not be true for the essential core \(Y\). \todo{find counterexample for reducible essential core}
\end{rem}

%%% Local Variables:
%%% mode: latex
%%% TeX-master: "../Master"
%%% End:
