\section{CAT(0) cube complexes}
\label{sec:complexes}

\subsection{Metric preliminaries}
\label{sec:metric}
This section is concerned with basic metric properties of spaces and their connection to CAT(\(0\)) spaces. The exposition follows closely~\cite{MR1744486}.

\begin{defin}~
  \begin{itemize}
  % \item Let \(X\neq \varnothing\) be a set. \(d\colon X \times X \to \R_{\geq 0} (\cup \{\infty\})\) is called a \emph{metric}, if for all \(x,y,z \in X\)
  %   \begin{enumerate}
  %   \item \(d(x,y) = 0\) if and only if \(x = y\),
  %   \item \(d(x,y) = d(y,x)\) and
  %   \item \(d(x,y) + d(y,z) \geq d(x,z)\).
  %   \end{enumerate}
  %   The pair \((X,d)\) is called a \emph{metric space}.
  \item Let \(X, d\) be a metric space. A \emph{geodesic} from \(x\) to \(y\) (\(x,y \in X\)) is a map \(c\colon [a,b] \to X\), such that \(c(a) = x\), \(c(b) = y\) and such that there exists a constant \(\lambda > 0\) with
    \begin{align*}
      d(c(t), c(t')) = \lambda \cdot |t - t'| 
    \end{align*}
    for all \(t,t' \in [a,b]\).
  \item \((X,d)\) is called \emph{\(r\)-geodesic} (\(r > 0\)), if \(d(x,y) < r\) implies that there is a geodesic joining \(x\) and \(y\). It is called \emph{uniquely \(r\)-geodesic}, if this geodesic is unique. \((X,d)\) is called \emph{(uniquely) geodesic}, if it is (uniquely) \(r\)-geodesic for all \(r > 0\).
  % \item The \emph{length} of a curve \(c\colon [a,b] \to X\) is defined as
  %   \begin{align*}
  %     l(c) \coloneqq \left\{\sum_{i=1}^n d(c(t_i), c(t_{i-1})) \relmid a = t_0 \leq t_0 \leq \dots \leq t_n = b \text{ a subdivision} \right\}.
  %   \end{align*}
  %   \(c\) is called \emph{rectifiable}, if \(l(c) < \infty\).
  % \item \(d\) is called a \emph{length metric}, if
  %   \begin{align*}
  %     d(x,y) = \inf \{l(c) \mid c \text{ rectifiable joining } x \text{ and } y\}
  %   \end{align*}
  %   for arbitrary \(x,y \in X\), where \(\inf\varnothing = \infty\). In this case \((X,d)\) is called a \emph{length space}.
  \end{itemize}
\end{defin}

\begin{rem}
  In this thesis the metric is allowed to take the value \(\infty\). This is a bit unusual, but convenient in the setting of length spaces, where it might be that a path does not exist.
\end{rem}

\begin{bsp}
  \todo{fill in some examples}
  \begin{enumerate}
  \item \((\E^n, d_0)\), where \(\E^n = \R^n\) and \(d_0(x,y) \coloneqq \|x - y\|_2\), is a length spaces which is uniquely geodesic. However, \(\E^n \setminus {0}\) is still a length space, but no longer geodesic. If a whole cube \(C= [0,1]^n\) is removed, \(\E^n \setminus C\) is no longer a length space, with regard to the induced metric. 
  % \item \((\sphere^n, d_\kappa)\), \(\kappa > 0\), is the unit sphere in \(\E^{n+1}\) together with the metric \(d_{\kappa}(x,y) = \frac{1}{\sqrt{\kappa}} \cdot \angle (x,y)\). This space is uniquely \(\frac{\pi}{\sqrt{\kappa}}\)-geodesic.
  % \item \((\hyperbole^n, d_\kappa)\), \(\kappa < 0\), is the hyperbolic \(n\)-space
  %   \begin{align*}
  %     \hyperbole^n & = \left\{x = (x_0, \dots, x_n)\in \R^{n+1} \relmid \langle x, x \rangle_{1,n} = -x_0^2 + \sum_{i=1}^n x_i^2 = -1 \right\} \quad \text{with}\\
  %     d_\kappa(x,y) & = \frac{1}{\sqrt{-\kappa}} \cdot \operatorname{arcosh} \langle x,y \rangle_{1,n}.
  %   \end{align*}
  %   This space is also a length space and uniquely geodesic.
  \end{enumerate}
\end{bsp}

% \begin{defin}
%   Let
%   \begin{align*}
%     M_\kappa^n \coloneqq
%     \begin{cases}
%       \sphere^n & \kappa > 0\\
%       \E^n & \kappa = 0\\
%       \hyperbole^n & \kappa < 0
%     \end{cases} \text{ and }
%     D_\kappa =
%     \begin{cases}
%       \frac{\pi}{\sqrt{\kappa}} & \kappa > 0\\
%       \infty & \kappa \leq 0
%     \end{cases}.
%   \end{align*}
%   Then \((M_\kappa^n, d_\kappa)\) is called the \emph{model space of curvature \(\kappa\)}. \(D_\kappa\) denotes its \emph{diameter}.
% \end{defin}

% \begin{rem}
%   As a reference for the above implicit claims~\cite[Sec.\ I.2, I.6]{MR1744486} can be used.
% \end{rem}

\begin{defin}[CAT(0) and non-positive curvature spaces]
  \label{def:cat}
  \begin{itemize}
  \item Let \((X,d)\) be a metric space. A \emph{geodesic triangle} \(\Delta \subset X\) consists of three points \(p,q,r \in X\), its \emph{vertices}, together with a choice of three geodesic segments \([p,q], [q,r], [r, p]\) joining them, its \emph{edges} (The choice of geodesics might not be unique). If necessary, the notation \(\Delta = \Delta(p,q,r)\) or \(\Delta = \Delta([p,q], [q,r], [r,p])\) will be used. The first case is a slight abuse of notation, as the three vertices might not determine the triangle.
  \item A \emph{comparison triangle} in \(\E^2\) for \(\Delta = \Delta(p,q,r)\) is a choice of three points \(\bar p, \bar q, \bar r \in \E^2\) such that \(d_\kappa(\bar p, \bar q) = d(p, q)\), \(d_\kappa(\bar q, \bar r), = d(q, r)\) and \(\|\bar r- \bar p\| = d(r, p)\). It will be denoted by \(\bar \Delta = \Delta(\bar p, \bar q, \bar r)\). Such a comparison triangle always exists \cite[cf.][Sec.\ I.2]{MR1744486}.
  \item \(\Delta\) satisfies the \emph{CAT(\(0\)) inequality}, if \(d(x,y) \leq \|\bar x, \bar y\|\) for any \(x, y \in \Delta\) and where \(\bar x \in \bar \Delta \subset \E^2\) is the unique point having the same distance to the two endpoints of the geodesic segment it lies on as \(x\) on its geodesic.
  \item \(X\) is called a \emph{CAT(\(0\)) space}, if \(X\) is geodesic and if each geodesic triangle \(\Delta\) satisfies the CAT(\(0\)) inequality.
  \item \(X\) is called \emph{of curvature \(\leq 0\)} or \emph{non-positively curved}, if it is locally a CAT(\(0\)) space, i.\,e.\ for each \(x \in X\) there exists an \(r > 0\), such that B(x,r) together with the induced metric is a CAT(\(0\)) space.
  \end{itemize}
\end{defin}

\begin{bsp}
  \todo{Elaborate CAT examples}
  \begin{enumerate}
 % \item CAT(\(\kappa\)) contained etc
  \item \(\E^n\) without a diamond. Still locally CAT(0), but not globally
  % \item \(\bigvee_{n \in \N} (\sphere_n, x_n)\), where \(\sphere_n = \frac{1}{\sqrt{n}} \sphere^1
    % \cong M_n^1\) and \(x_n \in \sphere_n\). Not CAT(\(\kappa\)) for any \(\kappa\), not even locally.
  \end{enumerate}
\end{bsp}

\begin{defin}[Cubes]
  A subset \(C = [0,1]^n \subset \E^n\) is called a \emph{cube}. Let \(H \subset M_\kappa\E^n\)  be a hyperplane, containing \(C\) in one of its half-spaces. If \(F \coloneqq H \cap C \neq \varnothing\), then \(F\) is called a \emph{face} of \(C\), it is called \emph{proper}, if \(F \neq C\). The notation \(F \preceq C\) will be applied for faces. The \emph{dimension} of \(F\) is the dimension of the smallest \(m\)-plane containing \(F\). The \emph{interior} of \(F\), \(\mathring F\), is the interior of \(F\) considered as a subspace of the \(m\)-plane equipped with its \(\E^m\)-structure. Any subset \(C \cap \{x_i = \text{\nfrac 1/2}\}\) is called a \emph{midcube of \(C\)}. The \emph{\(m\)-skeleton of \(C\)} is defined by
  \begin{align*}
    C^{(m)} \coloneqq \bigcup \{F \mid F \preceq C \text{ and } \dim F \leq m\}.
  \end{align*}

  Let \(x \in C\), then the \emph{support of \(x\)}, \(\supp(x)\), is the unique face of \(C\) containing \(x\) in its interior or alternatively the unique face with minimal dimension containing \(x\).

  The \emph{link of \(x\) in \(C\)} is given by
  \begin{align*}
    \lk(x,C) \coloneqq \{u \in U_x\E^n \mid \exists t > 0 \colon \exp_x(tu) \in C\} \subset U_x\E^n \cong \sphere^{n-1},
  \end{align*}
  where \(U_x\E^n\) is the unit tangent space at \(x\) in \(\E^n\) considered as a Riemannian manifold. Because of this Riemannian metric the unit tangent space can be isometrically identified with \(\sphere^{n-1}\).
\end{defin}

\begin{bsp}
  \todo{Add visualization for links, stars and the like}
\end{bsp}

\begin{rem}
  \(\lk(v,C) \subset \sphere^{n-1}\) is a simplex for all vertices \(v \in C\). Furthermore, in this case all its edges have length \(\frac{\pi}{2}\).
\end{rem}

\begin{defin}[Cube complexes]
  \begin{itemize}
  \item Let \((C_\lambda)_{\lambda \in \Lambda}\) be a family of cubes and \(\mathcal{C} \coloneqq \bigcup_{\lambda \in \Lambda} C_\lambda \times \{\lambda\}\) its disjoint union. Furthermore, let \(\sim\) denote an equivalence relation on \(\mathcal{C}\) and by \(X\) the space of equivalence classes with natural projection \(p \colon \mathcal{C} \to X\). Lastly, let \(p_\lambda \colon C_\lambda \to X,\ p_\lambda(x) \coloneqq p(x, \lambda)\).

    \(X\) is called a \emph{cube complex}, if
    \begin{enumerate}
    \item \(p_\lambda\) is injective and
    \item for arbitrary \(\lambda_1, \lambda_2 \in \Lambda\) and \(x_i \in C_{\lambda_i}\) such that \(p_{\lambda_1}(x_1) = p_{\lambda_2}(x_2)\), there exists an isometry \(h\colon \supp(x_1) \to \supp(x_2)\), such that \(p_{\lambda_1}|_{\supp(x_1)} = p_{\lambda_2} \circ h\).
    \end{enumerate}
  \item \(C \subset X\) is called an \emph{\(n\)-dimensional cube}, if it is the image of an \(n\)-dimensional face \(F \leq C_\lambda\) under \(p_\lambda\). The interior of \(C\) is given by \(\mathring C \coloneqq p_\lambda(\mathring F)\). A \emph{midcube of \(C\)} is the image under \(p_\lambda\) of a midcube of \(F\).
  \item The \emph{m-skeleton} of \(X\) is given by
    \begin{align*}
      X^{(m)} \coloneqq \quot{\bigcup_{\lambda \in \Lambda} C_\lambda^{(m)} \times \{\lambda\}}{\sim},
    \end{align*}
    where \(\sim\) is given by the restriction of the equivalence relation on \(\mathcal{C}\) to the disjoint union of the \(m\)-skeleta of the cells.

  \item Fix \(x \in X\). Then the \emph{star of \(x\)} is defined by
    \begin{align*}
      \st(x) \coloneqq \bigcup \left\{\mathring C \relmid C \subset X \text{ a cube and } x \in C\right\}.
    \end{align*}
  \item Let \((x_i)_{i \in I}\) be a family of all the points with \(x_i \in C_{\lambda(i)}\), such that \(p_{\lambda(i)}(x_i) = x\). Consider the disjoint union \(\bigcup_{i \in I} \lk(x_i, C_{\lambda(i)} \times {i})\) together with the equivalence relation \((u_i, i) \sim (u_j, j)\) if there exist \(t_i, t_j > 0\) such that \(\exp_{x_i}(t_i u_i) \in C_{\lambda(i)}\), \(\exp_{x_j}(t_j u_j) \in C_{\lambda(j)}\) and \(p_{\lambda(i)}(\exp_{x_i}(t_i u_i)) = p_{\lambda(j)}(\exp_{x_j}(t_j u_j))\). Then the \emph{link of \(x\) in \(K\)} is given by
    \begin{align*}
      \lk(x, X) \coloneqq \quot{\bigcup_{i \in I} \lk(x_{\lambda(i)}, C_{\lambda(i)})}{\sim}.
    \end{align*}
  \end{itemize}
\end{defin}

\begin{rem}
  \begin{itemize}
  \item In the language of \(M_\kappa\)-polyhedral complexes the link \(\lk(x, X)\) is a \(M_1\)-polyhedral complex, whenever \(x\) is a vertex of \(X\). For more details consider \textcite[Sectoin~I.7]{MR1744486}. Although all the cells of \(\lk(x,X)\) consist of simplices, it might happen that \(\lk(x,X)\) is not a simplicial complex. An example is given below.\todo{give specific example}
  \item The definition of a cube complex is not the standard one given, but that of a cubical complex (c.\,f.~\cite[Def.~I.7.37]{MR1744486}). Normally, one is satisfied, if the \(p_\lambda\) are injective on the interior of each face separately. However, as \textcite[Thm.~C.4]{MR3029427} has shown in the case of CAT(0) cube complexes the two definitions are equivalent, hence we will adopt it from the start.
  \item The above definition ascertains that two cubes either interesect in a common face or have an empty intersection. In this sense, they are completely analogous to simplicial complexes. 
  \end{itemize}
\end{rem}

\begin{bsp}
  \todo{Enter examples of polyhedral complexes and links.}
\end{bsp}

% \begin{defin}
%   \label{def:epsilon}
%   \begin{itemize}
%   \item \(c\colon [a, b] \to K\) is called \emph{piecewise geodesic}, if there exists a subdivision \(a = t_0 \leq \dots \leq t_k = b\) and geodesics \(c_i\colon [t_{i-1}, t_i] \to C_{\lambda(i)}\), such that \(c|_{[t_{i-1}, t_i]} = p_{\lambda(i)} \circ c_i\). The \emph{length} of a piecewise geodesic path is given by
%     \begin{align*}
%       l(c) \coloneqq \sum_{i=1}^n l(c_i).
%     \end{align*}
%   \item Let
%     \begin{align*}
%       &d\colon K \times K \to \R_{\geq 0} \cup \{\infty\},\\
%       &d(x,y) \coloneqq \inf\{l(c) \mid c(a) = x,\ c(b) = y \text{ and } c \text{ piecewiese geodesic}\}
%     \end{align*}
%     and
%     \begin{align}
%       \epsilon(x) \coloneqq \{d(x, C \setminus \st(x)) \mid C \subset K \text{ a cell and } x \in C\}. \label{eq:epsilon}
%     \end{align}
%   \end{itemize}
% \end{defin}

%%% Local Variables:
%%% mode: latex
%%% TeX-master: "../Master"
%%% End:

%  LocalWords:  xM
