\section{CAT(0) cube complexes}
\label{sec:complexes}

\subsection{Metric preliminaries}
\label{sec:metric}
This section is concerned with basic metric properties of spaces and their connection to CAT(\(\kappa\)) spaces. The exposition follows closely~\cite{MR1744486}.

\begin{defin}~
  \begin{itemize}
  \item Let \(X\neq \varnothing\) be a set. \(d\colon X \times X \to \R_{\geq 0} (\cup \{\infty\})\) is called a \emph{metric}, if for all \(x,y,z \in X\)
    \begin{enumerate}
    \item \(d(x,y) = 0\) if and only if \(x = y\),
    \item \(d(x,y) = d(y,x)\) and
    \item \(d(x,y) + d(y,z) \geq d(x,z)\).
    \end{enumerate}
    The pair \((X,d)\) is called a \emph{metric space}.
  \item A \emph{geodesic} from \(x\) to \(y\) (\(x,y \in X\)) is a map \(c\colon [a,b] \to X\), such that \(c(a) = x\), \(c(b) = y\) and such that there exists a constant \(\lambda > 0\) with
    \begin{align*}
      d(c(t), c(t')) = \lambda \cdot |t - t'| 
    \end{align*}
    for all \(t,t' \in [a,b]\).
  \item \((X,d)\) is called \emph{\(r\)-geodesic} (\(r > 0\)), if \(d(x,y) < r\) implies that ther is a geodesic joining \(x\) and \(y\). It is called \emph{uniquely \(r\)-geodesic}, if this geodesic is unique. \((X,d)\) is called \emph{(uniquely) geodesic}, if it is (uniquely) \(r\)-geodesic for all \(r > 0\).
  \item The \emph{length} of a curve \(c\colon [a,b] \to X\) is defined as
    \begin{align*}
      l(c) \coloneqq \left\{\sum_{i=1}^n d(c(t_i), c(t_{i-1})) \relmid a = t_0 \leq t_0 \leq \dots \leq t_n = b \text{ a subdivision} \right\}.
    \end{align*}
    \(c\) is called \emph{rectifiable}, if \(l(c) < \infty\).
  \item \(d\) is called a \emph{length metric}, if
    \begin{align*}
      d(x,y) = \inf \{l(c) \mid c \text{ rectifiable joining } x \text{ and } y\}
    \end{align*}
    for arbitrary \(x,y \in X\), where \(\inf\varnothing = \infty\). In this case \((X,d)\) is called a \emph{length space}.
  \end{itemize}
\end{defin}

\begin{rem}
  In this thesis the metric is allowed to take the value \(\infty\). This is a bit unusual, but convenient in the setting of length spaces, where it might be that a path does not exist.
\end{rem}

\begin{bsp}
  \todo{fill in some examples}
  \begin{enumerate}
  \item \((\E^n, d_0)\), where \(\E^n = \R^n\) and \(d_0(x,y) \coloneqq \|x - y\|_2\), is a length spaces which is uniquely geodesic. However, \(\E^n \setminus {0}\) is still a length space, but no longer geodesic. If a whole cube \(C= [0,1]^n\) is removed, \(\E^n \setminus C\) is no longer a length space, with regard to the induced metric. 
  \item \((\sphere^n, d_\kappa)\), \(\kappa > 0\), is the unit sphere in \(\E^{n+1}\) together with the metric \(d_{\kappa}(x,y) = \frac{1}{\sqrt{\kappa}} \cdot \angle (x,y)\). This space is uniquely \(\frac{\pi}{\sqrt{\kappa}}\)-geodesic.
  \item \((\hyperbole^n, d_\kappa)\), \(\kappa < 0\), is the hyperbolic \(n\)-space
    \begin{align*}
      \hyperbole^n & = \left\{x = (x_0, \dots, x_n)\in \R^{n+1} \relmid \langle x, x \rangle_{1,n} = -x_0^2 + \sum_{i=1}^n x_i^2 = -1 \right\} \quad \text{with}\\
      d_\kappa(x,y) & = \frac{1}{\sqrt{-\kappa}} \cdot \operatorname{arcosh} \langle x,y \rangle_{1,n}.
    \end{align*}
    This space is also a length space and uniquely geodesic.
  \end{enumerate}
\end{bsp}

\begin{defin}
  Let
  \begin{align*}
    M_\kappa^n \coloneqq
    \begin{cases}
      \sphere^n & \kappa > 0\\
      \E^n & \kappa = 0\\
      \hyperbole^n & \kappa < 0
    \end{cases} \text{ and }
    D_\kappa =
    \begin{cases}
      \frac{\pi}{\sqrt{\kappa}} & \kappa > 0\\
      \infty & \kappa \leq 0
    \end{cases}.
  \end{align*}
  Then \((M_\kappa^n, d_\kappa)\) is called the \emph{model space of curvature \(\kappa\)}. \(D_\kappa\) denotes its \emph{diameter}.
\end{defin}

\begin{rem}
  As a reference for the above implicit claims~\cite[Sec.\ I.2, I.6]{MR1744486} can be used.
\end{rem}

\begin{defin}\
  \begin{itemize}
  \item Let \((X,d)\) be a metric space. A \emph{geodesic triangle} \(\Delta \subset X\) consists of three points \(p,q,r \in X\), its \emph{vertices}, together with a choice of three geodesic segments \([p,q], [q,r], [r, p]\) joining them, its \emph{edges} (The choice of geodesics might not be unique). If necessary, the notation \(\Delta = \Delta(p,q,r)\) or \(\Delta = \Delta([p,q], [q,r], [r,p])\) will be used. The first case is a slight abuse of notation, as the three vertices might not determine the triangle.
  \item A \emph{comparison triangel} in \(M_\kappa^2\) for \(\Delta = \Delta(p,q,r)\) is a choice of three points \(\bar p, \bar q, \bar r \in M_\kappa^2\) such that \(d_\kappa(\bar p, \bar q) = d(p, q)\), \(d_\kappa(\bar q, \bar r), = d(q, r)\) and \(d_\kappa(\bar r, \bar p) = d(r, p)\). It will be denoted by \(\bar \Delta = \Delta(\bar p, \bar q, \bar r)\). It exists wehnever the perimeter \(p(\Delta) \coloneqq d(p,q) + d(q,r) + d(r,p) < 2D_\kappa\) \cite[cf.][Sec.\ I.2]{MR1744486}.
  \item \(\Delta\) satisfies the \emph{CAT(\(\kappa\)) inequality}, if \(d(x,y) \leq d_\kappa(\bar x, \bar y)\) for any \(x, y \in \Delta\) and where \(\bar x \in \bar \Delta \subset M_\kappa^2\) is the unique point having the same distance to the two endpoints of the geodesic segment it lies on as \(x\) on its geodesic.
  \item \(X\) is called a \emph{CAT(\(\kappa\)) space}, if \(X\) is \(D_\kappa\)-geodesic and if each geodesic triangle \(\Delta\) with \(p(\Delta) < 2 D_\kappa\) satisfies the CAT(\(\kappa\)) inequality.
  \item \(X\) is called \emph{of curvature \(\leq \kappa\)}, if it is locally a CAT(\(\kappa\)) space, i.\,e.\ for each \(x \in X\) there exists an \(r > 0\), such that B(x,r) together with the induced metric is a CAT(\(\kappa\)) space.
  \end{itemize}
\end{defin}

%%% Local Variables:
%%% mode: latex
%%% TeX-master: "../Master"
%%% End:
