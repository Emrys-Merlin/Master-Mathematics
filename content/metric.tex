\section{CAT(0) cube complexes}
\label{sec:complexes}

\subsection{Preliminaries on metric and CAT(0) spaces}
\label{sec:metric}
This section is concerned with basic metric properties of spaces and their connection to CAT(\(0\)) spaces. The exposition follows closely~\cite{MR1744486}.

\begin{defin}~
  \begin{itemize}
  % \item Let \(X\neq \varnothing\) be a set. \(d\colon X \times X \to \R_{\geq 0} (\cup \{\infty\})\) is called a \emph{metric}, if for all \(x,y,z \in X\)
  %   \begin{enumerate}
  %   \item \(d(x,y) = 0\) if and only if \(x = y\),
  %   \item \(d(x,y) = d(y,x)\) and
  %   \item \(d(x,y) + d(y,z) \geq d(x,z)\).
  %   \end{enumerate}
  %   The pair \((X,d)\) is called a \emph{metric space}.
  \item Let \((X, d)\) be a metric space. A \emph{geodesic} from \(x\) to \(y\) (\(x,y \in X\)) is a map \(c\colon [a,b] \to X\), such that \(c(a) = x\), \(c(b) = y\) and such that there exists a constant \(\lambda > 0\) with
    \begin{align*}
      d(c(t), c(t')) = \lambda \cdot |t - t'| 
    \end{align*}
    for all \(t,t' \in [a,b]\).
  \item The pair \((X,d)\) is called \emph{\(r\)-geodesic} (\(r > 0\)), if \(d(x,y) < r\) implies that there is a geodesic joining \(x\) and \(y\). It is called \emph{uniquely \(r\)-geodesic}, if this geodesic is unique. \((X,d)\) is called \emph{(uniquely) geodesic}, if it is (uniquely) \(r\)-geodesic for all \(r > 0\).
  % \item The \emph{length} of a curve \(c\colon [a,b] \to X\) is defined as
  %   \begin{align*}
  %     l(c) \coloneqq \left\{\sum_{i=1}^n d(c(t_i), c(t_{i-1})) \relmid a = t_0 \leq t_0 \leq \dots \leq t_n = b \text{ a subdivision} \right\}.
  %   \end{align*}
  %   \(c\) is called \emph{rectifiable}, if \(l(c) < \infty\).
  % \item \(d\) is called a \emph{length metric}, if
  %   \begin{align*}
  %     d(x,y) = \inf \{l(c) \mid c \text{ rectifiable joining } x \text{ and } y\}
  %   \end{align*}
  %   for arbitrary \(x,y \in X\), where \(\inf\varnothing = \infty\). In this case \((X,d)\) is called a \emph{length space}.
  \end{itemize}
\end{defin}

% \begin{rem}
%   In this thesis the metric is allowed to take the value \(\infty\). This is a bit unusual, but convenient in the setting of length spaces, where it might be that a path does not exist.
% \end{rem}

\begin{bsp}
  The pair \((\E^n, d_0)\), where \(\E^n = \R^n\) and \(d_0(x,y) \coloneqq \|x - y\|_2\), is a length space which is uniquely geodesic. The space \(\E^n \setminus {0}\) is still a length space, but no longer geodesic, since every pair of antipodal points can no longer be joined by a line segment.. If a whole cube \(C= [0,1]^n\) (or any other subset with non-empty interior) is removed, \(\E^n \setminus C\) is no longer a length space with regard to the induced metric. 
  % \item \((\sphere^n, d_\kappa)\), \(\kappa > 0\), is the unit sphere in \(\E^{n+1}\) together with the metric \(d_{\kappa}(x,y) = \frac{1}{\sqrt{\kappa}} \cdot \angle (x,y)\). This space is uniquely \(\frac{\pi}{\sqrt{\kappa}}\)-geodesic.
  % \item \((\hyperbole^n, d_\kappa)\), \(\kappa < 0\), is the hyperbolic \(n\)-space
  %   \begin{align*}
  %     \hyperbole^n & = \left\{x = (x_0, \dots, x_n)\in \R^{n+1} \relmid \langle x, x \rangle_{1,n} = -x_0^2 + \sum_{i=1}^n x_i^2 = -1 \right\} \quad \text{with}\\
  %     d_\kappa(x,y) & = \frac{1}{\sqrt{-\kappa}} \cdot \operatorname{arcosh} \langle x,y \rangle_{1,n}.
  %   \end{align*}
  %   This space is also a length space and uniquely geodesic.
\end{bsp}

% \begin{defin}
%   Let
%   \begin{align*}
%     M_\kappa^n \coloneqq
%     \begin{cases}
%       \sphere^n & \kappa > 0\\
%       \E^n & \kappa = 0\\
%       \hyperbole^n & \kappa < 0
%     \end{cases} \text{ and }
%     D_\kappa =
%     \begin{cases}
%       \frac{\pi}{\sqrt{\kappa}} & \kappa > 0\\
%       \infty & \kappa \leq 0
%     \end{cases}.
%   \end{align*}
%   Then \((M_\kappa^n, d_\kappa)\) is called the \emph{model space of curvature \(\kappa\)}. \(D_\kappa\) denotes its \emph{diameter}.
% \end{defin}

% \begin{rem}
%   As a reference for the above implicit claims~\cite[Sec.\ I.2, I.6]{MR1744486} can be used.
% \end{rem}

\begin{defin}[CAT(0) and non-positive curvature spaces]
  \label{def:cat}
  \begin{itemize}
  \item Let \((X,d)\) be a metric space. A \emph{geodesic triangle} \(\Delta \subset X\) consists of three points \(p,q,r \in X\), its \emph{vertices}, together with a choice of three geodesic segments \([p,q], [q,r], [r, p]\) joining them, its \emph{edges} (The choice of geodesics might not be unique). If necessary, the notation \(\Delta = \Delta(p,q,r)\) or \(\Delta = \Delta([p,q], [q,r], [r,p])\) will be used. The first case is a slight abuse of notation, as the three vertices might not determine the triangle.
  \item A \emph{comparison triangle} in \(\E^2\) for \(\Delta = \Delta(p,q,r)\) is a choice of three points \(\bar p, \bar q, \bar r \in \E^2\) such that \(\|\bar p- \bar q\| = d(p, q)\), \(\|\bar q- \bar r\|, = d(q, r)\) and \(\|\bar r- \bar p\| = d(r, p)\). It will be denoted by \(\bar \Delta = \Delta(\bar p, \bar q, \bar r)\). Such a comparison triangle always exists \cite[c.\,f.][Sec.\ I.2]{MR1744486}.
  \item \(\Delta\) satisfies the \emph{CAT(\(0\)) inequality}, if \(d(x,y) \leq \|\bar x- \bar y\|\) for any \(x, y \in \Delta\) and where \(\bar x \in \bar \Delta \subset \E^2\) is the unique point having the same distance to the two endpoints of the geodesic segment it lies on as \(x\) on its geodesic.
  \item \(X\) is called a \emph{CAT(\(0\)) space}, if \(X\) is geodesic and if each geodesic triangle \(\Delta\) satisfies the CAT(\(0\)) inequality.
  \item \(X\) is called \emph{of curvature \(\leq 0\)} or \emph{non-positively curved}, if it is locally a CAT(\(0\)) space, i.\,e.\ for each \(x \in X\) there exists an \(r > 0\), such that \(B(x,r)\) together with the induced metric is a CAT(\(0\)) space.
  \end{itemize}
\end{defin}

\begin{bsp}
  \todo{Elaborate CAT examples}
  \begin{enumerate}
 % \item CAT(\(\kappa\)) contained etc
  \item \(\E^n\) without a diamond. Still locally CAT(0), but not globally
  \item Trees
  % \item \(\bigvee_{n \in \N} (\sphere_n, x_n)\), where \(\sphere_n = \frac{1}{\sqrt{n}} \sphere^1
    % \cong M_n^1\) and \(x_n \in \sphere_n\). Not CAT(\(\kappa\)) for any \(\kappa\), not even locally.
  \end{enumerate}
\end{bsp}


% \begin{defin}
%   \label{def:epsilon}
%   \begin{itemize}
%   \item \(c\colon [a, b] \to K\) is called \emph{piecewise geodesic}, if there exists a subdivision \(a = t_0 \leq \dots \leq t_k = b\) and geodesics \(c_i\colon [t_{i-1}, t_i] \to C_{\lambda(i)}\), such that \(c|_{[t_{i-1}, t_i]} = p_{\lambda(i)} \circ c_i\). The \emph{length} of a piecewise geodesic path is given by
%     \begin{align*}
%       l(c) \coloneqq \sum_{i=1}^n l(c_i).
%     \end{align*}
%   \item Let
%     \begin{align*}
%       &d\colon K \times K \to \R_{\geq 0} \cup \{\infty\},\\
%       &d(x,y) \coloneqq \inf\{l(c) \mid c(a) = x,\ c(b) = y \text{ and } c \text{ piecewiese geodesic}\}
%     \end{align*}
%     and
%     \begin{align}
%       \epsilon(x) \coloneqq \{d(x, C \setminus \st(x)) \mid C \subset K \text{ a cell and } x \in C\}. \label{eq:epsilon}
%     \end{align}
%   \end{itemize}
% \end{defin}

This section will contain some important facts about general CAT(0) spaces. Where no proof is given, it can be found in \cite{MR1744486}. Recall that the general definition of a CAT(\(\kappa\)) space was given in Definition~\ref{def:cat}. 

\begin{prop}[{\cite[Prop II.1.4]{MR1744486}}]
  Let \(X\) be a CAT(0) space. Then
  \begin{enumerate}
  \item \(X\) is uniquely geodesic and
  \item \(X\) is contractible.
  \end{enumerate}
\end{prop}

\begin{defin}
  Let \(X,Y\) be metric spaces. \(\phi \colon X \to Y\) is called
  \begin{itemize}
  \item an \emph{isometric embedding}, if \(d_X(x,y) = d_Y(\phi(x), \phi(y))\) for any two \(x,y \in X\),
  \item an \emph{isometry}, if it is an isometric embedding and surjective (and hence bijective) and
  \item a \emph{local isometry}, if for each \(x \in X\) there exists an open neighborhood \(U \subset X\) containing \(x\), such that \(\phi|_U \colon U \to \phi(U)\) is an isometry.
  \end{itemize}
\end{defin}

\begin{prop}[{\cite[Propositions 1 \& 2]{Rolli2012}}]
  Let \(X,Y\) be geodesic spaces and let \(Y\) be CAT(0). Then every local isometry \(\phi \colon X \to Y\) is an isometric embedding. In particular, every local geodesic is a geodesic.
\end{prop}

On CAT(0) spaces one regularly defines a boundary via identifying certain geodesic rays. This so called \emph{visual boundary} will play a minor role in this thesis, but we will still need it. 

\begin{defin}[Visual boundary, {\cite[Sec.~II.8]{MR1744486}}]
  Let \(\gamma_i \colon [0, \infty) \to X\) be two geodesic rays. \(\gamma_1 \sim \gamma_2\) if and only if there exists a constant \(K > 0 \) such that \(d(\gamma_1(t), \gamma_2(t)) < K\) for all \(t \geq 0\). The set of equivalence classes of this relation \(\partial_\sphericalangle X\) is called the \emph{visual boundary of \(X\)}.

  Clearly, each group action on \(X\) by isometries extends to an action on \(\partial_\sphericalangle X\).
\end{defin}

\begin{rem}
  \(X \sqcup \partial_{\sphericalangle}X\) can be topologized in a way that it agrees with the topology induced by the metric on \(X\). If \(X\) is locally compact \(X \sqcup \partial_\sphericalangle X\) becomes compact (c.\,f.~\cite[Sec.~II.8]{MR1744486}). In general this can fail.
\end{rem}

\begin{bsp}
  \todo{examples for visual boundary}
\end{bsp}


%%% Local Variables:
%%% mode: latex
%%% TeX-master: "../Master"
%%% End:

%  LocalWords:  xM
