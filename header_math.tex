% Hier sollten meine wichtigsten Mathe Befehle zu finden sein.
% Weiter unten sind die spezielleren Befehle, die je nach dem
% vielleicht auskommentiert oder gelöscht werden sollten.

\usepackage{amssymb}

\usepackage[]{amsmath}
\numberwithin{figure}{section}
\numberwithin{equation}{section}
\numberwithin{table}{section}

\usepackage{amsthm}
\usepackage{mathtools}

%tikz
\usepackage{tikz}
\usetikzlibrary{shadows}
\tikzset{node distance=3cm, auto}
\usepackage{pgfplots}

%Abkürzungen für Standardzahlmengen
\let\C\relax
\NewDocumentCommand\R{}{\mathbb{R}}
\NewDocumentCommand\Q{}{\mathbb{Q}}
\NewDocumentCommand\N{}{\mathbb{N}}
\NewDocumentCommand\C{}{\mathbb{C}}
\NewDocumentCommand\Z{}{\mathbb{Z}}
\NewDocumentCommand\A{}{\mathcal{A}}
\NewDocumentCommand\K{}{\mathbb{K}}
\NewDocumentCommand\p{}{\mathbb{P}}
\NewDocumentCommand\h{}{\mathbb{H}}
\NewDocumentCommand\F{}{\mathcal{F}}
\NewDocumentCommand\D{}{\mathcal{D}}
\NewDocumentCommand\lie{}{\mathcal{L}}
\NewDocumentCommand\jo{}{\mathfrak{J}}
\NewDocumentCommand\hol{}{\mathcal{O}}
\NewDocumentCommand\mer{}{\mathcal{M}}
\NewDocumentCommand\diff{}{\mathcal{E}}

% ein Paar Abbildungssachen.
\NewDocumentCommand\Supp{}{\operatorname{Supp}}
\NewDocumentCommand\id{}{\operatorname{id}}
\NewDocumentCommand\supp{}{\operatorname{supp}}
\NewDocumentCommand\rank{}{\operatorname{rank}}
\NewDocumentCommand\tr{}{\operatorname{tr}}
\NewDocumentCommand\ord{}{\operatorname{ord}}

%Pfeile und Stuff
\NewDocumentCommand\Ra{}{\Rightarrow}
\NewDocumentCommand\La{}{\Leftarrow}
\NewDocumentCommand\LRa{}{\Leftrightarrow}
\NewDocumentCommand\ra{}{\rightarrow}
\NewDocumentCommand\la{}{\leftarrow}

% Faktorräume
\NewDocumentCommand\quot{m m}{\left .\raisebox{.2em}{$#1$}\middle/\raisebox{-.2em}{$#2$}\right .}

% richtiges epsilon und phi
\let\epsilon\relax
\NewDocumentCommand\epsilon{}{\varepsilon}
\let\phi\relax
\NewDocumentCommand\phi{}{\varphi}

% Differential
\let\d\relax
\NewDocumentCommand\d{ O{} }{\operatorname{d}\hspace{-0.1em}#1}

%amsthm
%\theoremstyle{plain}
\theoremstyle{definition}
\newtheorem{thm}{Theorem}[section]
\newtheorem{lemma}[thm]{Lemma}
\newtheorem{prop}[thm]{Proposition}
\newtheorem{cor}[thm]{Corollary}
%\theoremstyle{definition}
\newtheorem{defin}[thm]{Definition}
\newtheorem{bsp}[thm]{Example}
%\theoremstyle{remark}
\newtheorem{rem}[thm]{Remark}

% % Differentialgeometrie (Riemannsche Geometrie)
% \NewDocumentCommand\tang{ O{p} O{M}}{T_{#1}#2}
% \NewDocumentCommand\cotang{ O{p} O{M}}{T^\ast_{#1}#2}
% \NewDocumentCommand\del{ O{i} O{x} O{} }{\frac{\partial {#3}}{\partial {#2}^{#1}}}
% \NewDocumentCommand\delat{ O{p} O{i} O{x} O{} }{\left . \del[#2][#3][#4] \right |_{#1}}
% \NewDocumentCommand\christ{O{i} O{j} O{k} }{ \Gamma_{#1 #2}^{#3} }
% \NewDocumentCommand\g{m m}{\langle #1, #2 \rangle}
% \NewDocumentCommand\diam{}{\operatorname{diam}}
% \NewDocumentCommand\ric{}{\operatorname{ric}}
% \NewDocumentCommand\scal{}{\operatorname{scal}}
% \NewDocumentCommand\arsinh{}{\operatorname{arsinh}}

% % Bachelor-Arbeit
% \NewDocumentCommand\im{}{\operatorname{im}}
% \NewDocumentCommand\sm{}{\operatorname{sm}}
% \NewDocumentCommand\Reg{}{\operatorname{Reg}}
% \NewDocumentCommand\be{}{\mathfrak{B}}
% \NewDocumentCommand\pe{}{\mathfrak{P}}
% \NewDocumentCommand\res{}{\operatorname{res}}
% \let\S\relax
% \NewDocumentCommand\S{}{\mathcal{S}}
% \let\P\relax
% \NewDocumentCommand\P{}{\mathbb{P}}
% \NewDocumentCommand\Fix{}{\operatorname{Fix}}
% \NewDocumentCommand\Mat{}{\operatorname{Mat}}
% \NewDocumentCommand\SL{}{\operatorname{SL}}
% \NewDocumentCommand\GL{}{\operatorname{GL}}
% \NewDocumentCommand\PSL{}{\operatorname{PSL}}
% \let\Re\relax
% \NewDocumentCommand\Re{}{\operatorname{Re}}
% \let\Im\relax
% \NewDocumentCommand\Im{}{\operatorname{Im}}
% \NewDocumentCommand\Div{}{\operatorname{Div}}
% \NewDocumentCommand\Aut{}{\operatorname{Aut}}
% \NewDocumentCommand\Deck{}{\operatorname{Deck}}
% \NewDocumentCommand\runge{}{\mathfrak{h}}
% \NewDocumentCommand\fu{}{\mathfrak{U}}
% \NewDocumentCommand\dist{}{\mathcal{D}}
